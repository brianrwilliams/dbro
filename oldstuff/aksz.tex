\documentclass{article}
\usepackage{amssymb, amsmath, amsthm, mathtools, bbm, tikz-cd,stmaryrd,enumerate}
%-------------------
\usepackage{fancyhdr}
\pagestyle{fancy}
\fancyhf{}
\fancyhead[L]{\leftmark}
\fancyhead[R]{\thepage}
%----------------
\newcommand{\TT}{{T\oplus T^*}}
\newcommand{\JJ}{\mathcal{J}}
\newcommand{\KK}{\mathcal{K}}
\newcommand{\GG}{\mathcal{G}}
\newcommand{\Cc}{\mathbf{C}}
\newcommand{\RR}{\mathbb{R}}
\newcommand{\XX}{\mathfrak{X}}
\newcommand{\HH}{\mathcal{H}}
\newcommand{\FF}{\mathcal{F}}
\newcommand{\QQ}{\mathcal{Q}}
\newcommand{\LL}{\mathcal{L}}
%----------------------------------------
\newcommand{\PP}{\mathrm{P}}
\newcommand{\PPt}{\tilde{\mathrm{P}}}
\newcommand{\id}{\mathbbm{1}}
\newcommand{\nlr}{\overset{\leftrightarrow}{\n}}
\newcommand{\lc}{\mathring{\n}}
\newcommand{\im}{\mathrm{Im}\,}
\newcommand{\Ker}{\mathrm{Ker}\,}
\newcommand{\Lie}{\mathcal{L}}
\newcommand{\PS}{\mathcal{P}}
\newcommand{\ap}{\alpha}
\newcommand{\bt}{\beta}
\def\w{\wedge}
\newcommand{\p}{\partial}
\newcommand{\pt}{\tilde{\partial}}
\newcommand{\xt}{{\tilde{x}}}
\newcommand{\n}{\nabla}
\newcommand{\rd}{\mathrm{d}}
\newcommand{\PH}{(\PS,\eta,\omega)}
\newcommand{\Lt}{\tl{L}}
\newcommand{\Lb}{\mathbf{L}}
\newcommand{\s}{\mathbf{s}}
\newcommand{\se}{\Gamma}
\newcommand{\Endo}{\text{End}}
\newcommand{\ellt}{{\tl{\ell}}}
\newcommand{\ot}{{1/2}}
\newcommand{\inv}{{-1}}
\newcommand{\Aa}{\mathcal{A}}
\newcommand{\la}{\langle}
\newcommand{\ra}{\rangle}
\newcommand{\lara}{\la\ ,\ \ra}
\newcommand{\brac}{[\ ,\ ]}
\newcommand{\bl}{[\![}
\newcommand{\br}{]\!]}
\newcommand{\bracd}{\bl \ ,\ \br}
\newcommand{\yt}{\tl{y}}
\newcommand{\zt}{\tl{z}}
\newcommand{\tth}{\tl{\theta}}
\newcommand{\kk}{\mathrm{k}}
\newcommand{\bu}{\bullet}


\def\d{{\rm d}}

\def\brian{\textcolor{blue}{BW: }\textcolor{blue}}
\def\david{\textcolor{green}{DS: }\textcolor{green}}
%----------------------------------------
\def\gld{generalized Lie derivative }
\def\glds{generalized Lie derivatives }
\def\tl{\tilde}

% These will be typeset in italics
\newtheorem{theorem}{Theorem}[section]
\newtheorem{proposition}[theorem]{Proposition}
\newtheorem{lemma}[theorem]{Lemma}
\newtheorem{corollary}[theorem]{Corollary}
\newtheorem{fact}[theorem]{Fact}
\newtheorem*{theorem*}{Theorem}
\newtheorem*{lemma*}{Lemma}
\newtheorem*{proposition*}{Proposition}
\newtheorem{Rem}[theorem]{Remark}

% These will be typeset in Roman
\theoremstyle{definition}
\newtheorem{Def}[theorem]{Definition}
\newtheorem{Conj}[theorem]{Conjecture}

\theoremstyle{definition}
\newtheorem*{notation*}{Notation}
\newtheorem*{definition*}{Definition}

\theoremstyle{remark}
\newtheorem*{remark*}{Remark}
\newtheorem{Ex}[theorem]{Example}
\newtheorem{question}[theorem]{Question}
\newenvironment{claim}[1]{\par\noindent\underline{Claim:}\space#1}{}
\newenvironment{claimproof}[1]{\par\noindent\underline{Proof:}\space#1}{\hfill $[acksquare$}

\input xy

\xyoption{all}

\DeclareMathOperator{\End}{End}
\DeclareMathOperator{\rk}{rk}

\begin{document}
\section{AKSZ stuff}

\subsection{Two-dimensional TFT from GC structures}

In this section, we summarize the work of Kapustin-Li in the AKSZ formalism with an eye towards modification in the para geometric setting. 

\subsubsection{A recollection of generalized complex geometry}

Recall, an exact Courant algebroid on a smooth manifold $X$ is determined, up to equivalence, by a class $H \in H^1(X, \Omega^2_{cl})$, called its Severa class, or $H$-flux. 
The underlying vector bundle of an exact Courant algebroid is $TX \oplus T^*X$, and the $H$-flux deforms the standard Dorfman bracket. 

An almost generalized complex structure on a manifold $X$ is the data of a smooth bundle map
\[
\JJ : TX \oplus T^*X \to TX \oplus T^*X
\]
satisfying $\JJ^2 = - \id$ and $\langle\JJ u, \JJ v\rangle = \langle u,v\rangle$, where $\langle-,-\rangle$ denotes the obvious pairing between the tangent and cotangent bundle.
Given an almost generalized complex structure, we denote by $L \subset T X^{\Cc} \oplus T^* X^{\Cc}$ the complex $+i$ eigenbundle of $\JJ$. 

Fix an exact Courant algebroid on $X$ with class $H$. 
An almost generalized complex structure on $X$ is an $H$-{\em twisted generalized complex structure} if the subbundle $L$ is preserved under the $H$-twisted Dorfman bracket. 
An $H$-twisted generalized complex structure is equivalent to an $H$-twisted complex Dirac structure $L$ \footnote{A complex Dirac structure is a maximally isotropic involutive subbundle of the exact complex Courant algebroid $TX^{\Cc} \oplus T^*X^{\Cc}$} satisfying $L \cap \Bar{L} = \{0\}$. 
When $H = 0$, one simply calls this a generalized complex structure. 

\begin{Ex}\label{ex: complexstr}
Every almost complex structure on $X$ determines an almost generalized complex structure on $X$. 
It is a generalized complex structure if the almost complex structure is integrable.

In fact, consider the almost generalized complex structure
\begin{align*}
\JJ=
\begin{pmatrix}
I & 0 \\
0 & -I^*
\end{pmatrix},
\end{align*}
where $I : TX \to TX$ is an almost complex structure, and $I^*$ is the dual bundle map.
The $+i$ eigenbundle is $L=T^{(1,0)}\oplus T^{*(0,1)}$ and the $-i$ eigenbundle is $\bar{L}=T^{(0,1)}\oplus T^{*(1,0)}$. 
Involutivity of $L$ under $\brac_H$ is equivalent to the condition
\begin{align*}
[X+\bar{\ap},Y+\bar{\bt}]=[X,Y]+\Lie_X\bar{\bt}-\imath_Y\rd \bar{\ap}+H(X,Y),
\end{align*}
where $X,Y\in \XX^{(1,0)}$ and $\bar{\ap},\bar{\bt}\in \Omega^{(0,1)}$.
This implies that one must have have $[X,Y]\subset T^{(1,0)}$, meaning $I$ must be a complex structure. 
Moreover, inspecting the cotangent component of the above and splitting $\rd=\p+\bar{\p}$, the non-zero components are
\begin{align*}
\imath_X(\p\bar{\bt})-\imath_Y(\p \bar{\ap})+H(X,Y),
\end{align*}
where $\imath_X\p\bar{\bt}-\imath_Y\p \bar{\ap}$ is always in $T^{*(0,1)}$ and so $H(X,Y)$ must be in $T^{*(0,1)}$ as well, i.e. the $(3,0)$ component of $H$ must vanish. The same argument for $\bar{L}$ gives $H^{(0,3)}=0$.
\end{Ex}

\begin{Ex}
Suppose $\omega \in \Omega^2(X)$ is a nondegenerate $2$-form.
Then, $\omega$ determines an almost generalized complex structure on $X$.
It is a generalized complex structure for $H = \d \omega$.
In particular, every symplectic structure determines a generalized complex structure for $H = 0$.

Indeed, define the almost generalized complex structure
\begin{align*}
\JJ=
\begin{pmatrix}
0 & -\omega^{-1} \\
\omega & 0
\end{pmatrix},
\end{align*}
With eigenbundles given by $X\pm i\omega(X)$ for any $X\in \XX$. Checking the involutivity of these under $\brac_H$:
\begin{align*}
[X\pm i\omega(X),Y\pm i\omega(Y)]=[X,Y]\pm i \omega([X,Y])\pm i\rd \omega (X,Y)+H(X,Y),
\end{align*}
where some identities for Lie derivatives etc. have been used. Clearly $[X,Y]\pm i \omega([X,Y])$ is of the desired form and the remaining terms are in $\Omega^1$ so they have to vanish (cannot be of the form $Z\pm i \omega(Z)$). This gives
\begin{align*}
i\rd \omega (X,Y)+H(X,Y)=0 
\end{align*}
for all $X,Y$. 
\end{Ex}

More generally, one can speak of (almost) generalized complex structures in {\em any} (possibly non exact) Courant algebroid. 
If $E$ is the underlying vector bundle of the Courant algebroid, then an almost generalized complex structure is a bundle map $\JJ : E \to E$ satisfying the same conditions as above, namely $\JJ^2 = - \id$ and $\langle\JJ u, \JJ v\rangle = \langle u,v\rangle$.
It is a generalized complex structure if it is integrable for the bracket defining the Courant algebroid. 

Analogous to the exact case, we have the following equivalent characterization of generalized complex structures in $E$. 

\begin{proposition}
A generalized complex structure in a Courant algebroid $E$ is equivalent to a Dirac structure $L$ in the complex Courant algebroid $E^{\Cc}$ satisfying $L \cap \Bar{L} = 0$. 
\end{proposition}

\brian{Should I define what a Dirac structure in a general CA is?}

\subsubsection{The associated Lie algebroid and cohomology}

To any (twisted) generalized complex structure there exists a naturally associated (complex) Lie algebroid defined as follows.

Let $\JJ$ be an $H$-twisted generalized complex structure and $\Bar{L} \subset T X^{\Cc} \oplus T^* X^{\Cc}$ the complex $-i$ eigenbundle.
\brian{I think we want to use the $-i$ eigenbundle.}
While the Dorfman bracket does {\em not} define a Lie bracket on $T X^{\Cc} \oplus T^* X^{\Cc}$ it does define one on the subbundle $L$.
Indeed, by definition $\Bar{L}$ is isotropic and integrable, so the Jacobi identity is satisfied. 
The anchor map $a : \Bar{L} \to T X^{\Cc}$ is given by the restriction of the natural projection $T X^{\Cc} \oplus T^* X^{\Cc} \to T X^{\Cc}$ to $L$. 
We denote this Lie algebroid by $\Bar{L}_{\JJ, H}$. 

We define {\em cohomology} of an $H$-twisted generalized complex structure $\JJ$ to be the Lie algebroid cohomology of $L_{\JJ, H}$.

\begin{Ex}
When the generalized complex structure $\JJ$ is defined using an ordinary complex structure, as in Example \ref{ex: complexstr} the resulting Lie algebroid is given by 
\[
\Bar{L}_{\JJ, H} = T^{0,1} X \to T X^{\Cc} .
\]
The cohomology of the twisted generalized complex structure is equal to the Dolbeault cohomology $H^{0,*}(X)$. 
\end{Ex}

More generally, there is a (complex) Lie algebroid associated to a generalized complex structure $\JJ$ in any (possibly non exact) Courant algebroid $E$. 
Again, we define $\Bar{L} \subset E^{\Cc} $ to be the $-i$ eigenspace of $\JJ$. 
The same argument as in the exact case shows that $\Bar{L}$ has a natural Lie algebroid structure, where bracket is given by restricting the bracket of $E$ defining the Courant algebroid structure, and the anchor map is given by the composition
\[
a : \Bar{L} \hookrightarrow E^{\Cc} \to T X^{\Cc} .
\]
We denote the resulting Lie algebroid by $\Bar{L}_{\JJ, E}$, and define the cohomology of $\JJ$ to be the Lie algebroid cohomology of $\Bar{L}_{\JJ,E}$. 

\subsubsection{The AKSZ theory}

\def\fg{\mathfrak{g}}

From the point of view of topological field theory, Courant algebroids are important because they provide geometric examples of $2$-shifted symplectic spaces. 
Via the AKSZ construction, $2$-shifted symplectic spaces are the natural home for $3$-dimensional topological field theories in the BV formalism. 

\begin{Ex}
Any Lie algebra together with a non-degenerate invariant pairing defines a $2$-shifted symplectic structure on the classifying stack $B \fg$. 
The resulting AKSZ theory is Chern-Simons theory. 
\end{Ex}

To any Courant algebroid $E$, we can associate a derived stack $\XX_E$ which carries a $2$-shifted symplectic structure $\omega_E$. 
In the case that the Courant algebroid is exact, we denote the $2$-shifted symplectic space as $\XX_H$ where $H$ is the Severa class. 

To a compact, oriented $3$-manifold $M^3$ the AKSZ construction endows the derived mapping space 
\[
{\rm Map}(M^3 , \XX_E) 
\]
with a BV structure. 

\brian{Recall Lagrangian of a stack.}

An $H$-twisted generalized complex structure $\JJ$ defines a complex Dirac structure on $TX^{\Cc} \oplus T^*X^{\Cc}$, and hence a $2$-shifted Lagrangian inside of the $2$-shifted symplectic space $\XX_H$.  
More generally, if $\JJ$ is a generalized complex structure in an arbitrary Courant algebroid $E$, then we obtain a $2$-shifted Lagrangian inside of the $2$-shifted symplectic space $\XX_E$. 

The Lagrangian in $\XX_E$ associated to a generalized complex structure $\JJ$ defines a boundary condition for the three-dimensional AKSZ theory. 
In the case the the Courant algebroid is exact, the resulting boundary theory is the generalized $A/B$-model of \cite{KapustinLi}.

\subsection{Two dimensional TFT from GpC structures}

The goal in this section is to describe a construction of a two-dimensional topological field theory from the data of a generalized para complex structure. 
This construction is analogous to the generalized A/B-model of \cite{KapustinLi} that we have just recollected. 



\end{document}