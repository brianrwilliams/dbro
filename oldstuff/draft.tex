\documentclass{article}
\usepackage{amssymb, amsmath, amsthm, mathtools, bbm, tikz-cd,stmaryrd,enumerate}
%-------------------
\usepackage{fancyhdr}
\pagestyle{fancy}
\fancyhf{}
\fancyhead[L]{\leftmark}
\fancyhead[R]{\thepage}
%----------------
\newcommand{\TT}{{T\oplus T^*}}
\newcommand{\JJ}{\mathcal{J}}
\newcommand{\KK}{\mathcal{K}}
\newcommand{\GG}{\mathcal{G}}
\newcommand{\Cc}{\mathbf{C}}
\newcommand{\RR}{\mathbb{R}}
\newcommand{\XX}{\mathfrak{X}}
\newcommand{\HH}{\mathcal{H}}
\newcommand{\FF}{\mathcal{F}}
\newcommand{\QQ}{\mathcal{Q}}
\newcommand{\cE}{\mathcal{E}}
%----------------------------------------
\newcommand{\PP}{\mathrm{P}}
\newcommand{\PPt}{\tilde{\mathrm{P}}}
\newcommand{\id}{\mathbbm{1}}
\newcommand{\nlr}{\overset{\leftrightarrow}{\n}}
\newcommand{\lc}{\mathring{\n}}
\newcommand{\im}{\mathrm{Im}\,}
\newcommand{\Ker}{\mathrm{Ker}\,}
\newcommand{\Lie}{\mathcal{L}}
\newcommand{\PS}{\mathcal{P}}
\newcommand{\ap}{\alpha}
\newcommand{\bt}{\beta}
\def\w{\wedge}
\newcommand{\p}{\partial}
\newcommand{\pt}{\tilde{\partial}}
\newcommand{\xt}{{\tilde{x}}}
\newcommand{\n}{\nabla}
\newcommand{\rd}{\mathrm{d}}
\newcommand{\PH}{(\PS,\eta,\omega)}
\newcommand{\Lt}{\tl{L}}
\newcommand{\Lb}{\mathbb{L}}
\newcommand{\s}{\mathbf{s}}
\newcommand{\se}{\Gamma}
\newcommand{\Endo}{\text{End}}
\newcommand{\ellt}{{\tl{\ell}}}
\newcommand{\ot}{{1/2}}
\newcommand{\inv}{{-1}}
\newcommand{\Aa}{\mathcal{A}}
\newcommand{\la}{\langle}
\newcommand{\ra}{\rangle}
\newcommand{\lara}{\la\ ,\ \ra}
\newcommand{\brac}{[\ ,\ ]}
\newcommand{\bl}{[\![}
\newcommand{\br}{]\!]}
\newcommand{\bracd}{\bl \ ,\ \br}
\newcommand{\yt}{\tl{y}}
\newcommand{\zt}{\tl{z}}
\newcommand{\tth}{\tl{\theta}}
\newcommand{\kk}{\mathrm{k}}
%----------------------------------------
\def\gld{generalized Lie derivative }
\def\glds{generalized Lie derivatives }
\def\tl{\tilde}

\def\xto{\xrightarrow}

% These will be typeset in italics
\newtheorem{theorem}{Theorem}[section]
\newtheorem{proposition}[theorem]{Proposition}
\newtheorem{lemma}[theorem]{Lemma}
\newtheorem{corollary}[theorem]{Corollary}
\newtheorem{fact}[theorem]{Fact}
\newtheorem*{theorem*}{Theorem}
\newtheorem*{lemma*}{Lemma}
\newtheorem*{proposition*}{Proposition}
\newtheorem{Rem}[theorem]{Remark}

% These will be typeset in Roman
\theoremstyle{definition}
\newtheorem{Def}[theorem]{Definition}
\newtheorem{Conj}[theorem]{Conjecture}
\newtheorem{remark}[theorem]{Remark}

\newtheorem*{notation*}{Notation}

\theoremstyle{remark}
\newtheorem{Ex}[theorem]{Example}
\newtheorem{question}[theorem]{Question}
\newenvironment{claim}[1]{\par\noindent\underline{Claim:}\space#1}{}
\newenvironment{claimproof}[1]{\par\noindent\underline{Proof:}\space#1}{\hfill $[acksquare$}


\input xy

\xyoption{all}

\DeclareMathOperator{\End}{End}
\DeclareMathOperator{\rk}{rk}

\def\brian{\textcolor{blue}{BM: }\textcolor{blue}}
\def\david{\textcolor{red}{DB: }\textcolor{red}}

\title{Twisted parasupersymmetry and paracomplex geometry}

\begin{document}

\maketitle

\tableofcontents

\section{Introduction}

\subsection{Notations and conventions}

\brian{Conventions for odd Lie brackets $[-,-]$ and $\{-,-\}$.}

\section{Background on paracomplex and generalized geometry}
\david{would just put the introduction of paracomplex geometry (minus the generalized part) in appendix, we did that even in the other paper which is literally a differential geometry paper}

\subsection{Paracomplex geometry}

In this section we introduce para-complex geometry with emphasis on the analogy with complex geometry. More details can be found in \cite{Hu:2019zro} or \cite[Ch.~15]{Cortes:2010ykx}, for example. \brian{I think you should cite your work here too.} 

\begin{Def}\label{def:paracpx}
An (almost) {\bf product structure} on a smooth manifold $\PS$ is an endomorphism $K\in \Endo(T\PS)$ which squares to the identity, $K^2=\id_{T\PS}$, $K\neq \id$. An (almost) \textbf{para-complex structure} is a product structure such that the $+1$ and $-1$ eigenbundles of $K$ have the same rank.
\end{Def}
A direct consequence of above definition is that any para-complex manifold is of even dimension. From now on, the $+1$ and $-1$ eigenbundles of an almost product/para-complex structure will be denoted $L$ and $\Lt$, respectively.

\paragraph{Integrability} The use of the word {\it almost} as usual refers to integrability of the endomorphism, i.e. whether its eigenbundles are involutive with respect to the Lie bracket and therefore define a foliation of the underlying manifold. Similarly to the complex case, the integrability is governed by the \textbf{Nijenhuis tensor}
\begin{align}\label{eq:nijenhuis}
\begin{aligned}
N_K(X,Y)&=[X,Y]+[KX,KY]-K([KX,Y]+[X,KY])\\
&=(\n_{KX}K)Y+(\n_XK)KY-(\n_{KY}K)X-(\n_YK)KX\\
&=4(\PP[\PPt X,\PPt Y]+\PPt[\PP X,\PP Y]),
\end{aligned}
\end{align}
where $\n$ is any torsionless connection and $\PP\coloneqq\frac{1}{2}(\id+K)$ and $\PPt\coloneqq\frac{1}{2}(\id-K)$ are the projections onto $L$ and $\Lt$, respectively. We say $K$ is integrable and call $K$ a product/para-complex manifold if $N_K=0$. From \eqref{eq:nijenhuis} it is apparent that $K$ is integrable if and only if {\it both} eigenbundles are simultaneously Frobenius integrable, i.e. involutive distributions in $T\PS$. This is one of the main differences between complex and para-complex geometry; one of the eigenbundles can be integrable while the other is not. For this reason, it is useful to introduce the notion of {\bf half-integrability}:

%\begin{Def}
%Let $K$ be an almost para-complex. If the $+1$ (respectively $-1$) eigenbundle is Frobenius integrable, we call $K$ a {\bf p-para-complex} (respectively {\bf $n$-para-complex}) structure. A $K$ that is both $p$- and $n$- para-complex is simply a para-complex structure.
%\end{Def}

%\brian{The slash notation complex/product made it really hard to parse the definition since you already put parentheses. 
%If you want to add a definition for the product case make it another definition, or say "the same definition applies to integrability of product structures.
%Though, it doesn't seem like we use product structures at all below this, so maybe there is no point.}

\paragraph{Para-Complexification}
The analogy between complex and para-complex geometry can be made manifest by {\it para-complexification}. \brian{sentence split} 
This amounts to introducing the algebra of para-complex numbers $$\Cc=\RR\oplus \kk\RR,$$
\brian{This does not appear to be the most common mathematical terminology for this algebra. 
We should at least remark that these are more commonly referred to as ``split complex numbers", or "hyperbolic numbers".}
where $\kk^2=1$ is the para-complex unit, and take linear combinations over $\Cc$. 
The algebra $\Cc$ has the familiar natural operations
\begin{align}
\begin{aligned}
&\bullet\quad \text{Conjugation: }x+\kk y\mapsto \overline{x+\kk y}=x-\kk y\\
&\bullet\quad \text{Real part: } x+\kk y \mapsto \text{Re}(x+\kk y)=x\\
&\bullet\quad \text{Imaginary part: } x+\kk y \mapsto \text{Im}(x+\kk y)=y.
\end{aligned}
\end{align}

All tensorial objects can be extended to be $\Cc$-linear, in particular, the para-complexified tangent bundle decomposes as $T\PS\otimes \Cc\coloneqq T\PS^\Cc=T^{(1,0)}\oplus T^{(0,1)}$ to the $\pm \kk$ eigenbundles of the endomorphism $K$, now extended by linearity to $\End(T\PS^\Cc)$. 
There is an isomorphism between the para-complex vector spaces $\Cc^n$ and $\RR^{2n}$, explicitly (see \cite[Ch.~15]{Cortes:2010ykx} for more details):
\begin{align}\label{tab:identification}
\begin{array}{ccc}
(\Cc^n,\kk)&\simeq &(\RR^{2n},\id\times -\id)\\
(x^i+\kk y^i)_{i=1,\cdots,n} &\mapsto & (x^i+y^i,x^i-y^i)_{i=1,\cdots,n}\\
(\frac{x^i+\xt_i}{2}+\kk\frac{x^i-\xt_i}{2})_{i=1,\cdots,n} &\mapsfrom & (x^i,\xt_i)_{i=1,\cdots,n}.
\end{array}
\end{align}

\paragraph{Para-Holomorphic structure}
Let now $(\mathcal{P},K)$ be an almost para-complex manifold. If $K$ is integrable, we get a set of $2n$ coordinates $(x^i,\tilde{x}_i)$ called {\bf adapted coordinates}, $\mathcal{P}$ locally splits as $M\times \tilde{M}$, and $K$ acts as identity on $TM=L$ and negative identity on $T\tilde{M}=\tilde{L}$. The adapted coordinates therefore define two complimentary foliations of $\PS$. After para-complexification, we get $n$ {\bf holomorphic coordinates} \footnote{We should use the word para-holomorphic instead of holomorphic but we choose to omit the prefix ``para-" whenever no confusion with complex geometry is possible and recovering it only when needed.} $z^i=\frac{x^i+\xt_i}{2}+\kk \frac{x^i-\xt_i}{2}$, and the conjugate {\bf anti-holomorphic coordinates} $\bar{z}^i=\frac{x^i+\xt_i}{2}-\kk \frac{x^i-\xt_i}{2}$. The notion of holomorphicity in the context of para-complex geometry is now entirely analogous to complex geometry. As usual, a map of para-complex manifolds is called para-holomorphic if its pushforward commutes with the respective para-complex structures
\begin{Def}
Let $(M,K_M)$ and $(N,K_N)$ be para-complex manifolds. A map $F:M\rightarrow N$ is called para-holomorphic if
\begin{align}\label{eq:def_parahol}
K_N\circ F_*=F_*\circ K_M
\end{align}
\end{Def}

Locally, the map $F:M\rightarrow N$ of para-complex manifolds can be understood via composition of coordinates as a
\begin{align*}
F&: \RR^{2n}\rightarrow \RR^{2m}\\
F=(f^i,\tl{f}_j)&=(y^i(x^k,\xt_l),\yt_j(x^k,\xt_l))^{i,j=1,\cdots,m}_{k,l=1,\cdots, n}, 
\end{align*}
where $(x^k,\xt_l)$ and $(y^i,\yt_j)$ are the adapted coordinates on $M$ and $N$, respectively, or (after para-complexification) as a map
\begin{align*}
F&: \Cc^n\rightarrow \Cc^m\\
F&=F^i(z^k,\bar{z}^k)^{i=1,\cdots,m}_{k=1,\cdots, n}, 
\end{align*}
It is easy to check from \eqref{eq:def_parahol} that $F$ is a para-holomorphic map iff its components do not depend on anti-holomorphic coordinates on $M$, i.e.
\begin{align}\label{eq:holomorphic}
\frac{\p}{\p \bar{z}^k} F^j=\frac{\p}{\p z^k} \bar{F}^j=0.
\end{align}
In the real, adapted coordinates, the condition \eqref{eq:def_parahol} reads
\begin{align}\label{eq:holomorphic_real}
\frac{\p}{\p \xt_i}f^j=\frac{\p}{\p x^i}\tl{f}_j=0.
\end{align}
The equivalence between \eqref{eq:holomorphic} and \eqref{eq:holomorphic_real} can be checked using the relationships \eqref{tab:identification}.

The conditions \eqref{eq:holomorphic_real} tell us that the holomorphic functions preserve the foliations defined by the adapted coordinates on $M$ and $N$. This also means that the transition functions on a para-complex manifold (seen as maps $\RR^{2n}\rightarrow \RR^{2n}$) are holomorphic since the foliations must be preserved, i.e. the coordinates transform as
\begin{align*}
(x^i,\xt_i)\mapsto (y^j(x^i),\yt_j(\xt_i)).
\end{align*}


\paragraph{Type decomposition} The splitting of the tangent bundle $T\PS=L\oplus \Lt$ in the real case and $T\PS^\Cc$ in the para-complex case gives rise to a decomposition of tensors analogous to the $(p,q)$-decomposition in complex geometry. For differential forms, we denote the real and para-complex decompositions as
\begin{align}\label{eq_plusminus_decomp}
\Lambda^k (T^*\mathcal{P})&=\bigoplus_{k=m+n}\Lambda^{(+m,-n)}(T^*\mathcal{P}),\\
\Lambda^k (T^*\mathcal{P}^\Cc)&=\bigoplus_{k=m+n}\Lambda^{(m,n)}(T^*\mathcal{P}),
\end{align}
where $\Lambda^{(+m,-n)}(T^*\mathcal{P})=\Lambda^m(L^*)\otimes \Lambda^n(\Lt^*)$ and $\Lambda^{(m,n)}(T^*\mathcal{P})=\Lambda^m(T^{(1,0)*})\otimes \Lambda^n(T^{(0,1)*})$. The corresponding sections are denoted by $\Omega^{(+m,-n)}(\mathcal{P})$ and $\Omega^{(m,n)}$, respectively. The bigradings \eqref{eq_plusminus_decomp} yield the natural projections
\begin{align*}
\Pi^{(+p,-q)}:\Lambda^k(T^*\mathcal{P})&\rightarrow \Lambda^{(+p,-q)}(T^*\mathcal{P}),\\
\Pi^{(p,q)}:\Lambda^k(T^*\mathcal{P}^\Cc)&\rightarrow \Lambda^{(p,q)}(T^*\mathcal{P}),
\end{align*}
so that when $K$ is integrable, the de-Rham differential splits as $\rd=\p_++\p_-$, in the real case and $\rd=\p+\bar{\p}$ in the para-complexified case, where
\begin{align*}
\begin{array}{cc}
\p_+ \coloneqq \Pi^{(+p+1,-q)}\circ \rd, & \p_- \coloneqq \Pi^{(+p,-q-1)}\circ \rd,\\
\p \coloneqq \Pi^{(p+1,q)}\circ \rd, & \bar{\p} \coloneqq \Pi^{(p,q+1)}\circ \rd,
\end{array}
\end{align*}
are the \textbf{para-complex Dolbeault operators}, satisfying
%\begin{align}\label{eq:partials_plusminus}
%\begin{aligned}
%\p_+&:\Omega^{(+p,-q)}(\mathcal{P})\rightarrow \Omega^{(+p+1,-q)}(\mathcal{P})\\
%\p_-&:\Omega^{(+p,-q)}(\mathcal{P})\rightarrow \Omega^{(+p,-q-1)}(\mathcal{P}),
%\end{aligned}
%\end{align}
\begin{align*}
\p_+^2=\p_-^2=\p_+\p_-+\p_-\p_+&=0\\
\p^2=\bar{\p}^2=\p\bar{\p}+\bar{\p}\p&=0
\end{align*}

One can also introduce the {\it twisted differential $\rd^p\coloneqq(\Lambda^{k+1}K^*)\circ\rd\circ (\Lambda^kK^*)$}. When $K$ is integrable, it can be simply written as $\rd^p=(\p_++\p_-)$ on real forms and $\rd^p=\kk (\p+\bar{\p})$ on para-complex forms.
  
%\begin{lemma}
%Let $(\PS,K)$ be a paracomplex manifold. Then $\rd^p\coloneqq(\Lambda^{k+1}K)\circ\rd\circ (\Lambda^kK)$ can be expressed as
%\begin{align}\label{eq:dp-operator}
%\rd^p=\p_+-\p_-.
%\end{align}
%\end{lemma}
%\begin{proof}
%Let $\ap \in \Omega^{+m,-n}(\PS)$. Then we have
%\begin{align*}
%\rd^p\ap=(-1)^n(\Lambda^kK)\rd\ap=(-1)^{2n}\p_+ \ap +(-1)^{2n+1}\p_-\ap=(\p_+-\p_-)\ap,
%\end{align*}
%\end{proof}


\subsection{Generalized para-geometry}
In this paper, the term {\it generalized geometry} is used for the study of geometric \brian{geometrical is not a word} structures on the bundle $TM \oplus T^*M$ over some manifold $M$. We will typically abbreviate this bundle to $\TT$ whenever the base is understood or unimportant for the discussion.

\subsubsection{The exact Courant algebroid structure}
\brian{David, how do you feel about moving this later in the Courant/AKSZ section?}
We recall basic facts about Courant algebroids in Section \ref{sec: courant}. 
For now, we record a relationship between generalized paracomplex structures and exact Courant algebroids. 

The bundle $\TT$ has a natural Courant algebroid structure given by the symmetric pairing
\begin{align*}
\langle X+\ap,Y+\bt\rangle=\ap(Y)+\bt(Y),
\end{align*}
the Dorfman bracket,
\begin{align}\label{eq:dorfman}
[ X+\ap,Y+\bt]=[X,Y]+\Lie_X\bt-\imath_Y\rd \ap, 
\end{align}
and the anchor $\pi:X+\ap\mapsto X$. In the above, $X+\ap$ denotes a section of $\TT$ with the splitting to tangent and cotangent parts given explicitly. The Dorfman bracket can be thought of as an extension of the Lie bracket from $T$ to $\TT$ and therefore we opt to use the same notation for both brackets; the expression $[X,Y]$ is always the Lie bracket of vector fields whether we think of $\brac$ as the Lie bracket or the Dorfman bracket and no confusion is therefore possible.

The Courant algebroid on $\TT$ is exact, meaning that the associated sequence
\begin{align}\label{eq:exact_seq}
0\longrightarrow T^* \overset{\pi^T}{\longrightarrow} \TT\overset{\pi}{\longrightarrow} T\longrightarrow 0,
\end{align}
is exact. Here, $\pi^T$ is the transpose of $\pi$ with respect to the pairing $\lara$,
%\begin{align*}
%\la \pi^T(\ap),Y+\bt\ra=\la \ap,\pi(Y+\bt)\ra=\la \ap,Y\ra
%\end{align*}
i.e. $\pi^T: \ap \mapsto \ap+0$. In fact, all possible Courant algebroid structures on $\TT$ are parametrized by a closed three-form $H\in \Omega^3_{cl}$, which enters the definition of the bracket \eqref{eq:dorfman}, changing it to a {\it twisted} Dorfman bracket
\begin{align*}
[ X+\ap,Y+\bt]_H=[X,Y]+\Lie_X\bt-\imath_Y\rd \ap+\imath_Y\imath_X H.
\end{align*}
Moreover, any isotropic splitting of \eqref{eq:exact_seq} $s:T\rightarrow \TT$ is given by a two-form $b$, such that $X\overset{s}{\mapsto}X+b(X)$. This is equivalent to an action of a $b$-field transformation on $\TT$\footnote{Here we are using the term $b$-field transformation more liberally as it is customary to use the term only in the cases when $\rd b=0$ so that $e^b$ is a symmetry of $\brac$.}
\begin{align*}
e^b=
\begin{pmatrix}
\id & 0 \\
b & \id
\end{pmatrix}
\in \End(\TT),
\end{align*}
which consequently changes the bracket as
\begin{align*}
[ e^b(X+\ap),e^b(Y+\bt)]_H=e^b([ X+\ap,Y+\bt]_{H+\rd b}),
\end{align*}
which implies that when $H$ is trivial in cohomology, then a choice of a $b$-field transformation such that $\rd b=-H$ brings the twisted bracket $\brac_H$ into the standard form \eqref{eq:dorfman}. When $H$ is cohomologically non-trivial this can be done at least locally. This also means that any choice of splitting with a non-trivial $b$-field can be absorbed in the Courant algebroid bracket in terms of the {\it flux}\footnote{Flux is a term used mainly in physics, in this context simply meaning the ``tensorial contribution to the bracket''.} $\rd b$.

\paragraph{Dirac Structures}
An important object in Dirac geometry are (almost) dirac structures, which are subbundles $L\subset \TT$ with special properties.
\begin{Def}
An {\bf almost Dirac structure} $L$ is a maximally isotropic subbundle of $\TT$, i.e. $\la u,v\ra=0$ for any $u,v \in \se(L)$ and $\text{rank}(L)=\text{rank}(T)$. When $L$ is involutive under the Dorfman bracket, i.e. it satisfies $[L,L]\subset L$, we call $L$ simply a {\bf Dirac structure}.
\end{Def}
An important fact we will repeatedly use is the following:

\begin{proposition}[\cite{courant1990dirac}]\label{prop:dirac_Liealg}
Let $L$ be a Dirac structure in the Courant algebroid $\TT$ with a flux $H$. Then the restriction of the Dorfman bracket to $L$, $\brac\mid_L$ is skew and gives $L$ a structure of a Lie algebroid compatible with the anchor $\pi\mid_L$, a restriction of the projection $\pi:\TT\rightarrow T$ to $L$.
\end{proposition} 
 
 
We remark here that all the results in this paper remain valid for any exact courant algebroid $E$ (i.e. $E$ fits in the sequence \eqref{eq:exact_seq}), which can be always identified with $\TT$ by the choice of splitting equivalent to a choice of a representative $H\in\Omega^3_{cl}$. This also amounts to setting $b=0$ in all formulas since the $b$-field appears as a difference of two splittings.

\subsection{Generalized para-K\"ahler geometry}
\brian{I don't think this should be the title of this section. It's mostly about paracomplex geometry}

We now recall the notion of a generalized para-complex \cite{wade2004dirac,Zabzine:2006uz,Hu:2019zro} and a generalized para-K\"ahler \cite{Hu:2019zro} structure. We follow the discussion presented in \cite{Hu:2019zro}.

\begin{Def}
A \textbf{(twisted) generalized para-complex} (GpC) structure $\KK$ is an endomorphism of $\TT$, such that $\KK^2=\id$ and $\la\KK,\KK\ra=-\lara$, whose generalized Nijenhuis tensor vanishes:
\begin{align}\label{eq:gen_nijenhuis}
\mathcal{N}_\KK(u,v)=[\KK u,\KK v]_H+\KK^2[ u,v]_H-\KK([\KK u,v]_H+[ u,\KK v]_H)=0.
\end{align}
\end{Def}
It is easy to check that the condition \eqref{eq:gen_nijenhuis} is equivalent to the $\pm 1$ eigenbundles of $\KK$ to be involutive under the twisted Dorfman bracket. Since in this paper the flux $H$ will be generically non-zero but all results hold for the special case $H=0$ as well, we will drop the word ``twisted" from our definitions.

Generalized para-complex structures can be understood as the data one needs to specify a splitting of $\TT$ to a pair of integrable Dirac structures:

\begin{theorem*}[\cite{wade2004dirac}]\label{thm:pairofdirac}
There is a one-to-one correspondence between generalized para-complex structures on $M$ and pairs of transversal Dirac subbundles of $\TT$.
\end{theorem*}

Combining this result with the well-known result of \cite{Liu:1995lsa} which states that any pair of transversal Dirac structures $(L,\Lt)$ forms a Lie bialgebroid $(L,L^*\simeq \Lt)$, one can immediately infer the following

\begin{corollary}
Generalized para-complex structures on $\TT$ are in one-to-one correspondence with Lie bialgebroid pairs $(L,L^*)$ such that $L\oplus L^*=\TT$. 
\end{corollary}


\begin{Ex}[The trivial structure and its deformations]\label{ex:GpC_trivial}
Any manifold supports the following GpC structure
\begin{align*}
\KK_0=
\begin{pmatrix}
\id & 0 \\
0 & -\id
\end{pmatrix},
\end{align*}
that has eigenbundles $T$ and $T^*$ and is always integrable. The following two GpC structures can be seen as deformations of $\KK_0$ by either a two-form $b$ or a bi-vector $\beta$:
\begin{align*}
\KK_b=
\begin{pmatrix}
\id & 0 \\
2b & -\id
\end{pmatrix},\quad
\KK_\beta=
\begin{pmatrix}
\id & 2\beta \\
0 & -\id
\end{pmatrix}.
\end{align*}
$\KK_b$ is integrable iff $\rd b =-H$, and its eigenbundles are $\Lb_b=\text{graph}(b)=\{X+b(X)\mid X \in \XX\}$ and $\widetilde{\Lb}=T^*$. Similarly, $\KK_\beta$ is integrable iff $\beta$ is a Poisson structure, (see \cite[Lemma~2.13]{Hu:2019zro}) and its eigenbundles are $\Lb=T$ and $\widetilde{\Lb}=\text{graph}(-\beta)=\{\ap-\beta(\ap)\}\mid \ap \in \Omega\}$.
\end{Ex}

\begin{Ex}[Para-complex structures]
A para-complex structure $K\in \Endo(T)$, defines the diagonal generalized para-complex structure:
\begin{align*}
\KK_K=
\begin{pmatrix}
K & 0 \\
0 & -K^*
\end{pmatrix}.
\end{align*}
The corresponding Dirac structures are given by $\Lb=T^{(1,0)}\oplus T^{*(0,1)}$ and $\widetilde{\Lb}=T^{(0,1)}\oplus T^{*(1,0)}$, where the bigrading is with respect to $K$. The integrability of $\KK_K$ is equivalent to Frobenius integrability of $K$ , i.e. vanishing of the Nijenhuis tensor of $K$.
%
%The structure $\KK_P$ can be generalized to the following
%\begin{align*}
%\KK=
%\begin{pmatrix}
%P & \Pi \\
%\Omega & -P^*
%\end{pmatrix},
%\end{align*}
%where we can read off from the constraints in \eqref{eq:GpC_generalform} that $P^2=\id$ implies that $\Pi\Omega=0$ and both $\Pi$ and $\Omega$ have to be of type $(2,0)+(0,2)$ with respect to the grading given by $P$.

\end{Ex}

\begin{Ex}[Symplectic structures]\label{ex:GpC_sympl}
A symplectic form $\omega$ defines the anti-diagonal GpC structure
\begin{align*}
\KK_\omega=
\begin{pmatrix}
0 & \omega^{-1} \\
\omega & 0
\end{pmatrix}.
\end{align*}
The $\pm 1$ eigenbundles are given by $\text{graph}(\pm\omega)=\{X\pm\omega(X)\mid X\in \XX\}$, and the integrability of $\KK_\omega$ is equivalent to $\rd\omega=0$.
\end{Ex}

\brian{Can we import some of the other examples here?
Maybe some explicit ones like the toric fibrations you mentioned.}

\paragraph{Comparision with generalized complex structures}
Example \ref{ex:GpC_sympl} shows that a symplectic manifold $(M,\omega)$ is a GpC manifold and it is well-known \cite{Gualtieri:2003dx} that $(M,\omega)$ is also generalized complex (GC). However, while almost GC structures exist only on almost complex manifolds \cite{Gualtieri:2003dx}, Example \ref{ex:GpC_trivial} demonstrates that GpC structures exist on any Poisson manifold and in particular on any smooth manifold (with trivial Poisson structure). Another feature of GpC geometry that is not present in GC geometry is that the GpC structures can be half-integrable (similarly to ordinary para-complex structures): see the cases of $\KK_b$ and $\KK_\beta$ from Example \ref{ex:GpC_trivial} which are always at least half integrable and are fully integrable iff $b$ is closed and $\beta$ is Poisson, respectively. On the other hand, $\KK_\omega$ in Example \ref{ex:GpC_sympl} can only be half-integrable when the $H$-flux is non-zero.

\begin{Def}
A \textbf{generalized para-K\"ahler structure} (GpK) is a commuting pair $(\KK_+,\KK_-)$ of GpC structures, such that their product $\GG=\KK_+\KK_-$ defines a split-signature metric $G$ on $\TT$ via
\begin{align*}
G(\cdot,\cdot)\coloneqq \la \GG\cdot,\cdot\ra.
\end{align*}
\end{Def}

Because $\KK_\pm$ are GpC structures, they induce the splitting $\TT= \Lb_\pm\oplus\widetilde{\Lb}_\pm$. Moreover, because they commute, each of $\Lb_+$ and $\widetilde{\Lb}_+$ further splits to eigenbundles of $\KK_-$, so that one arrives at the decomposition of $\TT$ into four bundles:
\begin{align}\label{GpK_bundles}
\TT=\ell_+\oplus\ell_-\oplus \ellt_+\oplus \ellt_-,
\end{align}
such that the eigenbundles of $\KK_+$ are $\Lb_+=\ell_+\oplus\ell_-$ and $\tl{\Lb}_+=\ellt_+\oplus \ellt_-$, the eigenbundles of $\KK_-$ are $\Lb_-=\ell_+\oplus\ellt_-$ and $\tl{\Lb}_-=\ell_-\oplus \ellt_+$ and the eigenbundles of the generalized metric are $C_\pm=\ell_\pm\oplus\ellt_\pm$. All $\ell_\pm$ and $\ellt_\pm$ are isotropic and integrable under the Dorfman bracket \cite{Hu:2019zro} and bundles with opposite chirality are orthogonal.

Similarly to generalized K\"ahler structures, the GpK structures can be equivalently given in terms of a pair of para-Hermitian structures satisfying a particular integrability condition:
\begin{theorem}[\cite{Hu:2019zro}]
The data of a GpK structure is equivalent to a pair of para-Hermitian structures $(\eta,K_+,K_-)$, such that $\KK_\pm$ are given by
\begin{align}\label{eq:GpK_genform}
\KK_{\pm}=\frac{1}{2}
\begin{pmatrix}
\id & 0 \\
b & \id
\end{pmatrix}
\begin{pmatrix}
K_+\pm K_- & \omega^{-1}_+\mp \omega^{-1}_- \\
\omega_+\mp \omega_- & -(K_+^*\pm K_-^*)
\end{pmatrix}
\begin{pmatrix}
\id & 0 \\
-b & \id
\end{pmatrix},
\end{align}
for some $2$-form $b$. The integrability of $\KK_\pm$ then translates into $K_\pm$ being integrable and satisfying
\begin{align*}
\n^\pm K_\pm=0,
\end{align*}
where the connections $\n^\pm$ are defined by the Levi-Civita connection $\lc$ of $\eta$ and the $H$-flux:
\begin{align*}
\eta(\n^\pm_XY,Z)=\eta(\lc_XY,Z)\pm\frac{1}{2}(H+\rd b).
\end{align*}
\end{theorem}

The additional integrability condition on $K_\pm$, $\n^\pm K_\pm$, can be equivalently expressed using the fundamental forms associated to $K_\pm$, $\omega_\pm=\eta K_\pm$, and a para-complex version of the $\rd^c$-operator, which we call $\rd^p$:
\begin{align*}
\n^\pm K_\pm=0 \Longleftrightarrow \rd^p_\pm\omega_\pm=\pm (H+\rd b).
\end{align*}
$\rd^p_\pm$ here denote the $\rd^p$ operators associated to $K_\pm$, $\rd^p_\pm\coloneqq K_\pm^*\circ \rd\circ K_\pm^*$. We will call the geometry given by the data $(\eta, K_\pm, H+\rd b)$ above a {\bf bi-para-Hermitian} geometry.


\subsubsection{Isomorphism between $\TT$ and $T\oplus T$}\label{sec:isomorphism}
In both the GK and GpK cases, the product of the pair of G(p)C structures defines a generalized (indefinite) metric $\GG$, which means that
\begin{align*}
G(\cdot,\cdot)\coloneqq \la \GG\cdot,\cdot\ra,
\end{align*}
defines a genuine (indefinite) metric on the bundle $\TT$. It can be shown (see eg. \cite{Gualtieri:2003dx,Hu:2019zro}) that such metric can be always locally expressed in terms of a two-form $b$ and a metric $g$
\begin{align*}
\GG=\GG(g,b)=\begin{pmatrix}
\id & 0 \\
b & \id
\end{pmatrix}
\begin{pmatrix}
0 & g^{-1} \\
g & 0
\end{pmatrix}
\begin{pmatrix}
\id & 0 \\
-b & \id
\end{pmatrix},
\end{align*}
and the signature of $g$ then determines the signature of $\GG(g,b)$. Therefore, a generalized metric determines a splitting of $\TT\simeq e^b(T)\oplus T^*$ in which the eigenbundles $C_\pm$ of $\GG$ are $C_\pm=graph(\pm g)$ and the flux of this splitting is given by the closed three-form $H\overset{loc.}{=}\rd b$. In the usual splitting $\TT$, we then have $C_\pm=graph(b\pm g)$.

\begin{remark}
The most general $2D$ $(1,1)$ sigma model on a pseudo-Riemannian target $(M,g)$ is locally given by the action functional
\begin{align}\label{rem:sigmamodel}
S_{(1,1)}(\Phi)&=\int_{\hat{\Sigma}}[g(\Phi)+b(\Phi)]_{ij}D_+\Phi^iD_-\Phi^j,
\end{align}
where $\rd b=H$ is a global closed three-form. Therefore, the data of a generalized metric $\GG$ is equivalent to a specification of the sigma model  \eqref{rem:sigmamodel}.
\end{remark}

An important property of the generalized metric $\GG$ is that its  eigenbundles $C_\pm$ are isomorphic to the tangent bundle $T$ via the projection:
\begin{align}\label{pi_iso}
\begin{aligned}
\pi_\pm:C_\pm &\simeq T,\\
X+\ap &\overset{\pi_\pm}{\longmapsto} X,\\
X+(b\pm g)X &\overset{\pi^{-1}_\pm}{\longmapsfrom} X,
\end{aligned}
\end{align}
where $g$ and $b$ are the defining data of $\GG$. Therefore, since $\TT=C\oplus C_-$, we also have $\TT\simeq T\oplus T$:
\begin{align}
\begin{aligned}
(X_+,X_-)\overset{\pi^{-1}_+\oplus \pi^{-1}_-}{\longmapsto}& X_++(b+g)X_++X_-+(b-g)X_-\\
&=
\begin{pmatrix}
 X_++X_- \\
 g(X_+-X_-)+b(X_++X_-)
\end{pmatrix}.
\end{aligned}\label{map_pi_pm}
\end{align}
In particular, one can check that $\pi_+\oplus \pi_-$ maps the eigenbundles of the generalized structures $\KK_\pm$ ($\JJ_\pm$) to eigenbundles of the corresponding pair of tangent bundle endomorphisms $\KK_\pm$ ($\JJ_\pm$). This is because of the following relationship \cite{Hu:2019zro},
\begin{align*}
K_+=\pi_+\KK_\pm\pi_+^{-1},\quad K_-=\pm\pi_-\KK_\pm\pi_-^{-1},
\end{align*}
and analogously for $\JJ_\pm$ and $J_\pm$. From here, it is easy to see that the $+1$ eigenbundle of $\KK_+$ $\Lb_+$, for example, is given by
\begin{align*}
\Lb_+=\pi_+^{-1}(T^{(1,0)_+})\oplus\pi_-^{-1}(T^{(1,0)_-}),
\end{align*}
where $T^{(1,0)_\pm}$ denotes the $+1$ eigenbundle of $K_\pm$. Analogously, the $-1$ eigenbundle of $\KK_-$,
$\widetilde{\Lb}_-$, is given by
\begin{align*}
\widetilde{\Lb}_-=\pi_+^{-1}(T^{(0,1)_+})\oplus\pi_-^{-1}(T^{(1,0)_-}),
\end{align*}
and so on. Therefore, one should think about the pair $K_\pm$ corresponding to the GpK structure as each of the $K_\pm$ acting on its own copy of $T$, which are then mapped via $\pi_\pm$ to $C_\pm$ in $\TT$.

We will also notice that $\pi_\pm$ map the the pseudo-Riemannian structures $G$ and $g$ on each other:
\begin{align}\label{pi:gG_relationship}
g(X,Y)=\pm\frac{1}{2}\la \pi_\pm^{-1}X,\pi_\pm^{-1}Y\ra = \frac{1}{2}\la \GG \pi_\pm^{-1}X,\pi_\pm^{-1}Y\ra =\frac{1}{2}G(\pi_\pm^{-1}X,\pi_\pm^{-1}Y).
\end{align}
Another important property we will make use of is the fact that the connections $\n^\pm$ give rise via $\pi_\pm$ to a generalized connection $D$ on $\TT$ called the generalized Bismut connection \cite{Gualtieri:2007bq}:
\begin{align}\label{genBismut_pi_nabla_pm}
D_uv=\pi^{-1}_+\n^+_{\pi u}\pi_+ v_++\pi^{-1}_-\n^-_{\pi u}\pi_- v_-.
\end{align}
This connection preserves $\lara$ and $\GG$ and in the case when $\GG$ is a generalized metric corresponding to a GpK structure, $D$ also preserves all the eigenbundles in \eqref{GpK_bundles}. 

There is a tensorial quantity associated to any generalized connection called generalized torsion, which is defined as \cite{Gualtieri:2007bq}
\begin{align}\label{gentorsion_def}
T^D(u,v,w)=\la D_uv-D_vu-[u,v]_H,w\ra +\la D_wu,v\ra.
\end{align}
For the generalized Bismut connection, this torsion has only pure components in $\Lambda^3C_\pm$ and satisfies the following formula \cite[Prop.~2.29]{Hu:2019zro}:
\begin{align}\label{gentorsion_H}
T^D(\pi_\pm^{-1}X,\pi_\pm^{-1}Y,\pi_\pm^{-1}Z)=2H_b(X,Y,Z).
\end{align}


%\paragraph*{Relationship to Kapustin-Li}
%On pg. 12 of Kapustin-Li paper they implicitly use the isomorphisms $\pi_\pm$ to map the sections $\chi \in \se(T^{(0,1)_+})$ and $\lambda \in \se(T^{(0,1)_-})$ to the $-i$ eigenbundle of $\JJ_+$ as 
%\begin{align*}
%\begin{pmatrix}
%\chi+\lambda \\
%g(\chi-\lambda)
%\end{pmatrix},
%\end{align*}
%which is the same as \eqref{map_pi_pm} with $b=0$, which they implicitly absorb inside the $H$-flux.

\subsubsection{SKT Geometry and half generalized structures}
\david{will see if we'll be using this}
An {\bf SKT} structure is a Hermitian structure $(I,g)$ for which the fundamental form $\omega$ even though is not closed, is $\rd \rd^c$-closed, i.e. $\rd \rd^c \omega=0$. The version of this geometry is easily formulated for the para- case where we will call it para-SKT. It follows directly from the definition that the bi-para-Hermitian geometry consists of two such structures -- one for each chirality -- and here we will describe how one SKT structure can be described using the language of generalized geometry.

For this purpose we define a positive-chirality\footnote{Similarly one could define a negative-chirality half G(p)C structure by changing $C_+$ for $C_-$} {\bf half generalized almost (para-)complex structure} as a pair $(\GG,\JJ_{C_+})$, where $\GG$ is a (neutral) generalized metric which induces a splitting $\TT=C_+\oplus C_-$ and an isomorphism $C_+\oplus C_-\simeq T\oplus T$, and $\JJ_{C_+}$ is a (para-)complex endomorphism of $C_+$, i.e.
\begin{align*}
\JJ_{C_+}\in \End(C_+),\quad \JJ_{C_+}=\pm \id_{C_+},\quad \la \JJ_{C_+} u_+,v_+\ra=-\la u_+,\JJ_{C_+} v_+\ra,
\end{align*}
for any $u_+,v_+ \in \se(C_+)$. The integrability condition on $\JJ_{C_+}$ is then that its eigenbundles $\ell_+\oplus \ell_-=C_+$ (in the complex case $\ell\oplus \bar{\ell}=C_+\otimes \mathbb{C}$) are involutive with respect to the (twisted) Dorfman bracket. It is easy to see that $\JJ_{C_+}$ defines a (para-)Hermitian structure $(J_+,g)$, where
\begin{align*}
g(X,Y)=\frac{1}{2}\la \pi_+^{-1}X,\pi_+^{-1}Y\ra,\quad J_+=\pi_+\JJ_{C_+}\pi^{-1}_+
\end{align*}
and the integrability condition on $\JJ_{C_+}$ translates into the condition
\begin{align*}
\n^+J_+=0,\ \n^+=\lc+\frac{1}{2}g^{-1}(H+\rd b) \Longleftrightarrow \rd^{c/p}\omega_+=H+\rd b,
\end{align*}
i.e. giving exactly half of the data of a G(p)K geometry. If there is a negative-chirality half structure $\JJ_{C_-}$, then $(\GG,\JJ\coloneqq \JJ_{C_+}\oplus \JJ_{C_-})$ defines a genuine G(p)K structure.


\section{Two-dimensional parasupersymmetry}  \label{sec: parasusy}

In this section we introduce the parasupersymmetry algebra in two-dimensions. 
\brian{add more}

\subsection{A reminder on $(1,1)$ supersymmetry}

The $2$-dimensional $(1,1)$ supertranslation algebra is the super Lie algebra
\[
\mathfrak{t}_{(1,1)} = \RR^{1,1} \oplus \Pi (S_+) \oplus \Pi(S_-)
\]
where $S_\pm \cong \RR$ are the semi-spin representations of ${\rm Spin}(1,1)$ labeled by helicities $\pm \frac{1}{2}$. 
Note that there are natural ${\rm Spin}(1,1)$-equivariant maps
\[
\Gamma_{\pm} : {\rm Sym}^2(S_\pm) \to \RR^{1,1}
\]
which are non-degenerate and whose images hit the subalgebras spanned by $\partial_{\pm}$, respectively. 
The only nonvanishing Lie brackets in $\mathfrak{t}_{(1,1)}$ are defined by
\[
[Q_+, Q_+'] = \Gamma_+(Q_+ \otimes Q_+') \;\; , \;\; [Q_-, Q_-'] = \Gamma_-(Q_- \otimes Q_-')
\]  
where $Q_\pm, Q_\pm'$ are generic elements in $S_\pm$. 

We will use the lightcone basis $\p_{\pm}$ for $\RR^{1,1}$ and we also fix a basis $Q_\pm^1$ for $S_{\pm}$ with a relative normalization so that
\begin{align}\label{eq:(1,1)_susy}
[Q^1_\pm,Q^1_\pm] = 2\p_\pm.
\end{align}

\subsubsection{The $(1,1)$ supersymmetric $\sigma$-model} 

Let $\Sigma^{2|2}$ be a super Riemann surface with two real odd directions. 
The even part of this supermanifold is a Riemann surface (with coordinates $\sigma$, $\tau$).
For instance, in the flat case we consider the superspace $\RR^{2|2} = \{(x^i , \theta_1^{\pm}) \; | \; i = 1,2\}$, where $\theta_1^{\pm}$ are odd variables. 

The most general $(1,1)$ supersymmetric sigma model into a target pseudo-Riemannian manifold $(M,g)$ is given by the action functional
\begin{align}\label{eq:(1,1)action}
S_{(1,1)}(\Phi)=\int_{\Sigma^{2|2}} [g(\Phi)+b(\Phi)]_{ij}D^1_+\Phi^iD^1_-\Phi^j,
\end{align}
where $\Phi=(\Phi^i)_{i=1\cdots n}$ are local representatives for the $(1,1)$ {superfields} which are maps $\Phi: \Sigma^{2|2} \rightarrow M$.
In the action, $b$ denotes a local two-form potential for a closed three-form, $H=\rd b$, and $D_\pm$ are the superderivatives
\begin{align}\label{eq:D1}
D^1_\pm=\frac{\p}{\p \theta_1^\pm}-\theta_1^\pm \p_\pm,
\end{align}
where $\p_\pm=\frac{\p}{\p x_\pm}$ are derivatives along the lightcone coordinates on $\Sigma$, $x_\pm=\tau\pm\sigma$. 

In superworldsheet coordinates, the superfields decompose as
\begin{align*}
\Phi^i(\sigma,\tau,\theta_+,\theta_-)=\phi^i(\sigma,\tau)+\psi^i_+(\sigma,\tau)\theta^++\psi_-(\sigma,\tau)\theta^-+F^i(\sigma,\tau)\theta^+\theta^-.
\end{align*}
Here, $(\phi^i)$ are local representatives for a smooth map $\phi: \Sigma \rightarrow M$.

The fact that this action carries $(1,1)$ supersymmetry means that the action \eqref{eq:(1,1)action} is invariant under transformations generated by the two {supercharges} $Q^1_\pm$, 
\begin{align}\label{eq:Q1}
Q^1_\pm=\frac{\p}{\p \theta_1^\pm}+\theta_1^\pm \p_\pm,
\end{align}
obeying the supercommutation relations of the $(1,1)$ supersymmetry algebra (\ref{eq:(1,1)_susy}). 



\subsection{$(2,2)$ parasupersymmetry}

\brian{point to references who have also discussed versions of (2,2) para susy}
The {\bf (2,2) parasupersymmetry algebra} has the underlying supervector space
\[
\mathfrak{t}_{(2,2)} = \RR^{1,1} \oplus \Pi(S_+ \oplus S_+) \oplus \Pi(S_- \oplus S_-) .
\]
The Lie brackets are described as follows. 
If we choose a basis $\{Q^1_{\pm}, Q^2_\pm\}$ for $S_\pm \oplus S_\pm$ then the nontrivial brackets read 
\begin{equation}\label{(2,2)para}
(2,2) \; {\rm para} \;\; : \;\; [Q_\pm^1, Q_\pm^1] = 2 \partial_{\pm} \;\; , \;\; [Q_{\pm}^2, Q_{\pm}^2] = -2 \partial_{\pm} .
\end{equation}

\begin{remark}
Note that this differs from ordinary $(2,2)$ supersymmetry by a sign:
\begin{equation}\label{(2,2)para}
(2,2) \; {\rm ordinary} \;\; : \;\; [Q_\pm^1, Q_\pm^1] = 2 \partial_{\pm} \;\; , \;\; [Q_{\pm}^2, Q_{\pm}^2] =  2 \partial_{\pm} .
\end{equation}
\end{remark}


More generally, suppose $W_\pm$ are vector spaces of dimension $k_\pm$, respectively.
In addition, suppose $\eta_\pm$ are symmetric nondegenerate forms on $W_\pm$. 
Then, we can define a super Lie algebra of the form
\[
\mathfrak{t}_{W} =  \RR^{1,1} \oplus \Pi(S_+ \otimes W_+) \oplus \Pi(S_- \otimes W_-)
\]
with brackets defined by
\[
[Q_\pm^{a}, Q_\pm^{b}] = 2 \eta^{ab}_\pm \partial_\pm .
\] 
The case of $\mathfrak{t}_{(2,2)}$, which will be of most interested to us, corresponds to taking $k_+ = k_- = 2$ and $\eta_\pm^{11} = 1$, $\eta^{22}_\pm = - 1$, $\eta^{12}_{\pm} = 0$. 

\brian{Can change basis to off diagonal and write in more conventional way.}

\subsubsection{R-symmetry}
We have seen above that the $(2,2)$ para-SUSY algebra $\mathfrak{t}_{(2,2)}$ can be diagonalised in terms of the charges\footnote{Note that for the usual $(2,2)$ SUSY one cannot diagonalize the algebra over $\RR$.}
\begin{align}\label{eq:R-sym_charges}
\begin{aligned}
[Q_\pm^1,Q_\pm^1]&=2\p_\pm\\
[Q_\pm^2,Q_\pm^2]&=-2\p_\pm.
\end{aligned}
\end{align}
We make a change of basis by the formulas $Q_\pm=\frac{1}{\sqrt{2}}(Q^2_\pm+Q^1_\pm)$ and $\tl{Q}_\pm=\frac{1}{\sqrt{2}}(Q^2_\pm-Q^1_\pm)$. 
Equations \eqref{eq:R-sym_charges} can be written collectively as
\begin{align}\label{eq:R-sym_eta}
[Q_\pm^a,Q_\pm^b]=2\eta^{ab}\p_\pm,
\end{align}
where $\eta^{ab}=\begin{pmatrix}
1 & 0 \\
0 & -1
\end{pmatrix}$. From here it is easy to see that matrices $(M_\pm)^a_b$ that rotate the $\pm$-chirality charges among each other while preserving the commutation relations \eqref{eq:R-sym_eta},
\begin{align*}
[(M_\pm)^a_cQ_\pm^c,(M_\pm)^b_dQ_\pm^d]=2\eta^{ab}\p_\pm
\end{align*}
have to satisfy
\begin{align*}
(M_\pm)^a_c\eta^{cd}(M_\pm)^a_d=\eta^{ab},
\end{align*}
i.e. $M_\pm$ have to lie in $SO(1,1)$ for each subscript $\pm$ and the R-symmetry group is therefore $G_R=SO(1,1)_+\times SO(1,1)_-$.
\david{this is only relevant for the (2,2) formalism, will se if we need this or not}
The R-symmetry acts on the odd coordinates accordingly as
\begin{align*}
\begin{pmatrix}
\theta^1_\pm \\
\theta^2_\pm	
\end{pmatrix}
\mapsto
\begin{pmatrix}
\cosh(\alpha_\pm) & \sinh(\alpha_\pm) \\
\sinh(\alpha_\pm) & \cosh(\alpha_\pm)
\end{pmatrix}
\begin{pmatrix}
\theta^1_\pm \\
\theta^2_\pm	
\end{pmatrix}
\end{align*}
or, equivalently, $(\theta^\pm,\tth^\pm)\mapsto (e^{\alpha_\pm} \theta^\pm,e^{-\alpha_\pm}\tth^\pm)$. There are therefore two inequivalent embeddings of the Lorentz group $SO(1,1)$ into the R-symmetry group, we label them by $V$ and $A$:
\begin{align*}
R_V(\alpha)&:(\theta^\pm,\tth^\pm)\mapsto (e^\alpha \theta^\pm,e^{-\alpha}\tth^\pm)\\
R_A(\alpha)&:(\theta^\pm,\tth^\pm)\mapsto (e^{\pm \alpha} \theta^\pm,e^{\mp \alpha}\tth^\pm),
\end{align*}
mapping the supercharges as
\begin{align*}
R_V(\alpha):& (Q_\pm,\tl{Q}_\pm)\mapsto (e^{-\alpha}Q_\pm,e^{\alpha}\tl{Q}_\pm)\\
R_A(\alpha):& (Q_\pm,\tl{Q}_\pm)\mapsto (e^{\mp\alpha}Q_\pm,e^{\pm \alpha}\tl{Q}_\pm).
\end{align*}
The generators of these transformations are
\begin{align*}
F_{V/A}=\theta^+\frac{\p}{\p\theta^+}-\tth^+\frac{\p}{\p\tth^+}\pm \theta^-\frac{\p}{\p\theta^-}\mp\tth^-\frac{\p}{\p\tth^-}.
\end{align*}

Because $F_{V/A}$ has same commutators with other operators as $M$, the Lorentz generator, and additionally all $F_{V/A}$ and $M$ commute among themselves, we can define new "Lorentz" generators in one of the two following ways:
\begin{align*}
M_A=M+F_V,\quad M_B=M+F_A.
\end{align*}

\subsection{The $(2,2)$ parasupersymmetric $\sigma$-model}
It has been known since \cite{Zumino:1979et} that the $(1,1)$ supersymmetric $\sigma$-model \eqref{eq:(1,1)action} admits $(2,2)$ supersymmetry when $(M,g)$ is a K\"{a}hler manifold. In general, $(2,2)$ supersymmetry in fact requires the target to be {\bf generalized K\"ahler} \cite{Gualtieri:2003dx}, which is equivalent to a data of a bi-Hermitian geometry discovered by \cite{Gates:1984nk}. For $(2,2)$ para-SUSY, the story is analogous. It was shown in \cite{HullTwistedSUSY} that in order to extend the $(1,1)$ SUSY of \ref{eq:(1,1)action} to $(2,2)$ para-SUSY, one needs a para-complex version of the biHermitian geometry. In \cite{Hu:2019zro} it was then shown that this can be equivalently described by {\bf generalized para-K\"ahler} geometry.

\paragraph{Expanding the $(1,1)$ action}

First, we further analyze the $(1,1)$ supersymmetric theory. 
Recall the $(1,1)$ supersymmetric $\sigma$-model is described by the action
\begin{align*}
S_{(1,1)}(\Phi)&=\int_{\hat{\Sigma}}[g(\Phi)+b(\Phi)]_{ij}D_+\Phi^iD_-\Phi^j.
\end{align*}
In components, the superfield has the form
\[
\Phi^i =\phi^i+\theta^+\psi^i_++\theta^-\psi^i_-+\theta^+\theta^-F^i
\]
The super covariant derivatives $D_{\pm}$ applied to $\Phi^i$ read
\[
D_\pm\Phi^i =\psi^i_\pm\pm\theta^\mp F^i-\theta^\pm\p_\pm\phi^i\mp\theta^+\theta^-\p_\pm\psi^i_\mp .
\]
Further, we can expand the contributions from the metric and $b$-field as
\begin{align*}
[g(\Phi)+b(\Phi)]_{ij} = & [g(\phi)+b(\phi)]_{ij}+\p_k[g(\phi)+b(\phi)]_{ij}(\theta^+\psi_+^k+\theta^-\psi_-^k+\theta^+\theta^-F^k)\\
& +\frac{1}{2}\p_k\p_l[g(\phi)+b(\phi)]_{ij}(\theta^+\psi_+^k+\theta^-\psi_-^k)(\theta^+\psi_+^l+\theta^-\psi_-^l).
\end{align*}

After performing the odd integration, the only terms that survive are the $\theta^+\theta^-$ coefficients:
\begin{align*}
\theta^+\theta^-&\left[(g+b)_{ij}(\psi_+^i\p_-\psi^j_++\psi^j_-\p_-\psi^i_-+F^iF^j+\p_+\phi^i\p_-\phi^j)\right. \\
&+\p_k(g+b)_{ij}\left(F^k\psi^i_+\psi^j_-+\psi^k_+(-F^i\psi^j_--\psi^i_+\p_-\phi^j)+\psi^k_-(-\p_+\phi^i\psi^j_-+\psi_+^iF^j)\right)\\
&\left. +\frac{1}{2}\p_k\p_l(g+b)_{ij}(-\psi_+^k\psi_-^l\psi_+^i\psi_-^j+\psi_-^k\psi_+^l\psi^i_+\psi^j_-) \right].
\end{align*}

The field $F$ is auxiliary (since it enters only algebraically: no derivations of $F$ appear in the action), and therefore can be integrated out. To obtain the equations of motion for $F$, we use the formula $\p_kg_{ij}=\Gamma_{ikj}+\Gamma_{jki}$ and collect the terms involving $F$:
\begin{align*}
g_{ij}F^iF^j+&(\Gamma_{ikj}+\Gamma_{jki})(F^k\psi^i_+\psi^j_--F^i\psi^k_+\psi^j_-+F^j\psi^k_-\psi_+^i)\\
+&F^k\psi^i_+\psi^j_-(\p_kb_{ij}+\p_ib_{jk}+\p_jb_{ki}).
\end{align*}
Using $\Gamma_{ikj}=\Gamma_{kij}$ and $H_{ijk}=(\p_kb_{ij}+\p_ib_{jk}+\p_jb_{ki})$, we can rewrite this as
\begin{align*}
g_{ij}F^iF^j+2\Gamma_{ikj}F^j\psi^k_-\psi_+^i+F^k\psi^i_+\psi^j_-H_{ijk}=g_{ij}(F^iF^j+2F^j(\Gamma^-)_{lk}^i\psi_-^k\psi_+^l),
\end{align*}
where $\Gamma^-$ are the Christoffel symbols of the connection $\n^-=\lc-\frac{1}{2}g^{-1}H$. This yields the equation of motion for $F$,
\begin{align} \label{F_EoM}
\frac{\delta S_{(1,1)}}{\delta F^i}=0\quad \Longleftrightarrow\quad  F^i=-(\Gamma^-)^i_{jk}\psi^k_-\psi^j_+
\end{align}

%\paragraph{(2,2) SUSY}

%We propose the following symmetry by the $(2,2)$ parasupersymmetry algebra (\ref{(2,2)para}) (setting $\theta^\pm_1\coloneqq \theta^\pm$ for simplicity), which read in superspace notation:
%\begin{align*}
%Q^1_\pm&=\frac{\p}{\p\theta^\pm}+\theta^\pm\p_\pm \\
%Q^2_\pm\Phi^i&=(K_\pm)^i_j(\Phi)D_\pm\Phi^j \\
%\end{align*}
%where $D_\pm =\frac{\p}{\p\theta^\pm}-\theta^\pm\p_\pm$. 


\paragraph{Extended supersymmetry}

The $\sigma$-model we have just described has manifest $(1,1)$ supersymmetry. 
We now propose the a symmetry of the action $S_{(1,1)}$ by the extended $(2,2)$ parasupersymmetry algebra (\ref{(2,2)para}).
In general, this extended symmetry exists only after we make some additional assumptions on the geometric data prescribing the action. 

Setting $\theta^\pm_1\coloneqq \theta^\pm$ for simplicity, we propose the following $(2,2)$ parasupersymmetry in $(1,1)$ superfield notation:
\begin{align}
\label{q1} Q^1_\pm&=\frac{\p}{\p\theta^\pm}+\theta^\pm\p_\pm \\
\label{q2} Q^2_\pm\Phi^i&=(K_\pm)^i_j(\Phi)D_\pm\Phi^j
\end{align}
where $D_\pm =\frac{\p}{\p\theta^\pm}-\theta^\pm\p_\pm$. 
\brian{We need to either recall what $K_{\pm}$ is geometrically, or state that it is a free parameter which we will interpret momentarily.}
In other words, the infinitesimal action of the element $Q^1_{\pm} \in \mathfrak{t}_{(2,2)}$ is:
\[
\Phi^i \mapsto \Phi^i + \epsilon^{\pm}_1 \left( \frac{\p}{\p\theta^\pm}+\theta^\pm\p_\pm \right) \Phi^i
\]
where $\epsilon^{\pm}_1$ is an infinitesimal parameter. 
Similarly, the infinitesimal action by the element $Q^2_{\pm} \in \mathfrak{t}_{(2,2)}$ is:
\[
\Phi \mapsto \Phi + \epsilon^{\pm}_2 (K_\pm)^i_j(\Phi)D_\pm\Phi^j 
\]
where $\epsilon^\pm_2$ is an infinitesimal parameter. 

\begin{lemma}
Expanding out in the components of the $(1,1)$ superfield, the infinitesimal action by $Q^1_{+}$ and in Equation (\ref{q1}) reads:
\[
Q^1_{+} : (\phi^i, \psi^i_+, \psi^i_-, F^i) \mapsto (\phi^i, \psi^i_{\pm}, F^i) + \epsilon^+_1 (\psi^i_+, \partial_+ \phi, F^i, \partial_+ \psi^i_-) . 
\]
and by $Q^1_-$:
\[
Q^1_{-} : (\phi^i, \psi^i_+, \psi^i_-, F^i)  \mapsto (\phi^i, \psi^i_{\pm}, F^i) + \epsilon^-_1 (\psi^i_-, \partial^i_- \phi^i, - F^i, - \partial_- \psi^i_+)
\]
Similarly, the infinitesimal action by $Q^2_{+}$ in Equation (\ref{q2}) reads:
\begin{equation}\label{q2plus}
Q^2_{+} : (\phi^i, \psi^i_+, \psi^i_-, F^i) \mapsto ((K_+)^i_j \psi_+ ^j , - (K_+)^{i}_{j} \partial_+ \phi^j - \partial_k(K_+)^i_j \psi_+^k \psi_+^k, (K_+)^i_j F^j+\p_k(K_+)^i_j\psi^k_-\psi^j_+, \brian{F term})
\end{equation}
and for $Q^2_-$:
\begin{equation}\label{q2minus}
Q^2_{-} : (\phi^i, \psi^i_+, \psi^i_-, F^i) \mapsto ((K_-)^i_j \psi_- ^j , - (K_-)^{i}_{j} \partial_- \phi^j - \partial_k(K_-)^i_j \psi_-^k \psi_-^k, (K_-)^i_j F^j+\p_k(K_-)^i_j\psi^k_+\psi^j_-, \brian{F term})
\end{equation}
\end{lemma}

%\begin{align}\label{Q12_action}
%\begin{aligned}
%[Q_\pm^1 ,\phi^i]&=\psi_\pm^i,& [Q_\pm^2 ,\phi^i] &=(K_\pm)^i_j\psi_\pm^j,\\
%[Q_\pm^1 ,\psi^i_\pm] &=\p_\pm\phi^i,& [Q_\pm^2 ,\psi^i_\pm] &=-(K_\pm)^i_j\p_+\phi^j{\textcolor{red}{+}}\p_k(K_\pm)^i_j\psi_+^k\psi_+^j\\
%[Q_\pm^1 ,\psi^i_\mp]&=\pm F^i,& [Q_\pm^2 ,\psi^i_\mp] &=(K_\pm)^i_jF^j+\p_k(K_\pm)^i_j\psi^k_\mp\psi^j_\pm.
%\end{aligned}
%\end{align}
%
%\textcolor{red}{There's a sign issue above, the $+$ needs to be $-$.}
\begin{proof}
Recall, in components the superfield is of the form
\[
\Phi^i =\phi^i+\theta^+\psi^i_++\theta^-\psi^i_-+\theta^+\theta^-F^i .
\]
The action of the $Q^1_\pm$ charges is straightforward and can be simply read off from the infinitesimal action. 
%\begin{align*}
%[Q^1_\pm,\Phi]=[Q_\pm^1,\phi^i]+[Q_\pm^1,\psi^i_\pm]\theta^\pm+[Q_\pm^1,\psi^i_\mp]\theta^\mp+[Q_\pm^1,F]\theta^\pm\theta^\mp.
%\end{align*}
%For instance $[Q_{\pm}^1, \phi^i] = \theta^{\pm} \partial_{\pm} \phi^i$, thus 
To obtain the action by the charges $Q^2_\pm$ on the individual components we first need to expand $K_\pm(\Phi)$ in terms of the components of the superfield:
\begin{align*}
(K_\pm)^i_j(\Phi)=(K_\pm)^i_j(\phi)+\p_k(K_\pm)^i_j\theta^+\psi_+^k+\p_k(K_\pm)^i_j\theta^-\psi_-^k+\p_k(K_\pm)\theta^+\theta^-F^k,
\end{align*}
so that the infinitesimal action by $Q^2_\pm$ on $\Phi^i$ is:
\begin{align}
\Phi^i + \epsilon_2^\pm (K_\pm)^i_j(\Phi)(\psi^j_\pm\pm\theta^\mp F^j-\theta^\pm\p_\pm\phi^j\mp\theta^+\theta^-\p_\pm\psi^j_\mp) .
\end{align}
Reading off the various components this expression yields Equations (\ref{q2plus}) and (\ref{q2minus}).
\end{proof}



\paragraph{The $(2,2)$ formalism}
\brian{
There is the following superspace model for $(2,2)$ parasupersymmetry. 
Consider the superspace $\RR^{1,1 | 4}$ 
}

\subsection{$(2,1)$ and $(2,0)$ parasupersymmetry}

Many of the constructions above make sense when there is less total supersymmetry, and we briefly mention two important cases. 

The first concerns the $(2,1)$ parasupersymmetry algebra.
This algebra $\mathfrak{t}_{(2,1)}$ consists of generators $\{Q^1_\pm, Q^2_+\}$ satisfying the same algebra as in (\ref{(2,2)para}). 
Starting with the $(1,1)$ supersymmetric $\sigma$-model (\ref{eq:(1,1)action}), and labeling the odd coordinates by $\theta^\pm = \theta_1^\pm$, one can ask for an additional odd symmetry 
\[
Q_2^+ :  \Phi \mapsto \Phi + \epsilon_2^+ K_+(\Phi)^j_i D_+ \Phi^i
\]
which determines a symmetry by the $(2,1)$ parasupersymmetry algebra.

\begin{proposition}
The odd vector field $Q_2^+$ is a symmetry of the $(1,1)$ supersymmetric $\sigma$-model provided $(K_+)^{j}_i$ is a paracomplex structure satisfying \brian{What conditions here?}
Thus, the $(1,1)$ supersymmetric $\sigma$-model has $(2,1)$ parasupersymmetry provided the target is \brian{same conditions}. 
\end{proposition}

The next symmetry algebra we wish to consider is the $(2,0)$ parasupersymmetry algebra $\mathfrak{t}_{(2,0)}$. 
This algebra has the same even generators as the $(2,1)$ and $(2,2)$ algebras, and odd generators $\{Q^1_+, Q^2_+\}$ satisfying the relations
\[
[Q^1_+, Q^1_+] =  2 \partial_\pm \;\;\; , \;\;\; [Q^2_+, Q^2_+] = - 2 \partial_\pm
\]

\brian{The (1,0) action}

We propose an extended symmetry of the $(1,0)$ $\sigma$-model by the following action by the odd vector field $Q_+^2$:
\[
Q^2_+ : \Phi \mapsto \Phi + \epsilon_2^+ K_+(\Phi)^j_i D_+ \Phi^i .
\]

\begin{proposition}
The odd vector field $Q_2^+$ is a symmetry of the $(1,0)$ supersymmetric $\sigma$-model provided $(K_+)^{j}_i$ is a paracomplex structure. 
Thus, the $(1,0)$ supersymmetric $\sigma$-model has $(2,0)$ parasupersymmetry provided the target is paracomplex. \brian{there is a version for a target vector bundle as well...}
\end{proposition}

\section{The AKSZ formalism and 3d TFT}
\def\fg{\mathfrak{g}}
In this section we will show how to associate $2D$ topological theories to a generalized (para-)complex structure and relate them to the topological twists of $2D$ $(2,2)$ (para-)SUSY sigma models described in Section \ref{sec:toptwist} and in \cite{Kapustin:2004gv}. 
The key to our construction is to realize the data of a G(p)C structure in terms of its eigenbundles, which are Dirac structures $L$ in the Courant algebroid $\TT$. 
Via the AKSZ construction \cite{AKSZ}, this correspondence translates into the statement that the Dirac structure $L$ defines a topological boundary theory of a $3D$ topological theory defined by the Courant algebroid $\TT$ \cite{Roytenberg:2002nu}.

Throughout, we will use symplectic $NQ$ manifold as our model for shifted symplectic geometry. 
We will recall the general setup of associating a $3D$ topological field theory to the data of a Courant algebroid. 
For a reference, see \cite{Roytenberg:2002nu,??}.

\subsection{Courant algebroids, $3D$ TFTs, and boundary conditions}
From the point of view of topological field theory, Courant algebroids are important because they provide geometric examples of $2$-shifted symplectic spaces. 
Via the AKSZ construction, $2$-shifted symplectic spaces are the natural home for $3$-dimensional topological field theories in the BV formalism. 
\brian{give citations here.}

In \cite{Roytenberg:2002nu}, it is shown how to associate a $2$-shifted symplectic NQ manifold from the data of a Courant algebroid. 

\begin{theorem}[\cite{Roytenberg:2002nu}]
There is an equivalence between isomorphism classes of $2$-shifted symplectic NQ manifolds with body $M$ and isomorphism classes of Courant algebroids on $M$.
\end{theorem}

We briefly recall this construction. 

Recall, the data of a Courant algebroid on a manifold $M$ is a vector bundle $E$, whose sheaf of sections we denote $\cE$, together with a nondegenerate symmetric bilinear pairing $\langle -, -\rangle : \cE \otimes \cE \to C^\infty(M)$ , an anchor map $a : E \to T M$, and a bilinear operator
\[
[-,-] : \cE \times \cE \to \cE
\]
called the Courant-Dorfman bracket. 
\brian{spell out properties, find a good reference}
Denote by $a^* : T^* M \to E^*$ the bundle map dual to $a$. 

To every Courant algebroid we can define the following sequence of vector bundles on $M$
\[
T^*M \xto{a^*} E \xto{a} TM .
\]
A Courant algebroid is called {\em exact} if this sequence of bundles is exact. 

\def\Sym{{\rm Sym}}

Using this sequence, we can define the associated $NQ$ manifold $X_E$, following \cite{Ryotenberg:2002nu}.
The body of the NQ manifold is $M$, and the underlying graded vector bundle is given by $T^*[2] \oplus E[1]$. 
Define the degree $+1$ vector field 
\[
Q : \Gamma(M , \Sym \left(T^*[2] \oplus E[1]\right)^*) \to \Gamma(M , \Sym \left(T^*[2] \oplus E[1]\right)^*) 
\]
as follows: for $\psi \in \Sym^{k + \ell + 2} \left(T^*[2] \oplus E[1]\right)^*)$ and
\[
(\xi_0 \otimes \cdots \otimes \xi_k) \otimes (\alpha_0 \otimes \cdots \otimes \alpha_\ell) \in \Sym^{k+1}(T^*[2]) \otimes \Sym^{\ell + 1}(E[1]) 
\]
define
\begin{align*}
(Q \psi) ((\xi_0 \otimes \cdots \otimes \xi_k) \otimes (\alpha_0 \otimes \cdots \otimes \alpha_\ell)) = \brian{finish}
\end{align*}

All that remains is to describe the $2$-shifted symplectic structure on the NQ manifold $X_E$. 
This is defined via the obvious pairing between $T^*[2]$ and $T$ together with the pairing $\langle - ,- \rangle$ on $E[1]$.
In local coordinates \brian{finish}

The simplest type of Courant algebroid is one over a point.

\begin{Ex}\label{ex: cs}
Any Lie algebra $\fg$ together with a non-degenerate invariant pairing defines a $2$-shifted symplectic structure on the graded manifold $\fg[1]$ living over a point. 
The dg algebra of functions is the Chevalley-Eilenberg cochain complex computing Lie algebra cohomology $C^*(\fg)$. 
All such $2$-symplectic NQ manifolds over a point are of this form. 
%The resulting AKSZ theory is Chern-Simons theory. 
\end{Ex}

\subsubsection*{The $2$-shifted symplectic space of an exact Courant Algebroid}

We recall the description of the $2$-shifted symplectic NQ manifold $X_E$ associated to an exact Courant algebroid $E$.
Of course, as a vector bundle $E = \TT$, and hence as a graded manifold $X_E$ is of the form $T^*[2] T[1] M$.
Concretely, the coordinates on this manifold are given by $(x^i,v^a,p_i,\mu_a)_{i,a=1\cdots n}$, where $x^i$ are the usual coordinates on $M$ with degree $0$, $v^a$ are the corresponding ``velocities'', i.e. the fibre coordinates on $T[1]M$ with a degree $1$, and $p_i,\mu_a$ are the momenta corresponding to $x^i,v^a$ with degrees $2,1$, respectively. The canonical symplectic form
\begin{align*}
\Omega= dx^idp_i+dv^ad\mu_a,
\end{align*}
is then of degree $2$. $T^*[2]T[1]M$ is the minimal symplectic realization of the shifted bundle $(\TT)[1]M$, which is recovered by setting $p_i=0$. The symmetric part of $\Omega$ represents the pairing $\lara$ on $(\TT)M$; this is because we can identify $d\mu_a\leftrightarrow \p_i$ and $dv^a\leftrightarrow dx^i$

\brian{Severa classification}

\begin{theorem}\label{thm: severa}
The set of isomorphism classes of exact Courant algebroids is a torsor for $H^3(X)$. 
\end{theorem}

\subsubsection*{The Poisson Courant algebroid}

\brian{Both real and holomorphic versions}

\subsubsection{The AKSZ theory associated to a Courant algebroid}

To any Courant algebroid $E$, we can associate an NQ manifold $X_E$ which carries a $2$-shifted symplectic structure that we denote by $\omega_E$. 
In the case that the Courant algebroid is exact, call the associated $2$-shifted symplectic space $X_H$, where $H$ labels the Severa class. 

\def\sA{\mathcal{A}}
\def\d{{\rm d}}
\def\dbar{\Bar{\partial}}

The starting point of the AKSZ construction is the mapping space
\[
{\rm Map}\left((M, (\sA^*_M, \d_M)), X_E\right)
\]
where $(M, (\sA^*_M, \d_M))$ is a dg manifold with body a smooth oriented manifold $M$. 
That is, $\sA^*_M$ is a sheaf of graded commutative algebras over the de Rham complex $\Omega^*_M$ and $\d_M$ is a linear operator of degree $+1$ satisfying:
\begin{itemize}
\item[(1)] $\sA_M$ is concentrated in finitely many degrees;
\item[(2)] For each $k$, $\sA^k_M$ is a locally free sheaf of $C^\infty_M$-modules of finite rank;
\item[(3)] The differential $\d_M : \sA_M^* \to \sA_M^{*+1}$ is square zero differential operator making $(\sA_M , \d_M)$ into a sheaf of commutative dg algebras over the de Rham complex $\Omega^*_M$
\end{itemize}

\begin{Ex}
The most important example for us will be the source dg manifold $(M, \Omega^*_M)$, that is $\sA_M = \Omega^*_M$ with de Rham differential. 
We denote this dg manifold by $M_{\rm dR}$.
\end{Ex}

\begin{Ex}
If $M$ has a para-complex structure, then $(M, \Omega^{0,*}_M)$ has the structure of a dg manifold with differential given by the para $\dbar$-operator. 
\brian{David, is this consistent notation?}
We will denote this dg manifold by $M_{p \dbar}$. 
\end{Ex}

\begin{Ex}
If $\fg$ is a Lie algebra equipped with a non-degenerate invariant pairing, and $X_E = \fg[1]$ as in Example \ref{ex: cs}, then the AKSZ theory (whose target base manifold is a point) is equivalent to Chern-Simons theory on $M$. 
\end{Ex}

\subsubsection{Dirac structures and boundary conditions}

\subsection{Two dimensional TFT from GpC structures}

\brian{Recall that a generalized (para-)complex structure $\JJ$ defines a Dirac structure in $\TT$, and hence a Lagrangian inside of the $2$-shifted symplectic space $\XX_H$.  
%More generally, if $\JJ$ is a generalized complex structure in an arbitrary Courant algebroid $E$, then we obtain a $2$-shifted Lagrangian inside of the $2$-shifted symplectic space $\XX_E$. 
This Lagrangian then defines a boundary condition for the three-dimensional AKSZ theory. }

\subsubsection*{Trivial GpC structure}

\subsubsection*{(Real) Poisson $\sigma$-model}

\subsubsection*{Para-complex $A/B$-models}

We point out two special cases of the above general construction. 

\brian{The $A$-model is the usual thing.}

\begin{Def}
The {\bf para-complex $B$-model} with source a closed Riemann surface $\Sigma$ and target a para-complex manifold $X$ is the AKSZ theory with source the de Rham space $\Sigma_{\rm dR}$ and target the $1$-shifted symplectic space $T^*[1] X_{p \dbar}$:
\[
{\rm Map}\left(\Sigma_{dR}, T^*[1]  X_{p \dbar}\right) .
\]
\end{Def}

\brian{Discuss Si's work on perturbative quantization}

\hrulefill

\brian{***para Kahler example, **example where the paracomplex structures commute but are not equal (this is where people see ``twisted chiral"), para version of T-duality, **Poisson $\sigma$-model, going back and forth between generalized (para) complex.
}  

 \brian{discuss anomalies}

\subsection{Para holomorphic variants}

So far, in each of the $\sigma$-models we have discussed in the AKSZ formalism, the fields have depended only topologically on the source Riemann surface. 
There are closely related $\sigma$-models which depending {\em para-holomorphically} on the source Riemann surface that we briefly discuss. 



\section{Twisted parasupersymmetry}\label{sec:toptwist}

In this section we will present how para-supersymmetry theories can be twisted, in exact analogy with twisting in ordinary supersymmetry.
Applied to the $(2,2)$ parasupersymmetric theories that we have constructed in \S \ref{sec: parasusy}, we find that twisting yield topological field field theories. 
In the next section, we will construct these topological theories geometrically using the AKSZ formalism. 

The procedure for twisting will again follow the analogy between generalized K\"{a}hler (GK) and generalized para-K\"{a}hler (GpK). geometries, and we will follow the approach presented in \cite{Kapustin:2004gv} for the GK case.

\brian{geometry, change in gradings}
\david{how does the embedding of $SO(1,1)$ into $G_R$ reflect on the definition of $\QQ$?}

For the topological twist, we will use the following nilpotent supercharge
\begin{align}
\QQ=Q_L+Q_R,\quad Q_L=\frac{1}{2}(Q^1_++Q^2_+),\quad Q_R=\frac{1}{2}(Q^1_-+Q^2_-),
\end{align}
and we also introduce the usual notation 
\begin{align*}
\chi=\frac{1}{2}(\id+K_+)\psi_+=P_+\psi_+=\psi_+^{(1,0)_+},\quad\lambda=\frac{1}{2}(\id+K_-)\psi_-=P_-\psi_-=\psi_-^{(1,0)_-}.
\end{align*}

\begin{proposition}
The action of $Q_{L/R}$ on the fields $\chi$, $\lambda$ and $\phi$ is given by
\begin{align}\label{eq:Qcoh}
\begin{aligned}
[Q_L,\phi^i]&=\chi^i & [Q_R,\phi^i]&=\lambda^i\\
[Q_L,\chi^i]&=0 & [Q_R,\lambda^i]&=0\\
[Q_L,\lambda^i]&=-(\Gamma^-)^i_{jk}\chi^j\lambda^k & [Q_R,\chi^i]&=-(\Gamma^+)^i_{jk}\lambda^j\chi^k.
\end{aligned}
\end{align}
\end{proposition}
\begin{proof}
The action of $Q_{L/R}$ on $\phi$ can be immediately read off \eqref{Q12_action}:
\begin{align}
[Q_L,\phi^i]=\frac{1}{2}[Q_+^1+Q^2_+,\phi^i]=\chi^i.
\end{align}

We now prove the formulas for the action of $Q_{L/R}$ on $\chi$, the action on $\lambda$ is entirely analogous. First, we expand
\begin{align}\label{QL_chi}
[Q_L,\chi^i]=[Q_L,(P_+)^i_j\psi_+^j]=[Q_L,(P_+)^i_j]\psi_+^j\textcolor{red}{+}(P_+)^i_j[Q_L,\psi_+^j].
\end{align}
\textcolor{red}{Or maybe here? I think this is the one}
Because $K_+$ is a function of the bosonic coordinates, we also have $P_+=P_+(\phi)$ and so
\begin{align}\label{QL_P}
[Q_L,(P_+)^i_j]=\p_k(P_+)^i_j[Q_L,\phi^k]=\p_k(P_+)^i_j\chi^k.
\end{align}
Combining \eqref{QL_chi} and \eqref{QL_P} and using \eqref{Q12_action}, we arrive at
\begin{align*}
[Q_L,\chi^i]&=\p_k(P_+)^i_j\chi^k\psi_+^j\textcolor{red}{\overset{?}{\pm}}(P_+)^i_j[(\tl{P}_+)^j_k\p_+\phi^k\textcolor{red}{+}\frac{1}{2}\p_k(K_+)^j_l\psi_+^k\psi_+^l]\\
&=\p_k(P_+)^i_j(P_+)^k_l\psi_+^l\psi_+^j\textcolor{red}{-}(P_+)^i_j\p_k(P_+)^j_l\psi_+^k\psi_+^l,
\end{align*}
where we denoted $\tl{P}_+=\frac{1}{2}(\id-K_+)$ and used $P_+\tl{P}_+=0$. We now expand the projectors, $(P_+)^i_j=\frac{1}{2}(\delta^i_j+(K_+)^i_j)$:

\begin{align*}
[Q_L,\chi^i]&=\p_k(P_+)^i_j(P_+)^k_l\psi_+^l\psi_+^j\textcolor{red}{-}(P_+)^i_j\p_k(P_+)^j_l\psi_+^k\psi_+^l\\
&=\frac{1}{2}(\p_j(K_+)^i_l(P_+)^j_k\textcolor{red}{-}(P_+)^i_j\p_k(K_+)^j_l)\psi_+^k\psi_+^l\\
&=\frac{1}{4}(\p_j(K_+)^i_l(K_+)^j_k\textcolor{red}{-}(K_+)^i_j\p_k(K_+)^j_l)\psi_+^k\psi_+^l\\
&+\frac{1}{4}(\p_j(K_+)^i_l\psi_+^j\psi_+^l\textcolor{red}{-}\p_k(K_+)^i_l\psi_+^k\psi_+^l)=-\frac{1}{2}(N_{K_+})^i_{kl}\psi_+^k\psi^l_+,
\end{align*}
\textcolor{red}{This must be the right sign choice} where in the last line we rewrote $\psi_+^k\psi^l_+=\frac{1}{2}(\psi^k\psi^l-\psi^l\psi^k)$ to get the expression for the Nijenhuis tensor of $K_+$:
\begin{align*}
4(N_{K_+})^i_{kl}=(K_+)^j_k\p_j(K_+)^i_l-(K_+)^j_l\p_j(K_+)^i_k \textcolor{red}{-} (K_+)^i_j(\p_k(K_+)^j_l-\p_l(K_+)^j_k).
\end{align*}
Invoking integrability of $K_+$, $N_{K_+}=$ and we conclude that $[Q_L,\chi^i]=0$.

Similarly, $[Q_L,\lambda]$ can be expanded
\begin{align}\label{QL_lambda}
[Q_L,\lambda^i]&=[Q_L,(P_-)^i_j\psi_-^j]=[Q_L,(P_-)^i_j]\psi_-^j+(P_-)^i_j[Q_L,\psi_-^j]\\
&=\p_k(P_-)^i_j\chi^k\psi_-^j+(P_-)^i_k((P_+)^k_jF^j+\frac{1}{2}\p_l(K_+)^k_j\psi^l_-\psi_+^j),
\end{align}
where we used
\begin{align*}
[Q_L,\psi_-^i]=(P_+)^i_jF^j+\frac{1}{2}\p_k(K_+)^i_j\psi^k_-\psi_+^j.
\end{align*}
Next, we rewrite the derivative in terms of $\n^-$ and use the property $\n^-P_-=0$:
\begin{align*}
\p_k(P_-)^i_j&=\n^-_k(P_-)^i_j-(\Gamma^-)^i_{kl}(P_-)^l_j+(\Gamma^-)^l_{kj}(P_-)^i_l\\
&=-(\Gamma^-)^i_{kl}(P_-)^l_j+(\Gamma^-)^l_{kj}(P_-)^i_l,
\end{align*}
so that
\begin{align}\label{eq:calc1}
\begin{aligned}
\p_k(P_-)^i_j\chi^k\psi_-^j=(-(\Gamma^-)^i_{kl}(P_-)^l_j+(\Gamma^-)^l_{kj}(P_-)^i_l)(P_+)^k_m\psi_+^m\psi_-^j.
\end{aligned}
\end{align}
Similarly,
\begin{align}\label{eq:calc2}
\begin{aligned}
\frac{1}{2}(P_-)^i_k\p_l(K_+)^k_j\psi^l_-\psi_+^j&=(P_-)^i_k\p_l(P_+)^k_j\psi^l_-\psi_+^j\\
&=(P_-)^i_k(-(\Gamma^-)^k_{ml}(P_+)^m_j+(\Gamma^-)^m_{jl}(P_+)^k_m)\psi^l_-\psi_+^j,
\end{aligned}
\end{align}
where we used
\begin{align*}
(\Gamma^+)^i_{jk}=g^{im}(\Gamma_{mjk}+\frac{1}{2}H_{jkm})=g^{im}(\Gamma_{mkj}-\frac{1}{2}H_{kjm})=(\Gamma^-)^i_{kj}.
\end{align*}
Plugging \eqref{eq:calc1} and \eqref{eq:calc2} into \eqref{QL_lambda} with use of the equation of motion for $F$ \eqref{F_EoM} then yields
\begin{align*}
[Q_L,\lambda^i]&=((\Gamma^-)^i_{km}(P_-)^m_l-(\Gamma^-)^m_{kl}(P_-)^i_m)(P_+)^k_j\psi_-^l\psi_+^j\\
&+(P_-)^i_k(-(\Gamma^-)^k_{ml}(P_+)^m_j+(\Gamma^-)^m_{jl}(P_+)^k_m)\psi^l_-\psi_+^j\\
&-(P_-)^i_k(P_+)^k_m(\Gamma^-)^m_{jl}\psi_-^l\psi_+^j
 \end{align*}
  


 terms cancel (\textcolor{red}{up to another relative minus sign!}) as well as cancelling the term $(P_-)^i_k(P_+)^k_jF^j$ by
\begin{align*}
(P_-)^i_k(\Gamma^+)^m_{lj}(P_+)^k_m\psi^l_-\psi_+^j=(P_-)^i_k(P_+)^k_j(\Gamma^-)^j_{ml}\psi^l_-\psi_+^m=-(P_-)^i_k(P_+)^k_jF^j
\end{align*}
and we arrive at
\begin{align*}
\{Q_L,\lambda^i\}=-(\Gamma^-)^i_{kl}\chi^k\lambda^l.
\end{align*}

Now for $Q_R$, we have
\begin{align*}
\{Q_R,\phi^i\}=\lambda^i
\end{align*}
\end{proof}

The observables of the twisted theory are therefore polynomials in the odd variables $\chi,\lambda$ with even coefficients dependent on $\phi$:
\begin{align*}
{\cal O}_f=f(\phi)_{i_1\cdots i_a,j_1\cdots j_b}\chi^{i_1}\cdots\chi^{i_a}\lambda^{j_1}\cdots\lambda^{j_b}.
\end{align*}
Because $\chi \in \XX^{(1,0)_+}$ and $\lambda \in \XX^{(1,0)_-}$, we can identify ${\cal O}_f$ with a function on the supermanifold $(T^{(1,0)_+}\oplus T^{(1,0)_-})[1]$, which in turn can be identified with differential form
\begin{align*}
\Omega_f=f(\phi)_{i_1\cdots i_a,j_1\cdots j_b}dx^{i_1}_+\cdots dx^{i_a}_+ dx^{j_1}_- \cdots dx^{j_b}_-,
\end{align*}
where $dx_\pm$ are the one-forms satisfying $K_\pm dx_\pm=dx_\pm$, i.e. sections of the bundle $T^{*(1,0)_\pm}$. Via this identification, the operator ${\cal Q}\coloneqq Q_L+Q_R$ defines a differential $\rd_\QQ$, which acts on the sections of $\bigwedge^k(T^{*(1,0)_+}\oplus T^{*(1,0)_-})$ and because $\QQ$ squares to zero $\rd_\QQ$ also squares to zero. Therefore, $\rd_\QQ$ gives rise to a Chevalley-Eilenberg complex $\text{CE}(L)$ of a certain Lie algebroid $L$ which we now describe.

\paragraph{Lie algebroid from bi-para-Hermitian data} The underlying vector bundle of the Lie algebroid is $L=T^{(1,0)_+}\oplus T^{(1,0)_-}$, and the anchor is the sum of the two inclusions $\imath_\pm:\ T^{(1,0)_\pm}\hookrightarrow T$, i.e.
\begin{align*}
a:L=T^{(1,0)_+}\oplus T^{(1,0)_-} &\rightarrow T\\
 e=(x_+,x_-) &\mapsto \imath_+(x_+)+\imath_-(x_-).
\end{align*}
From \eqref{eq:Qcoh} we can read off the action of $\rd_\QQ$ on degree one elements, i.e. sections $\alpha=(\alpha^+,\alpha^-)$ of $L^*=T^*_{(1,0)_+}\oplus T^*_{(1,0)_-}$:
\begin{align*}
\rd_\QQ\alpha&=\p_k\ap_i^+[dx_+^k+dx_-^k]\w dx_+^i-\ap^+_i(\Gamma^+)^i_{jk}dx_-^j\w dx_+^k\\
&+\p_k\ap_i^-[dx_+^k+dx_-^k]\w dx_-^i-\ap^-_i(\Gamma^-)^i_{jk}dx_+^j\w dx_-^k.
\end{align*}
%In coordinate-free form, we get
%\begin{align*}
%\rd_\QQ\ap=(\p_++\n^+_{P_-(\bullet)})\ap^++(\p_-+\n^-_{P_+(\bullet)})\ap^-,
%\end{align*}
By contracting in two sections of $L$, $e_1=x_++x_-$, $e_2=y_-+y_-$, we get the coordinate-free expression
\begin{align}\label{eq:dQ}
\begin{aligned}
(\rd_\QQ\ap)(e_1,e_2)&=(\p_+\ap^+)(x_+,y_+)+\n^+_{x_-}\ap^+(y_+)-\n^+_{y_-}\ap^+(x_+)\\
&+(\p_-\ap^-)(x_-,y_-)+\n^-_{x_+}\ap^-(y_-)-\n^-_{y_+}\ap^-(x_-),
\end{aligned}
\end{align}
where $\p_\pm$ denotes the para-complex Dolbeault operators $\p$ for $K_\pm$. We can further simplify this expression by writing $\ap=(\eta(\tl{z}_+),\eta(\tl{z}_-))$ for $\tl{z}_\pm \in \se(T^{(1,0)_\pm})$, invoking the formula $\p_\pm\ap^\pm(x_\pm,y_\pm)=\lc_{x_\pm}\ap^\pm(y_\pm)-\lc_{y_\pm}\ap^\pm(x_\pm)$ and recalling $\n^\pm=\lc\pm \frac{1}{2}\eta^{-1}H$:
\begin{align}
\begin{aligned}
(\rd_\QQ\ap)(e_1,e_2)&=\eta((\lc_{x_+}+\n_{x_-}^+)\zt_+,y_+)-\eta((\lc_{y_+}+\n_{y_-}^+)\zt_+,x_+)\\
&+\eta((\lc_{x_-}+\n_{x_+}^-)\zt_-,y_-)-\eta((\lc_{y_-}+\n_{y_+}^-)\zt_-,x_-)\\
&=\eta(\n^+_{x_++x_-}\zt_+,y_+)-\eta(\n_{y_++y_-}^+\zt_+,x_+)-H(x_+,\zt_+,y_+)\\
&+\eta(\n_{x_++x_-}^-\zt_-,y_-)-\eta(\n_{y_++y_-}^-\zt_-,x_-)+H(x_-,\zt_-,y_-)
\end{aligned}
\end{align}


\paragraph{Lie algebroid from generalized para-K\"ahler data} Recall that a GpK structure defines the decomposition \eqref{GpK_bundles}
\begin{align*}
\TT=\ell_+\oplus\ell_-\oplus \ellt_+\oplus \ellt_-,
\end{align*}
where the the eigenbundles of $\KK_+$ are $\Lb_+=\ell_+\oplus\ell_-$ and $\tl{\Lb}_+=\ellt_+\oplus \ellt_-$. Because $\Lb$ is a Dirac structure, it defines a Lie algebroid by restriction of the Courant algebroid on $\TT$ to $\Lb$, i.e. the Lie algebroid bracket and anchor are given by the restriction of the Dorfman bracket and the projection onto tangent bundle, respectively (see Proposition \ref{prop:dirac_Liealg}.

\begin{theorem}
The Lie algebroids $(L,a,\rd_\QQ)$ and $(\Lb_+,\pi_T\!\!\mid_{\Lb_+},\brac\!\!\mid_{\Lb_+})$ are isomorphic via the map $\pi=\pi_+\oplus \pi_-$.
\end{theorem}
\begin{proof}
First, we notice that the bundles themselves are related by $\pi$,  $\pi(\Lb_+)=\pi(\ell_+\oplus \ell_-)=\pi_+(\ell_+)\oplus \pi_-(\ell_-)=T^{(1,0)_+}\oplus T^{(1,0)_-}=L$ and also $\pi$ is clearly an isomorphism since both $\pi_\pm$ are. Additionally, the anchors are related by $a\circ\pi=\pi_T\!\!\mid_{\Lb_+}$. The only non-trivial task is therefore to show that the respective differentials satisfy
\begin{align}\label{eq:d_pi}
\pi^*\circ\rd_Q  =\rd_{\Lb_+}\circ \pi^*,
\end{align}
where $\pi^*=\pi^*_+\oplus\pi_-^*$ and $\pi^*_\pm$ are dual maps to $\pi_\pm:\ell_+\rightarrow T^{(1,0)_\pm}$, extended to a map $\pi^*:\Lambda^k(L^*)\rightarrow\Lambda^k(\Lb_+^*)$.

Because of the chain rule, we only need to check this on degree $0$ and degree $1$ elements in the respective complexes. Degree $0$ follows immediately:
\begin{align*}
(\rd_Q f)(\pi(u))= a\circ \pi(u)[f]=\pi_T\!\!\mid_{\Lb_+}(u)[f]=(\rd_{\Lb_+}f)(u),
\end{align*}
%For degree one, we use the decomposition of the respective bundles and check for each component
%\begin{alignat*}{2}
%\rd_Q:&&T^*_{(1,0)_+}\oplus T^*_{(1,0)_-} &\rightarrow \Lambda^2(T^*_{(1,0)_+}\oplus T^*_{(1,0)_-})\\
%\rd_{\Lb_+}:&& \ell_+^*\oplus \ell_-^* &\rightarrow \Lambda^2(\ell_+^*\oplus \ell_-^*)
%\end{alignat*}
and we will now prove \ref{eq:d_pi} for degree $1$ elements. In the following we will make use of the identifications provided by the metric $\eta$ and pairing $\lara$:
\begin{alignat*}{2}
\eta:&& T^*_{(1,0)_\pm} &\leftrightarrow T^{(0,1)_\pm}\\
\lara:&& \ell^*_\pm &\leftrightarrow \ellt_\pm,
\end{alignat*}
so that elements in $T^*_{(1,0)_\pm}$ can be written as $\eta(\tl{z}_\pm)$ with $\tl{z}_\pm$ vector in $T^{(0,1)_\pm}$ and similarly, elements in $\ell^*_\pm$ can be expressed as $\la \tl{w}_\pm, \cdot\ra$ with $\tl{w}_\pm$ section of $\ellt_\pm$. To declutter notation, we shall also denote the bracket and anchor on $\Lb_+$ by $\brac$ and $\pi$, respectively.

The differential $\rd_{\Lb_+}$ is defined by the bracket on the Lie algebroid $\Lb_+$ via the following formula:
\begin{align}\label{eq:proof_Liethm}
\rd_{\Lb_+}  w^* (u,v)=\pi(u) w^*(v)-\pi(u) w^*(v)- w^*([u,v]),
\end{align}
where $ w^* \in \se(\Lb_+^*)$ and $u,v\in \se(\Lb_+)$. Writing $ w^*$ as $\la \tl{w},\cdot\ra$, with $\tl{w}\in\se (\tl{\Lb}_+)$ we can rewrite this as
\begin{align}\label{dL}
\begin{aligned}
\rd_{\Lb_+}  w^* (u,v)&=\pi(u)\la v ,\tl{w}\ra-\pi(v)\la u,\tl{w}\ra-\la [u,v],\tl{w}\ra\\
&=\la D_uv,\tl{w}\ra+\la v,D_u \tl{w}\ra-\la D_v u,\tl{w}\ra-\la u,D_v\tl{w}\ra-\la D_uv-D_vu,\tl{w}\ra\\
&+{T}(u,v,\tl{w})\\
&=\la D_u \tl{w},v\ra-\la D_v\tl{w},u\ra+{T}(u,v,\tl{w})\\
&=\la D_u \tl{w}_+,v_+\ra-\la D_v\tl{w}_+,u_+\ra+{T}(u_+,v_+,\tl{w}_+)\\
&+\la D_u \tl{w}_-,v_-\ra-\la D_v\tl{w}_-,u_-\ra+{T}(u_-,v_-,\tl{w}_-),
\end{aligned}
\end{align}
where we used \eqref{gentorsion_def} to express $\brac$ in terms of $D$, the generalized Bismut connection of the generalized metric for the GpK structure. We then used the fact that $D$ preserves $\lara$, $C_\pm$ as well as $\Lb_+$ and that $T$ only has components in $\Lambda^3 C_+\oplus\Lambda^3 C_-$ \cite{Gualtieri:2010fd}.

We will now show that $\rd_{\Lb_+}$ is related to $\rd_\QQ$ via \eqref{eq:d_pi}, so that $\rd_\QQ=(\pi^{-1})^*\circ \rd_{\Lb_+}\circ \pi^*$, i.e.
\begin{align*}
\rd_\QQ\ap(e_1,e_2)=\rd_{\Lb_+}(\pi^*\ap)(\pi^{-1}e_1,\pi^{-1}e_2).
\end{align*}
To read this off from \eqref{dL} and compare with \eqref{eq:dQ}, we need to express $\pi^*\ap$ as $\pi^*\ap=w^*=\la \tl{w},\cdot\ra$ for some $\tl{w}\in\se (\tl{\Lb}_+)$ and $\ap=(\eta(\zt_+),\eta(\zt_-))$:
\begin{align*}
\pi^*\ap=\pi^*(\eta(\zt_+),\eta(\zt_-))=(\eta(\zt_+,\pi_+\cdot),\eta(\zt_-,\pi_-\cdot))=\frac{1}{2}\la\pi^{-1}(\zt_+,-\zt_-),\cdot\ra,
\end{align*}
where $\zt_\pm\in \se(T^{(1,0)_\pm})$ and we used \eqref{pi:gG_relationship}. Therefore, denoting $e_1=(x_+,x_-)$ and $e_2=(y_+,y_-)$ as in \eqref{eq:dQ}, \eqref{dL} yields (denoting $e_1=(x_+,x_-)$ and $e_2=(y_+,y_-)$ as in \eqref{eq:dQ})
\begin{align*}
\rd_{\Lb_+}(&\pi^*\ap)(\pi^{-1}e_1,\pi^{-1}e_2)\\
&=\eta(\n^+_{x_++x_-}\zt_+,y_+)-\eta(\n^+_{y_++y_-}\zt_+,x_+)+\frac{1}{2}T(\pi_+^{-1}x_+,\pi_+^{-1}y_+,\pi_+^{-1}\zt_+)\\
&+\eta(\n^-_{x_++x_-}\zt_-,y_-)-\eta(\n^-_{y_++y_-}\zt_-,x_-)-\frac{1}{2}T(\pi_-^{-1}x_-,\pi_-^{-1}y_-,\pi_-^{-1}\zt_-),
\end{align*}
where we used \eqref{genBismut_pi_nabla_pm} and once again \eqref{pi:gG_relationship}. Finally, invoking \eqref{gentorsion_H}, we arrive at \eqref{eq:dQ}, showing that \eqref{eq:d_pi} holds and completing the proof.
\end{proof}

\subsection{Topological twists of the $(2,2)$-model}

\subsection{The (para)holomorphic twist of the $(2,0)$-model}

\bibliographystyle{JHEP}
\bibliography{mybib}
\end{document}