\documentclass[letterpaper,12pt]{article}
\usepackage{amssymb, amsmath, amsthm, mathtools, bbm, tikz-cd,stmaryrd,enumerate,hyperref,bm}
%-------------------
\usepackage{fancyhdr}
\pagestyle{fancy}
\fancyhf{}
\fancyhead[L]{\leftmark}
\fancyhead[R]{\thepage}
\addtolength{\headwidth}{\marginparsep}
\addtolength{\headwidth}{\marginparwidth}
\addtolength{\headwidth}{0.25in}
%----------------
%\setlength{\marginparwidth}{0pt} % width of margin notes
%\setlength{\marginparsep}{0pt} % width of space between body text and margin notes
\setlength{\evensidemargin}{0.125in} % Adds 1/8 in. to binding side of all 
\setlength{\oddsidemargin}{0.125in} % Adds 1/8 in. to the left of all pages
\setlength{\textwidth}{6.375in} % assuming US letter paper (8.5 in. x 11 in.) and 
\raggedbottom
\setlength{\parskip}{\medskipamount}
\renewcommand{\baselinestretch}{1}
%---------------------------------
\newcommand{\TT}{{T\oplus T^*}}
\newcommand{\JJ}{\mathcal{J}}
\newcommand{\KK}{\mathcal{K}}
\newcommand{\GG}{\mathcal{G}}
\newcommand{\Cc}{\mathbf{C}}
\newcommand{\RR}{\mathbb{R}}
\newcommand{\XX}{\mathfrak{X}}
\newcommand{\HH}{\mathcal{H}}
\newcommand{\FF}{\mathcal{F}}
\newcommand{\QQ}{\mathcal{Q}}
\newcommand{\cE}{\mathcal{E}}
\newcommand{\cN}{\mathcal{N}}
\newcommand{\sA}{\mathcal{A}}
\newcommand{\cO}{\mathcal{O}}
\def\d{{\rm d}}
\def\dbar{\Bar{\partial}}
%----------------------------------------
\newcommand{\PP}{\mathrm{P}}
\newcommand{\PPt}{\tilde{\mathrm{P}}}
\newcommand{\id}{\mathbbm{1}}
\newcommand{\nlr}{\overset{\leftrightarrow}{\n}}
\newcommand{\lc}{\mathring{\n}}
\newcommand{\im}{\mathrm{Im}\,}
\newcommand{\Ker}{\mathrm{Ker}\,}
\newcommand{\Lie}{\mathcal{L}}
\newcommand{\PS}{\mathcal{P}}
\newcommand{\ap}{\alpha}
\newcommand{\bt}{\beta}
\def\w{\wedge}
\newcommand{\p}{\partial}
\newcommand{\pt}{\tilde{\partial}}
\newcommand{\xt}{{\tilde{x}}}
\newcommand{\n}{\nabla}
\newcommand{\rd}{\mathrm{d}}
\newcommand{\PH}{(\PS,\eta,\omega)}
\newcommand{\Lt}{\tl{L}}
\newcommand{\Lb}{\mathbb{L}}
\newcommand{\s}{\mathbf{s}}
\newcommand{\se}{\Gamma}
\newcommand{\Endo}{\text{End}}
\newcommand{\ellt}{{\tl{\ell}}}
\newcommand{\ot}{{1/2}}
\newcommand{\inv}{{-1}}
\newcommand{\Aa}{\mathcal{A}}
\newcommand{\la}{\langle}
\newcommand{\ra}{\rangle}
\newcommand{\lara}{\la\ ,\ \ra}
\newcommand{\brac}{[\ ,\ ]}
\newcommand{\bl}{[\![}
\newcommand{\br}{]\!]}
\newcommand{\bracd}{\bl \ ,\ \br}
\newcommand{\yt}{\tl{y}}
\newcommand{\zt}{\tl{z}}
\newcommand{\tth}{\tl{\theta}}
\newcommand{\kk}{\mathrm{k}}
\newcommand{\Mt}{\tl{M}}
\newcommand{\pd}{\overline{\bm{\p}}}
\newcommand{\Mb}{\mathbb{M}}
\newcommand{\Drm}{\mathrm{D}}
\newcommand{\Dcal}{\mathcal{D}}
\newcommand{\wtl}{\widetilde}
%----------------------------------------
\def\gld{generalized Lie derivative }
\def\glds{generalized Lie derivatives }
\def\tl{\tilde}

\def\xto{\xrightarrow}

% These will be typeset in italics
\newtheorem{theorem}{Theorem}[section]
\newtheorem{proposition}[theorem]{Proposition}
\newtheorem{lemma}[theorem]{Lemma}
\newtheorem{corollary}[theorem]{Corollary}
\newtheorem{fact}[theorem]{Fact}
\newtheorem*{theorem*}{Theorem}
\newtheorem*{lemma*}{Lemma}
\newtheorem*{proposition*}{Proposition}
\newtheorem{Rem}[theorem]{Remark}

% These will be typeset in Roman
\theoremstyle{definition}
\newtheorem{Def}[theorem]{Definition}
\newtheorem{Conj}[theorem]{Conjecture}
\newtheorem{remark}[theorem]{Remark}

\newtheorem*{notation*}{Notation}

\theoremstyle{remark}
\newtheorem{Ex}[theorem]{Example}
\newtheorem{question}[theorem]{Question}
\newenvironment{claim}[1]{\par\noindent\underline{Claim:}\space#1}{}
\newenvironment{claimproof}[1]{\par\noindent\underline{Proof:}\space#1}{\hfill $[acksquare$}


\input xy

\xyoption{all}

\DeclareMathOperator{\End}{End}
\DeclareMathOperator{\rk}{rk}

\def\brian{\textcolor{blue}{BM: }\textcolor{blue}}
\def\david{\textcolor{red}{DB: }\textcolor{red}}
\def\btd{\textcolor{orange}{BM to DB: }\textcolor{orange}}

\title{Topological Sigma Models for Generalized Para-Complex Geometry}

\begin{document}

\maketitle
\begin{abstract}
We study a class of $2d$ topological sigma models that are formulated on generalized para-complex target manifolds via the AKSZ framework. When the target is additionally generalized para-K\"ahler, these topological theories can be realized as topological twists of $(2,2)$ para-supersymmetric sigma models of Abou-Zeid and Hull. We compare our results to the case of generalized complex and generalized K\"ahler geometry, where similar results have been described in the context of the usual $(2,2)$ SUSY by Kapustin and Li. We present several examples and show that our construction recovers the A-model and the Poisson sigma model as special cases. Moreover, we also introduce new examples of topological theories inherently tied to the para-complex geometry, the para-B-model and a para-complex version of the Landau-Ginzburg model.


\end{abstract}
\newpage
\tableofcontents
\newpage
\section{Introduction}

In \cite{HullTwistedSUSY}, Abou-Zeid and Hull studied the target geometry of certain $2D$ $(2,2)$ supersymmetric sigma models, which compared to the usual $(2,2)$ supersymmetry differ by a choice of sign in the supercommutator relations of its $4$ real supercharges:
\begin{align*}
\{Q_\pm,\tl{Q}_\pm\}=-2\p_\pm,
\end{align*}
with all other anti-brackets vanishing. The geometry of the target space of these sigma models reflects this sign change and in contrast to complex geometry needed for the usual supersymmetry, the target needs to be para-complex. More precisely, the required target geometry must be bi-para-Hermitian. In \cite{HullTwistedSUSY}, this type of supersymmetry was dubbed {twisted} SUSY, while the name {pseudo}-SUSY was used in various other works. Here we use the name {\bf para-SUSY} to reflect its relationship to para-complex geometry and to reserve the adjective ``twisted'' for the context of topological twists.

Para-complex structure on a manifold $\Mb$ is given by the data of an integrable endomorphism $K \in \End(T\Mb)$ such that $K^2=\id$ and its two eigenbundles corresponding to the $+1$ and $-1$ eigenvalues have the same rank. Consequently, $\Mb$ needs to have an even dimension $2n$. A para-Hermitian structure on $(\Mb,\eta)$ is then obtained by adding a compatible metric $\eta$:
\begin{align*}
\eta(K\cdot,K\cdot)=-\eta(\cdot,\cdot).
\end{align*}
This condition forces $\eta$ to be of signature $(n,n)$ and implies that $\omega=\eta K$ is an almost symplectic structure called the fundamental form of $K$. A bi-para-Hermitian geometry is then defined as a pair of para-Hermitian structures $(\eta,K_\pm)$ sharing the same metric and satisfying an additional condition
\begin{align}
\rd^p_\pm\omega_\pm=\pm H \in \Omega^3_{cl.}(\Mb),
\end{align}
where $\rd^p_\pm=K^*\circ \rd\circ K_\pm^*$ is the para-complex twisted differential and $\omega_\pm=\eta K_\pm$ are the fundamental forms of $K_\pm$.

In \cite{Hu:2019zro}, it is shown that the bi-para-Hermitian geometry $(\Mb,\eta,K_\pm,H)$ is equivalent to the data of a generalized para-K\"ahler geometry, which is defined as a pair of {\it generalized} para-complex (GpC) structures $\KK_\pm$ \cite{wade2004dirac,Zabzine:2006uz,Hu:2019zro} on $\Mb$,
\begin{align*}
\KK_\pm \in \End((\TT)\Mb), \quad \KK^2=\id, \quad \la \KK\cdot,\KK\cdot\ra=-\la\cdot,\cdot \ra,
\end{align*}
that commute and whose product $\GG=\KK_+\KK_-$ defines a split signature metric $G$ on $(\TT)\Mb$ via $G(\cdot,\cdot)=\la \GG\cdot,\cdot\ra$. Here, $\la\cdot,\cdot \ra$ is the symmetric duality pairing on $(\TT)M$ and the generalized para-complex endomorphisms $\KK_\pm$ satisfy a Courant integrability condition which ensures that its $\pm 1$-eigenbundles are Dirac structures \cite{courant1990dirac}.


Para-complex and para-Hermitian geometry was also recently studied as the description of the extended geometry in the context of T-duality \cite{freidel2017generalised,Freidel:2018tkj,Svoboda:2018rci,Marotta:2018myj}, following the earlier work of Vaisman \cite{vaisman2012geometry,vaisman2013towards}. In \cite{Marotta:2019eqc}, Szabo and Marotta described the bosonic sector of the $2D$ sigma models targeted in this extended geometry. Various topological models in the AKSZ framework on the para-Hermitian manifold $\Mb=T^*M\simeq M^n\times \RR^n$ were studied in \cite{Kokenyesi:2018xgj} and $3D$ membrane sigma models with this target were the subject of \cite{Chatzistavrakidis:2018ztm}. The generalized geometry corresponding to Para-Hermitian geometry and its refined version, Born geometry, has been studied in \cite{Hu:2019zro}.


\paragraph{Main results} In this paper, we combine the results of \cite{HullTwistedSUSY} and \cite{Hu:2019zro} to study the $2D$ topological sigma models with a generalized para-complex target manifold $(\Mb,\KK_+)$. This geometry leads to a $2D$ topological theory via the AKSZ formalism as follows. The Courant algebroid $(\TT)\Mb$ defines a $3D$ topological theory with a $2$-shifted symplectic NQ target space $({\cal E},\Omega)$ \cite{Roytenberg:2002nu}. Because the $\pm 1$-eigenbundles of $\KK$ are Dirac structures in $(\TT)\Mb$, they each define a Lagrangian in $({\cal E},\Omega)$, which are $1$-shifted symplectic NQ manifolds and as such define a $2D$ topological boundary theories.

Additionally, we show that when the GpC manifold $(\Mb,\KK_+)$ additionally admits a generalized para-K\"ahler structure $(\Mb,\KK_+,\KK_-)$, the topological theories for $(\Mb,\KK_+)$ described above can be realized as topological twists of the $(2,2)$ para-SUSY theory attached to $(\Mb,\KK_\pm)$.

We also present several explicit examples of our construction. The well-known examples include the Poisson sigma model of a real Poisson manifold $(\Mb,\Pi)$ that is described by the GpC structure
\begin{align*}
\KK_\Pi=
\begin{pmatrix}
\id & 2\Pi \\
0 & -\id
\end{pmatrix},
\end{align*}
the A-model of a symplectic manifold $(\Mb,\omega)$ described by the GpC structure
\begin{align*}
\KK_\omega=
\begin{pmatrix}
0 & \omega^{-1} \\
\omega & 0
\end{pmatrix},
\end{align*}
and also the BFF theory (?) that is simply given by the trivial GpC structure
\begin{align*}
\KK_0=
\begin{pmatrix}
\id & 0 \\
0 & -\id
\end{pmatrix}.
\end{align*}

The new examples that we describe are all tied to para-complex geometry. They are the para-complex version of the B-model of a para-complex manifold $(\Mb,K)$ that is given by the GpC structure
\begin{align*}
\KK_K=
\begin{pmatrix}
K & 0 \\
0 & -K^*
\end{pmatrix},
\end{align*}
and the para-complex topological Landau-Ginzburg model that can be seen as a deformation of the above geometry by a para-holomorphic poisson bi-vector $Q$
\begin{align*}
\KK_Q=
\begin{pmatrix}
K & Q \\
0 & -K^*
\end{pmatrix}.
\end{align*}

Importantly, our construction can also be applied to the more notoriously known generalized complex (GC) geometry, which was in the context of topological sigma models extensively studied in \cite{Zucchini:2004ta,Ikeda:2004cm,Kapustin:2004gv,Pestun:2006rj,Cattaneo:2009zx} to name some of the works, but to the authors' knowledge a similar construction to ours via a boundary reduction was not proposed before. Similarly in the GpC case, the GC structure defines a pair of Dirac structures (that are now tied by complex conjugation) and the $2D$ topological theory is recovered again as a boundary theory of the $3D$ theory defined by the underlying Courant algebroid. The advantage of our approach is that it provides a unified way to extract the data of a topological theory from any GC structure via a clear geometric procedure on the underlying shifted symplectic spaces. In the special case when the manifold is generalized K\"ahler \cite{Gualtieri:2003dx,Gualtieri:2010fd} -- implying it admits a $(2,2)$ SUSY --  the topological theories can again be realized as topological twists \cite{Kapustin:2004gv} of the $(2,2)$ theory. Here, the main examples are given by the A- and B- model corresponding to the GC structures $\JJ_\omega$ and $\JJ_I$, respectively:
\begin{align*}
\JJ_\omega=
\begin{pmatrix}
0 & -\omega^{-1} \\
\omega & 0
\end{pmatrix}, \quad
\JJ_I=
\begin{pmatrix}
I & 0 \\
0 & -I^*
\end{pmatrix},
\end{align*}
as well as the topological Landau-Ginzburg model which is again viewed via the deformation of $\JJ_I$ by a holomorphic poisson bi-vector \cite{Gualtieri:2007bq} $\sigma=-\frac{1}{4}(IQ+iQ)$:
\begin{align*}
\JJ_Q=
\begin{pmatrix}
I & Q \\
0 & -I^*
\end{pmatrix}.
\end{align*}

Notice, however, that for example the usual Poisson sigma model cannot be obtained in the realm of GC geometry because no GC structure is defined by a real Poisson structure $\Pi$.

%Since the work of (..) \cite{?} it has been known that on any K\"ahler manifold $(M,I,\omega)$, one can define a $2d$ sigma model admitting a $(2,2)$ supersymmetry (SUSY) and such model has a simple local description in terms of the K\"ahler potential.  Moreover, Witten has shown \cite{?} that such sigma model gives rise to two distinct topological theories -- the A- and B-models -- via a procedure called topological twisting. These topological theories have an important interpretation in terms of their underlying geometry: the A-model only depends on the symplectic structure $(M,\omega)$, while the B-model only depends on the complex structure $(M,I)$. Additionally, there exists an ${\mathbb Z}_2$ automorphism acting on the original $(2,2)$ supersymmetric theory that exchanges its A- and B-models. This is the simplest way one can observe the phenomenon of mirror symmetry in the context of $2d$ sigma models.
%
%The K\"ahler geometry does not describe the most general $(2,2)$ sigma model, as was shown in \cite{Gates-hull-rocek-biherm}. Instead, the most general geometry is described by a bi-Hermitian geometry, defined by a pair of complex structures $I_\pm$ compatible with the same Riemannian metric $g$ and satisfying the compatibility relation
%\begin{align*}
%\rd^c_\pm\omega_\pm=\pm H \in \Omega^3_{cl.}(M),
%\end{align*}
%$\omega_\pm$ being the fundamental forms of $I_\pm$ and $\rd^c_\pm$ the associated $\rd^c$-operators. For this geometry, the local description is more complicated than in the K\"ahler case, in particular because the analogue of the K\"ahler potential is much harder to describe. Consequently, the topological twists are also  more complicated.
%
%This problem is nevertheless partially mitigated by an alternative geometric formulation in terms of generalized K\"ahler (GK) geometry, which has been shown in \cite{Gualtieri:2003dx} to be equivalent to the above-described bi-Hermitian geometry $(g,I_\pm,H)$. A GK structure on $M$ is given by the data of two complex endomorphisms $\JJ_\pm$ of the bundle $(\TT)M$:
%\begin{align*}
%\JJ_\pm \in \End((\TT)M),\quad \JJ_\pm^2=-\id, \quad \la \JJ \cdot,\JJ\cdot \ra= \lara,
%\end{align*}
%where $\lara$ denotes the natural duality pairing on $(\TT)M$:
%\begin{align*}
%\la X+\ap,Y+\bt\ra=\ap(Y)+\bt(X),\quad X,Y\in \se(TM),\quad \ap,\bt \in \se(T^*M).
%\end{align*}
%Additionally, $\JJ_\pm$ are required to commute and their product $\GG\coloneqq -\JJ_+\JJ_-$ to define a 

\subsection{Notations and conventions}

\brian{Conventions for odd Lie brackets $[-,-]$ and $\{-,-\}$.}

\section{Background on paracomplex and generalized geometry}
\david{would just put the introduction of paracomplex geometry (minus the generalized part) in appendix, we did that even in the other paper which is literally a differential geometry paper}

\subsection{Paracomplex geometry} \label{sec: paracomplex}

In this section we introduce para-complex geometry with emphasis on the analogy with complex geometry. More details can be found in \cite{Hu:2019zro} or \cite[Ch.~15]{Cortes:2010ykx}, for example. \brian{I think you should cite your work here too.} 

\begin{Def}\label{def:paracpx}
An (almost) {\bf product structure} on a smooth manifold $\PS$ is an endomorphism $K\in \Endo(T\PS)$ which squares to the identity, $K^2=\id_{T\PS}$, $K\neq \id$. An (almost) \textbf{para-complex structure} is a product structure such that the $+1$ and $-1$ eigenbundles of $K$ have the same rank.
\end{Def}
A direct consequence of above definition is that any para-complex manifold is of even dimension. From now on, the $+1$ and $-1$ eigenbundles of an almost product/para-complex structure will be denoted $L$ and $\Lt$, respectively.

\paragraph{Integrability} The use of the word {\it almost} as usual refers to integrability of the endomorphism, i.e. whether its eigenbundles are involutive with respect to the Lie bracket and therefore define a foliation of the underlying manifold. Similarly to the complex case, the integrability is governed by the \textbf{Nijenhuis tensor}
\begin{align}\label{eq:nijenhuis}
\begin{aligned}
N_K(X,Y)&=[X,Y]+[KX,KY]-K([KX,Y]+[X,KY])\\
&=(\n_{KX}K)Y+(\n_XK)KY-(\n_{KY}K)X-(\n_YK)KX\\
&=4(\PP[\PPt X,\PPt Y]+\PPt[\PP X,\PP Y]),
\end{aligned}
\end{align}
where $\n$ is any torsionless connection and $\PP\coloneqq\frac{1}{2}(\id+K)$ and $\PPt\coloneqq\frac{1}{2}(\id-K)$ are the projections onto $L$ and $\Lt$, respectively. We say $K$ is integrable and call $K$ a product/para-complex manifold if $N_K=0$. From \eqref{eq:nijenhuis} it is apparent that $K$ is integrable if and only if {\it both} eigenbundles are simultaneously Frobenius integrable, i.e. involutive distributions in $T\PS$. This is one of the main differences between complex and para-complex geometry; one of the eigenbundles can be integrable while the other is not. In such cases, we call $\PS$ {\bf half-integrabile}. For more details, see for example \cite{??,??}.

\paragraph{Type decomposition} The splitting of the tangent bundle $T\PS=L\oplus \Lt$ gives rise to a decomposition of tensors analogous to the $(p,q)$-decomposition in complex geometry. For differential forms, we denote the decomposition as
\begin{align}\label{eq_plusminus_decomp}
\Lambda^k (T^*\mathcal{P})&=\bigoplus_{k=m+n}\Lambda^{(m,n)}(T^*\mathcal{P}),
\end{align}
where $\Lambda^{(m,n)}(T^*\mathcal{P})=\Lambda^m(T^{*}_{1,0})\otimes \Lambda^n(T^{*}_{0,1})$ and the corresponding sections are denoted by $\Omega^{(m,n)}(\mathcal{P})$. The bigradings \eqref{eq_plusminus_decomp} yield the natural projections
\begin{align*}
\Pi^{(p,q)}:\Lambda^k(T^*\mathcal{P})&\rightarrow \Lambda^{(p,q)}(T^*\mathcal{P}),
\end{align*}
so that when $K$ is integrable, the de-Rham differential splits as $\rd=\p+\tl{\p}$, where
\begin{align}\label{paracomplex_dels}
\begin{array}{cc}
\p \coloneqq \Pi^{(p+1,q)}\circ \rd, & \tl{\p} \coloneqq \Pi^{(p,q+1)}\circ \rd,
\end{array}
\end{align}
are the \textbf{para-complex Dolbeault operators}, satisfying
\begin{align*}
\p^2=\tl{\p}^2=\p\tl{\p}+\tl{\p}\p=0.
\end{align*}

One can also introduce the {\it twisted differential $\rd^p\coloneqq(\Lambda^{k+1}K^*)\circ\rd\circ (\Lambda^kK^*)$}. When $K$ is integrable, it can be simply written as $\rd^p=(\p_++\p_-)$ on real forms and $\rd^p=\kk (\p+\bar{\p})$ on para-complex forms.





\paragraph{Para-Holomorphic structure}

Let now $(\mathcal{P},K)$ be an almost para-complex manifold of dimension $2n$. If $K$ is integrable, we get a set of $2n$ coordinates $(x^i,\tilde{x}_i)$ called {\bf adapted coordinates}, $\mathcal{P}$ locally splits as $M\times \tilde{M}$, and $K$ acts as identity on $TM=L$ and negative identity on $T\tilde{M}=\tilde{L}$. The adapted coordinates therefore define two complimentary foliations that we will call {\bf canonical foliations} of the para-complex manifold $\PS$ and denote $M$ and $\Mt$. Both $M$ and $\Mt$ are therefore $n$-dimensional manifolds composed of connected components called leafs $M_i$, $\Mt_i$:
\begin{align*}
M=\bigcup_i M_i,\quad \Mt=\bigcup_i \Mt_i,
\end{align*}
and each are as sets equal to $\PS$.

As usual, a map of para-complex manifolds is called para-holomorphic if its pushforward commutes with the respective para-complex structures
\begin{Def}
Let $(M,K_M)$ and $(N,K_N)$ be para-complex manifolds. A map $F:M\rightarrow N$ is called para-holomorphic if
\begin{align}\label{eq:def_parahol}
K_N\circ F_*=F_*\circ K_M
\end{align}
\end{Def}
\begin{remark}
In the following we will omit the prefix ``para-'' in para-holomorphic whenever no confusion with complex holomorphicity is possible.
\end{remark}

Locally, the map $F:M\rightarrow N$ of para-complex manifolds can be understood via composition of coordinates as a
\begin{align*}
F&: \RR^{2n}\rightarrow \RR^{2m}\\
F=(f^i,\tl{f}_j)&=(y^i(x^k,\xt_l),\yt_j(x^k,\xt_l))^{i,j=1,\cdots,m}_{k,l=1,\cdots, n}, 
\end{align*}
where $(x^k,\xt_l)$ and $(y^i,\yt_j)$ are adapted local coordinates on $M$ and $N$, respectively. It is easy to check from \eqref{eq:def_parahol} that $F$ is a holomorphic map iff the components satisfy
\begin{align}\label{eq:holomorphic_real}
\frac{\p}{\p \xt_i}f^j=\frac{\p}{\p x^i}\tl{f}_j=0.
\end{align}

The conditions \eqref{eq:holomorphic_real} tell us that the holomorphic functions map the canonical foliations of the para-complex manifolds $M$ and $N$. This also means that the transition functions on a para-complex manifold (seen as maps $\RR^{2n}\rightarrow \RR^{2m}$) are holomorphic since the foliations must be preserved, i.e. the coordinates transform as
\begin{align}\label{para-coordinate-transf}
(x^i,\xt_i)\mapsto (y^j(x^i),\yt_j(\xt_i)).
\end{align}

This also gives us an intuition into what holomorphic bundles and their holomorphic sections look like. First, we notice that the tangent bundle $T\PS$ itself is a holomorphic bundle (see for example \cite{Cortes:2003zd,Hu:2019zro}); this is simply because of the form of the transition functions \eqref{para-coordinate-transf} that induce a holomorphic structure on the tangent bundle. The holomorphic sections of $T\PS$ -- the holomorphic vector fields -- are locally of the form
\begin{align*}
X= X^i(x)\p_i+\tl{X}_i(\xt)\pt^i,
\end{align*}
i.e. the components in $L$ and $\Lt$ individually define vector fields on the foliations $M$ and $\Mt$, respectively.  

Similarly, the cotangent bundle is a holomorphic bundle. For the higher wedge powers, we find that only $\Lambda^{(k,0)+(0,k)}(T^*\PS)$, $1\leq k\leq n$, are holomorphic bundles. The {\bf para-complex Dolbeault complex} is therefore given by
\begin{align}\label{para-dolbeault}
\left(\Omega^{(\bullet,0)}\otimes\Omega^{(0,\bullet)},(\p,\pt)\right)\coloneqq \left(\mathbf{\Omega}^{0,\bullet},\pd\right).
\end{align}

\subsubsection{Para-Complexified Version}
In order to make the analogy between complex and para-complex geometry more explicit, one can \textit{paracomplexify} all objects by tensoring with the algebra $\mathbf{C}$ of \textbf{para-complex numbers}, which are given by ordered pairs $(x,y)\in \mathbb{R}\otimes (1,k)$ generated by $1$ and the symbol $k$, satisfying $k^2=1$ and called the para-complex unit. The elements $(x,y)\in \mathbf{C}$ can be therefore written as
\begin{align*}
\mathbf{C} \ni z=(x,y)=x+k\cdot y.
\end{align*}
On $\mathbf{C}$, we define the following familiar operations:\begin{align}
\begin{aligned}
&\bullet\quad \text{Para-complex conjugation: }x+\kk y\mapsto \overline{x+\kk y}=x-\kk y\\
&\bullet\quad \text{Real part: } x+\kk y \mapsto \text{Re}(x+\kk y)=x\\
&\bullet\quad \text{Imaginary part: } x+\kk y \mapsto \text{Im}(x+\kk y)=y.
\end{aligned}
\end{align}


Therefore, we see that $\mathbf{C}^n$ is a paracomplex vector space with the paracomplex structure given by the multiplication by $k$. Conversely, any $2n$-dimensional paracomplex vector space $(V,K)$ can be identified with $(\mathbf{C}^n,k)$; the $\mathbf{C}$-module structure is defined as $(a+kb)v=av+bK(v)$ and if $(x^i, \xt^i)$ is a basis of $V$ spanning the $\pm 1$ eigenspaces of $K$, then $z^i\coloneqq \frac{1+k}{2}x^i+\frac{1-k}{2}\xt^i$ is a basis of $\Cc^n$. In other words, we have the following identification of paracomplex vector spaces
\begin{align}\label{para-complexification}
\begin{aligned}
\begin{array}{ccc}
(\RR^n\times\RR^n,\id \times -\id) &\longleftrightarrow &(\Cc^n, k)\\
(x^i,\xt^i) &\mapsto &\frac{1+k}{2}x^i+\frac{1-k}{2}\xt^i\\
(u^i+v^i,u^i-v^i) &\mapsfrom &u^i+kv^i,
\end{array}
\end{aligned}
\end{align}
where $i=1,\cdots,n$. If $(V,K)$ is a paracomplex vector space, then its paracomplexification, $V\otimes_\mathbb{R} \mathbf{C}\coloneqq V^\Cc$, splits to $\pm k$ eigenspaces of $K$, which is extended to $V^\Cc$ by linearity: $V^\Cc=V^{1,0}\oplus V^{0,1}$. The conjugation map is then an isomorphism between the two eigenspaces, i.e. $\overline{{V^{1,0}}}=V^{0,1}$. The splitting is given by the projections
\begin{align}
\PP^{1,0}=\frac{1}{2}(\id +kK),\quad \PP^{0,1}=\frac{1}{2}(\id -kK). \label{eq:projections_parac}
\end{align}


Let now $(\PS,K)$ be an almost paracomplex manifold. The para-complexified tangent bundle, $T\mathcal{P}^\mathbf{C}=T\mathcal{P}\otimes_\mathbb{R} \mathbf{C}$, splits to a $\pm k$ eigenbundles of $K$ as $T\mathcal{P}^\mathbf{C}=T\mathcal{P}^{1,0}\oplus T\mathcal{P}^{0,1}$. 
%and $T\mathcal{P}^{1,0}$ is isomorphic to $T\PS$ as real vector bundles:
%\begin{align*}
%\begin{aligned}
%\begin{array}{ccc}
%T\PS &\simeq &T\mathcal{P}^{1,0} \\
%X &\longmapsto &\frac{1}{2}(X+kKX) \\
%Re(X^{1,0})=\frac{1}{2}(X^{1,0}+\overline{X^{1,0}})&\longmapsfrom &X^{1,0}
%\end{array}
%\end{aligned}
%\end{align*}
Similarly to the real case, this induces a splitting of $\Lambda^k(T^*\mathcal{P}^\mathbf{C})$:
\begin{align*}
\Lambda^k(T^*\mathcal{P}^\mathbf{C})=\bigoplus_{k=p+q}\Lambda^{p,q}(T^*\mathcal{P}^\mathbf{C}),
\end{align*}
with the sections denoted as $\Omega^{p,q}_\mathbf{C}(\PS)$.
\begin{notation*}
To distinguish between the real and para-complexified bigradings, the former being with respect to the $\pm 1$ eigenvalues of $K$, while the latter is with respect to the $\pm k$ eigenvalues of $K$, we denote the real bigrading in round brackets, $(p,q)$, and the para-complexified grading without brackets, $p,q$.
\end{notation*}

The de-Rham differential also splits accordingly into the \textbf{para-complexified Dolbeault operators}
\begin{align*}
\p^\mathbf{C}&:\Omega^{p,q}_\mathbf{C}(\mathcal{P})\rightarrow \Omega_\mathbf{C}^{p+1,q}(\mathcal{P}),\\
\bar{\p}^\mathbf{C}&:\Omega_\mathbf{C}^{p,q}(\mathcal{P})\rightarrow \Omega_\mathbf{C}^{p,q+1}(\mathcal{P})\ .
\end{align*}

\newcommand{\Cp}{{\mathbf{C}}}
The isomorphism of para-complex vector spaces \eqref{para-complexification} induces an isomorphism between $\Lambda^{(p,q)}(T^*\PS)$ and $\Lambda^{p,q}(T^*\PS^\mathbf{C})$ as well as between the sections $\Omega^{(p,q)}$ and $\Omega_\Cp^{p,q}$, where the bigrading of the real bundle is with respect to the $\pm 1$ eigenvalues of $K$, while the bigrading of the para-complexified bundle is with respect to the $\pm k$ eigenvalues of $K$. Explicitly, we have
\begin{align}\label{isom_bigradings1}
\phi:\ (\ap,\tl{\ap})\ \in\ \Omega^{(p,q)}\otimes \Omega^{(q,p)}\longmapsto \frac{1+k}{2}\ap + \frac{1-k}{2}\tl{\ap}\ \in\ \Omega_\Cp^{p,q}
\end{align}
with the inverse
\begin{align}
\phi^{-1}:\ \bm{\ap}\ \in\ \Omega_\Cp^{p,q}\longmapsto (\text{Re}(\bm{\alpha})+\text{Im}(\bm{\ap}),\text{Re}(\bm{\alpha})-\text{Im}(\bm{\ap}))\ \in\ \Omega^{(p,q)}\otimes \Omega^{(q,p)}.
\end{align}
The point we would like to emphasize here is that to one para-complex differential form of degree $p,q$ corresponds a {\it pair} of real forms in degrees $(p,q)$ and $(q,p)$. In particular, a para-complex form in degree $k,0$ corresponds to a pair of forms in degrees $(k,0)$ and $(0,k)$, i.e. a two $k$-forms on the two canonical para-complex foliations. Consequently, recalling the para-complex Dolbeault complex \eqref{para-dolbeault}, we can easily deduce that the corresponding {\bf para-complexified Dolbeault complex} is given by
\begin{align}
\left( \Omega^{0,\bullet}_\Cp,\bar{\p}^\Cp\right),
\end{align}
In particular, the condition of holomorphicity in the para-complex coordinates $z^i\coloneqq \frac{1+k}{2}x^i+\frac{1-k}{2}\xt^i$ and $\bar{z}^i$ is equivalent to a local independence of the coordinates $\bar{z}^i$ and is therefore entirely analogous to complex geometry.
%\begin{lemma}
%Let $(\PS,K)$ be a paracomplex manifold. Then $\rd^p\coloneqq(\Lambda^{k+1}K)\circ\rd\circ (\Lambda^kK)$ can be expressed as
%\begin{align}\label{eq:dp-operator}
%\rd^p=\p_+-\p_-.
%\end{align}
%\end{lemma}
%\begin{proof}
%Let $\ap \in \Omega^{+m,-n}(\PS)$. Then we have
%\begin{align*}
%\rd^p\ap=(-1)^n(\Lambda^kK)\rd\ap=(-1)^{2n}\p_+ \ap +(-1)^{2n+1}\p_-\ap=(\p_+-\p_-)\ap,
%\end{align*}
%\end{proof}

%\paragraph{The para-holomorphic volume form}
%In K\"ahler geometry, the notion of a holomorphic volume form is very important and in particular gives rise to Calabi-Yau manifolds, perhaps the most important subclass of K\"ahler manifolds. Here we discuss the para-complex version of this story because it plays an important role when considering the anomalies of the R-symmetry at the quantum level.
%
%Remember that a holomorphic volume form on a complex manifold $M$ of complex dimension $k$ is a non-vanishing section of the canonical bundle, $\Omega \in \Omega^{(k,0)}(M)$, that is holomorphic, $\bar{\p}\Omega=0$. Recalling the para-complex Dolbeault complex \eqref{para-dolbeault}, we can immediately infer what should be the analogous object in para-complex geometry:
%\begin{Def}
%Let $(\PS,K)$ be a para-complex manifold of dimension $2n$. We define a {\bf para-holomorphic volume form} to be a nowhere vanishing section $\hat{\Omega}=(\Omega,\tl{\Omega})\in \Omega^{(n,0)+(0,n)}(\PS)$, where $\Omega \in \Omega^{(n,0)}(\PS)$ and $\tl{\Omega}\in \Omega^{(0,n)}(\PS)$ are each non-vanishing and satisfy
%\begin{align*}
%\pt \Omega=\p \tl{\Omega}=0.
%\end{align*}
%\end{Def}
%Because $\Omega^{(n,0)}(\PS)$ is naturally isomorphic to $\Omega^n(M)$, the section $\Omega$ is non-vanishing and $\pt \Omega=0$ implies it is invariant under the translation along $\Mt$, the above definition implies that $\Omega$ defines an honest volume form on $M$. Similarly, $\tl{\Omega}$ is a volume form on $\Mt$ and therefore both the canonical foliations $M$ and $\Mt$ of the para-complex manifold are orientable.
%
%As an extension of the above definition and the direct analogy with Calabi-Yau manifolds, we can introduce a definition of the para-Calabi-Yau manifolds:
%\begin{Def}
%A {\bf para-Calabi-Yau manifold} is a compact, para-K\"ahler manifold $(\PS,\eta,K)$ with a para-Holomorphic volume form $\hat{\Omega}$.
%\end{Def}

\subsubsection{Para-Hermitian Geometry}
We now introduce a metric $\eta$ compatible with the para-complex structure $K$, which by contraction $\eta \circ  K$ induces a non-degenerate two-form $\omega$. This gives rise to para-Hermitian geometry:
\begin{Def}
	Let $(\PS,K)$ be a para-complex manifold and let $\eta$ be a pseudo-Riemannian metric that satisfies $\eta(K\cdot,K\cdot)=-\eta$. Then we call $(\PS,K,\eta)$ a \textbf{para-Hermitian manifold}\footnote{If $K$ is not integrable, i.e. $(\PS,K)$ is almost para-complex, we would call $(\PS,K,\eta)$ an almost para-Hermitian manifold.}.
\end{Def}
\noindent
The above definition implies that the tensor $\omega\coloneqq \eta K$ is skew
\begin{align*}
\omega(X,Y)=\eta(KX,Y)=-\eta(X,KY)=-\omega(Y,X),
\end{align*}
and nondegenrate (because $\eta$ is nondegenerate), therefore $\omega$ is an almost symplectic form, sometimes called the \textbf{fundamental form}. From $K^2=\id$ we also have $K=\eta^{-1}\omega=\omega^{-1}\eta$. Another observation is that since the eigenbundles of $K$ have the same rank, $\eta$ has split signature $(d,d)$. Furthermore, the eigenbundles of $K$ are isotropic with respect to both $\eta$ and $\omega$. This means that the almost symplectic form $\omega$ is of the type $(1,1)$, $\omega \in \Omega^{(1,1)}(\PS)$.
\begin{remark}
	As shown above, the data $(\PS,K,\eta)$, $(\PS,\eta,\omega)$ and $(\PS,K,\omega)$ are on a para-Hermitian manifold equivalent and so we may use the different triples interchangeably to refer to a para-Hermitian manifold.
\end{remark}
Again, in a complete analogy with K\"ahler geometry, we get para-K\"ahler geometry whenever the fundamental form $\omega$ is closed
\begin{Def}
	Let $(\PS,\eta,\omega)$ be a para-Hermitian manifold with $\rd\omega=0$. We call $(\PS,\eta,\omega)$ a \textbf{para-K\"ahler manifold}.
\end{Def}

\paragraph{Para-Calabi-Yau geometry}

In \cite{mythesis}, the notion of para-Calabi-Yau geometry is introduced. As the name suggests, these manifolds should play the role of the para-complex version of Calabi-Yau manifolds in complex geometry. We now recall some basic properties of para-Calabi-Yau manifolds, motivating the definition by Calabi-Yau geometry, which is a standard notion in the literature.

There are many different ways to define what a Calabi-Yau manifold is, but for the purpose of our discussion we will call a Calabi-Yau manifold a K\"ahler manifold $(M,g,I)$ with the holonomy group of the underlying Riemannian metric to be $\text{Hol}(g)\subseteq SU(d)$, as opposed to $U(d)$, which is the case for ordinary K\"ahler manifolds.\footnote{Typically, a part of the definition of a Calabi-Yau manifold is the requirement of compactness of $M$ and sometimes, one requires that $\text{Hol}(g)$ is exactly equal to $SU(d)$.} A consequence of this property is that there exists a {\it holomorphic volume form}, i.e. a non-vanishing section $\Omega \in \Omega^{d,0}$ that is holomorphic, $\bar{\p}\Omega=0$ and satisfies
\begin{align*}
    \frac{\omega^d}{d!}=(-1)^\frac{d(d-1)}{2}\left(\frac{i}{2}\right)^d \Omega\w \bar{\Omega},
\end{align*}
where $\omega=gI$ is the K\"ahler form.

% a Calabi-Yau manifolds

% A Calabi-Yau manifold is a K\"ahler manifold $(M,\omega,I)$ of complex dimension $k$ with a non-vanishing holomorphic volume form $\Omega$,
% \begin{align*}
%     \Omega \in \Omega^{(k,0)}(M),\quad \bar{\p}\Omega=0,
% \end{align*}
% which satisfies a compatibility condition with the K\"ahler fundamental form $\omega$ (sometimes called the complex Monge-Ampere equation)
% \begin{align*}
%     \frac{\omega^k}{k!}=(-1)^\frac{k(k-1)}{2}\left(\frac{i}{2}\right)^k \Omega\w \bar{\Omega}.
% \end{align*}
% Calabi-Yau manifolds play an important role in theoretical physics, especially in string theory where they serve as the geometric model for string compactifications. For more details on Calabi-Yau manifolds, the reader may consult for example the book \cite{joyce2000compact}.

We now turn to the analogue of the above notions in the para-complex setting. Let $(K,\eta,\omega)$ be a $2d$-dimensional para-K\"ahler manifold with $\lc$ the Levi-Civita conencion of $\eta$. From $\lc\omega=\lc\eta$, we have that the holonomy group of a para-K\"ahler manifold is $GL(d)=O(d,d)\cap Sp(2d)$. For a para-Calabi-Yau manifold, the natural definition is therefore the following:

\begin{Def}
A {\bf para-Calabi-Yau manifold} is a para-K\"ahler manifold $(\PS,K,\eta)$ of dimension $2d$ such that the holonomy group of $\eta$ is $\text{Hol}(\eta)\subseteq SL(d)$.
\end{Def}
\begin{remark}
Note the interesting property that both $SU(d)$ and $SL(d)$ are real forms of $SL_\mathbb{C}(d)$.
\end{remark}

Let us now investigate what is the structure on para-Calabi-Yau analogous to the holomorphic volume form. We choose the local adapted coordinates $(x^i,\xt_j)$ in which $\eta$ and $\omega$ have the canonical form
\begin{align*}
    \eta=dx^i\otimes d\xt_i+d\xt_i\otimes dx^i,\quad \omega=dx^i\w d\xt_i.
\end{align*}
Because the holonomy of $\eta$ is now contained in $SL(d)$, there is also a covariantly constant tensor $\hat{\Omega}$
\begin{align*}
    \hat{\Omega}=dx^1\w\cdots\w dx^d+d\xt_1\w\cdots\w d\xt_d=\Omega+\tl{\Omega},
\end{align*}
which satisfies $\pt\Omega=\p\tl{\Omega}=0$ as well as
\begin{align*}
   \frac{\omega^d}{d!}=(-1)^\frac{d(d-1)}{2} \Omega\w \tl{\Omega}. 
\end{align*}

We call $\hat{\Omega}=(\Omega,\tl{\Omega})$ a {\bf para-holomorphic volume form}. Remember that on a para-complex manifold $\PS$ of dimension $2n$, the appropriate analogue of the holomorphic forms are the para-holomorphic forms, given by pairs $(\alpha,\tl{\alpha})\in\Omega^{(k,0)+(0,k)}(\PS)$ (here the bigrading is with respect to the para-complex structure), satisfying $\pt\ap=\p\tl{\ap}=0$. Therefore, the {para-holomorphic volume form} is exactly the correct analogue of the holomorphic volume form in case of the Calabi-Yau manifolds. Furthermore, because $\Omega^{(n,0)}(\PS)$ is naturally isomorphic to $\Omega^n(M)$, the section $\Omega$ is non-vanishing and $\pt \Omega=0$ implies it is invariant under the translation along $\Mt$, and so it defines a volume form on $M$. Similarly, $\tl{\Omega}$ is a volume form on $\Mt$ and we see that the para-Calabi-Yau structure on $\PS$ augments the canonical foliations $M$ and $\Mt$ of the para-complex manifold with volume forms and both the foliation manifolds are consequently orientable.

\subsection{Basics of generalized geometry}
In this paper, the term {\it generalized geometry} is used for the study of geometric structures on the bundle $TM \oplus T^*M$ over some manifold $M$. We will typically abbreviate this bundle to $\TT$ whenever the base is understood or unimportant for the discussion.

\subsubsection{The exact Courant algebroid structure}
The bundle $\TT$ has a natural Courant algebroid structure given by the symmetric pairing
\begin{align*}
\langle X+\ap,Y+\bt\rangle=\ap(Y)+\bt(Y),
\end{align*}
the Dorfman bracket,
\begin{align}\label{eq:dorfman}
[ X+\ap,Y+\bt]=[X,Y]+\Lie_X\bt-\imath_Y\rd \ap, 
\end{align}
and the anchor $\pi:X+\ap\mapsto X$. In the above, $X+\ap$ denotes a section of $\TT$ with the splitting to tangent and cotangent parts given explicitly. The Dorfman bracket can be thought of as an extension of the Lie bracket from $T$ to $\TT$ and therefore we opt to use the same notation for both brackets; the expression $[X,Y]$ is always the Lie bracket of vector fields whether we think of $\brac$ as the Lie bracket or the Dorfman bracket and no confusion is therefore possible.

The Courant algebroid on $\TT$ is exact, meaning that the associated sequence
\begin{align}\label{eq:exact_seq}
0\longrightarrow T^* \overset{\pi^T}{\longrightarrow} \TT\overset{\pi}{\longrightarrow} T\longrightarrow 0,
\end{align}
is exact. Here, $\pi^T$ is the transpose of $\pi$ with respect to the pairing $\lara$,
%\begin{align*}
%\la \pi^T(\ap),Y+\bt\ra=\la \ap,\pi(Y+\bt)\ra=\la \ap,Y\ra
%\end{align*}
i.e. $\pi^T: \ap \mapsto \ap+0$. In fact, all possible Courant algebroid structures on $\TT$ are parametrized by a closed three-form $H\in \Omega^3_{cl}$, which enters the definition of the bracket \eqref{eq:dorfman}, changing it to a {\it twisted} Dorfman bracket
\begin{align*}
[ X+\ap,Y+\bt]_H=[X,Y]+\Lie_X\bt-\imath_Y\rd \ap+\imath_Y\imath_X H.
\end{align*}
Moreover, any isotropic splitting of \eqref{eq:exact_seq} $s:T\rightarrow \TT$ is given by a two-form $b$, such that $X\overset{s}{\mapsto}X+b(X)$. This is equivalent to an action of a $b$-field transformation on $\TT$\footnote{Here we are using the term $b$-field transformation more liberally as it is customary to use the term only in the cases when $\rd b=0$ so that $e^b$ is a symmetry of $\brac$.}
\begin{align*}
e^b=
\begin{pmatrix}
\id & 0 \\
b & \id
\end{pmatrix}
\in \End(\TT),
\end{align*}
which consequently changes the bracket as
\begin{align*}
[ e^b(X+\ap),e^b(Y+\bt)]_H=e^b([ X+\ap,Y+\bt]_{H+\rd b}),
\end{align*}
which implies that when $H$ is trivial in cohomology, then a choice of a $b$-field transformation such that $\rd b=-H$ brings the twisted bracket $\brac_H$ into the standard form \eqref{eq:dorfman}. When $H$ is cohomologically non-trivial this can be done at least locally. This also means that any choice of splitting with a non-trivial $b$-field can be absorbed in the Courant algebroid bracket in terms of the {\it flux}\footnote{Flux is a term used mainly in physics, in this context simply meaning the ``tensorial contribution to the bracket''.} $\rd b$.

\paragraph{Dirac Structures}
An important object in Dirac geometry are (almost) Dirac structures, which are subbundles $L\subset \TT$ with special properties.
\begin{Def}
An {\bf almost Dirac structure} $L$ is a maximally isotropic subbundle of $\TT$, i.e. $\la u,v\ra=0$ for any $u,v \in \se(L)$ and $\text{rank}(L)=\text{rank}(T)$. When $L$ is involutive under the Dorfman bracket, i.e. it satisfies $[L,L]\subset L$, we call $L$ simply a {\bf Dirac structure}.
\end{Def}
An important fact we will repeatedly use is the following:

\begin{proposition}[\cite{courant1990dirac}]\label{prop:dirac_Liealg}
Let $L$ be a Dirac structure in the Courant algebroid $\TT$ with a flux $H$. Then the restriction of the Dorfman bracket to $L$, $\brac\mid_L$ is skew and gives $L$ a structure of a Lie algebroid compatible with the anchor $\pi\mid_L$, a restriction of the projection $\pi:\TT\rightarrow T$ to $L$.
\end{proposition} 
 
 
We remark here that all the results in this paper remain valid for any exact courant algebroid $E$ (i.e. $E$ fits in the sequence \eqref{eq:exact_seq}), which can be always identified with $\TT$ by the choice of splitting equivalent to a choice of a representative $H\in\Omega^3_{cl}$. This also amounts to setting $b=0$ in all formulas since the $b$-field appears as a difference of two splittings.

\subsection{Generalized para-complex geometry}
We now recall the notion of a generalized para-complex \cite{wade2004dirac,Zabzine:2006uz,Hu:2019zro} and a generalized para-K\"ahler \cite{Hu:2019zro} structure. We follow the discussion presented in \cite{Hu:2019zro}.

\begin{Def}
A \textbf{(twisted) generalized para-complex} (GpC) structure $\KK$ is an endomorphism of $\TT$, such that $\KK^2=\id$ and $\la\KK,\KK\ra=-\lara$, whose generalized Nijenhuis tensor vanishes:
\begin{align}\label{eq:gen_nijenhuis}
\mathcal{N}_\KK(u,v)=[\KK u,\KK v]_H+\KK^2[ u,v]_H-\KK([\KK u,v]_H+[ u,\KK v]_H)=0.
\end{align}
\end{Def}

It is a straightforward exercise to check that the most general form of an almost GpC structure is given by
\begin{align}\label{eq:GpC_generalform}
 \KK =
 \begin{pmatrix}
 A & \Pi \\
 \Omega & -A^*
 \end{pmatrix}, \text{ with }
 \begin{cases}
  A^2 + \Pi \Omega & = \id\\
  A\Pi - \Pi A^* & = 0\\
  \Omega A + A^*\Omega & = 0
 \end{cases}.
\end{align}
where $A \in \End(T)$ and $\Omega \in \Omega^2(M)$, $\Pi \in \se(\Lambda^2 T)$ are skew tensors. The integrability conditions on these tensors are a bit more involved and we will only discuss their special cases when discussing examples. An interested reader can consult \cite{crainic-integrability}, where the integrability is presented in the most general form for generalized complex structures, where the conditions are analogous.


Let $L_\KK$ and $\Tilde{L}_{\KK}$ denote the $+1$ and $-1$ eigenbundles of $\KK$, respectively. 
It is easy to check that the condition \eqref{eq:gen_nijenhuis} is equivalent to the $\pm 1$ eigenbundles of $\KK$ to be involutive under the twisted Dorfman bracket. Since in this paper the flux $H$ will be generically non-zero but all results hold for the special case $H=0$ as well, we will drop the word ``twisted" from our definitions.

Generalized para-complex structures can be understood as the data one needs to specify a splitting of $\TT$ to a pair of integrable Dirac structures:

\begin{theorem*}[\cite{wade2004dirac}]\label{thm:pairofdirac}
There is a one-to-one correspondence between generalized para-complex structures on $M$ and pairs of transversal Dirac subbundles of $\TT$.
\end{theorem*}

Combining this result with the well-known result of \cite{Liu:1995lsa} which states that any pair of transversal Dirac structures $(L,\Lt)$ forms a Lie bialgebroid $(L,L^*\simeq \Lt)$, one can immediately infer the following

\begin{corollary}
Generalized para-complex structures on $\TT$ are in one-to-one correspondence with Lie bialgebroid pairs $(L,L^*)$ such that $L\oplus L^*=\TT$. 
\end{corollary}


\begin{Ex}[The trivial structure and its deformations]\label{ex:GpC_trivial}
Any manifold supports the following GpC structure
\begin{align*}
\KK_0=
\begin{pmatrix}
\id & 0 \\
0 & -\id
\end{pmatrix},
\end{align*}
that has eigenbundles $T$ and $T^*$ and is always integrable. The following two GpC structures can be seen as deformations of $\KK_0$ by either a two-form $b$ or a bi-vector $\beta$:
\begin{align*}
\KK_b=
\begin{pmatrix}
\id & 0 \\
2b & -\id
\end{pmatrix},\quad
\KK_\beta=
\begin{pmatrix}
\id & 2\beta \\
0 & -\id
\end{pmatrix}.
\end{align*} 
$\KK_b$ is integrable iff $\rd b =-H$, and its eigenbundles are $L_{b} = \text{graph}(b)=\{X+b(X)\mid X \in \XX\}$ and $\widetilde{L}_b =T^*$. Similarly, $\KK_\beta$ is integrable iff $\beta$ is a Poisson structure, (see \cite[Lemma~2.13]{Hu:2019zro}) and its eigenbundles are $L_\beta =T$ and $\widetilde{L}_\beta =\text{graph}(-\beta)=\{\ap-\beta(\ap)\}\mid \ap \in \Omega\}$.
The Lie algebroid $\Tilde{L}_{\beta}$ is equivalent to the Lie algebroid with underlying bundle $T^*$ and anchor given by the Poisson bivector $\beta : T^* \to T$. 
\end{Ex}

\begin{Ex}[Para-complex structures]\label{eg: pc}
An almost para-complex structure $K\in \Endo(T)$, defines the diagonal generalized almost para-complex structure:
\begin{align*}
\KK_K=
\begin{pmatrix}
K & 0 \\
0 & -K^*
\end{pmatrix}.
\end{align*}
The integrability of $\KK_K$ is equivalent to Frobenius integrability of $K$ , i.e. vanishing of the Nijenhuis tensor of $K$.
Thus a para-complex structure defines a generalized para-complex structure of the above form. 
The corresponding Dirac structures are given by the $+1$ eigenbundle $L_K =T^{(1,0)}\oplus T^{*(0,1)}$ and $-1$ eigenbundle $\widetilde{L}_K =T^{(0,1)}\oplus T^{*(1,0)}$, where the bigrading is with respect to $K$.
%
%The structure $\KK_P$ can be generalized to the following
%\begin{align*}
%\KK=
%\begin{pmatrix}
%P & \Pi \\
%\Omega & -P^*
%\end{pmatrix},
%\end{align*}
%where we can read off from the constraints in \eqref{eq:GpC_generalform} that $P^2=\id$ implies that $\Pi\Omega=0$ and both $\Pi$ and $\Omega$ have to be of type $(2,0)+(0,2)$ with respect to the grading given by $P$.
\end{Ex}

\begin{Ex}[Symplectic structures]\label{ex:GpC_sympl}
A pre-symplectic form $\omega$ defines the anti-diagonal almost generalized para-complex structure
\begin{align*}
\KK_\omega=
\begin{pmatrix}
0 & \omega^{-1} \\
\omega & 0
\end{pmatrix}.
\end{align*}
The integrability of $\KK_\omega$ is equivalent to $\rd\omega=0$.
Thus, if $\omega$ is symplectic then $\KK_\omega$ is a generalized para-complex structure.
The $\pm 1$ eigenbundles are given by $\text{graph}(\pm\omega)=\{X\pm\omega(X)\mid X\in \XX\}$.
\end{Ex}



\paragraph{Comparision with generalized complex structures}
Example \ref{ex:GpC_sympl} shows that a symplectic manifold $(M,\omega)$ is a GpC manifold and it is well-known \cite{Gualtieri:2003dx} that $(M,\omega)$ is also generalized complex (GC). However, while almost GC structures exist only on almost complex manifolds \cite{Gualtieri:2003dx}, Example \ref{ex:GpC_trivial} demonstrates that GpC structures exist on any Poisson manifold and in particular on any smooth manifold (with trivial Poisson structure). Another feature of GpC geometry that is not present in GC geometry is that the GpC structures can be half-integrable (similarly to ordinary para-complex structures): see the cases of $\KK_b$ and $\KK_\beta$ from Example \ref{ex:GpC_trivial} which are always at least half integrable and are fully integrable iff $b$ is closed and $\beta$ is Poisson, respectively. On the other hand, $\KK_\omega$ in Example \ref{ex:GpC_sympl} can only be half-integrable when the $H$-flux is non-zero.


\paragraph{Para-Complexified version}
One can also carry out all the above discussion in the setting of the para-complexified generalized tangent bundle $(\TT)\otimes_\RR \Cp$. In particular, one gets a one-to-one correspondence of GpC structures with {\it paracomplex} Dirac structures:
\begin{proposition}\label{prop: pcomplexified}
A generalized paracomplex structure is equivalent to a paracomplex Dirac structure $L \subset (\TT)\otimes_\mathbb{R} \mathbf{C}$ of real index zero, i.e. $L \cap \bar{L}=0$.
\end{proposition}
\begin{proof}
The proof is entirely analogous to []; if $\KK$ is a generalized paracomplex structure and $L_\KK$ is its $+k$ eigenbundle, $L_\KK$ is automatically maximally isotropic by 
\begin{align*}
\langle x,y \rangle = -\langle\KK x,\KK y \rangle = -\langle k x,k y \rangle = -\langle x,y\rangle=0.
\end{align*}
The para-complex conjugate $\bar{L}_\KK$ is then a $-k$ eigenbundle and $L\cap\bar{L}=0$. Conversely, given $L$ and $\bar{L}$, we can define $\KK$ to be the multiplication by $+k$ on $L$ and $-k$ on $\bar{L}$.
\end{proof}

\subsection{Generalized para-K\"ahler geometry}\label{sec:GpK}
\begin{Def}
A \textbf{generalized para-K\"ahler structure} (GpK) is a commuting pair $(\KK_+,\KK_-)$ of GpC structures, such that their product $\GG=\KK_+\KK_-$ defines a split-signature metric $G$ on $\TT$ via
\begin{align*}
G(\cdot,\cdot)\coloneqq \la \GG\cdot,\cdot\ra.
\end{align*}
\end{Def}

Because $\KK_\pm$ are GpC structures, they induce the splitting $\TT= \Lb_\pm\oplus\widetilde{\Lb}_\pm$. Moreover, because they commute, each of $\Lb_+$ and $\widetilde{\Lb}_+$ further splits to eigenbundles of $\KK_-$, so that one arrives at the decomposition of $\TT$ into four bundles:
\begin{align}\label{GpK_bundles}
\TT=\ell_+\oplus\ell_-\oplus \ellt_+\oplus \ellt_-,
\end{align}
such that the eigenbundles of $\KK_+$ are $\Lb_+=\ell_+\oplus\ell_-$ and $\tl{\Lb}_+=\ellt_+\oplus \ellt_-$, the eigenbundles of $\KK_-$ are $\Lb_-=\ell_+\oplus\ellt_-$ and $\tl{\Lb}_-=\ell_-\oplus \ellt_+$ and the eigenbundles of the generalized metric are $C_\pm=\ell_\pm\oplus\ellt_\pm$. All $\ell_\pm$ and $\ellt_\pm$ are isotropic and integrable under the Dorfman bracket \cite{Hu:2019zro} and bundles with opposite chirality are orthogonal.

Similarly to generalized K\"ahler structures, the GpK structures can be equivalently given in terms of a pair of para-Hermitian structures satisfying a particular integrability condition:
\begin{theorem}[\cite{Hu:2019zro}]
The data of a GpK structure is equivalent to a pair of para-Hermitian structures $(\eta,K_+,K_-)$, such that $\KK_\pm$ are given by
\begin{align}\label{eq:GpK_genform}
\KK_{\pm}=\frac{1}{2}
\begin{pmatrix}
\id & 0 \\
b & \id
\end{pmatrix}
\begin{pmatrix}
K_+\pm K_- & \omega^{-1}_+\mp \omega^{-1}_- \\
\omega_+\mp \omega_- & -(K_+^*\pm K_-^*)
\end{pmatrix}
\begin{pmatrix}
\id & 0 \\
-b & \id
\end{pmatrix},
\end{align}
for some $2$-form $b$. The integrability of $\KK_\pm$ then translates into $K_\pm$ being integrable and satisfying
\begin{align*}
\n^\pm K_\pm=0,
\end{align*}
where the connections $\n^\pm$ are defined by the Levi-Civita connection $\lc$ of $\eta$ and the $H$-flux:
\begin{align*}
\eta(\n^\pm_XY,Z)=\eta(\lc_XY,Z)\pm\frac{1}{2}(H+\rd b).
\end{align*}
\end{theorem}

The additional integrability condition on $K_\pm$, $\n^\pm K_\pm$, can be equivalently expressed using the fundamental forms associated to $K_\pm$, $\omega_\pm=\eta K_\pm$, and a para-complex version of the $\rd^c$-operator, which we call $\rd^p$:
\begin{align*}
\n^\pm K_\pm=0 \Longleftrightarrow \rd^p_\pm\omega_\pm=\pm (H+\rd b).
\end{align*}
$\rd^p_\pm$ here denote the $\rd^p$ operators associated to $K_\pm$, $\rd^p_\pm\coloneqq K_\pm^*\circ \rd\circ K_\pm^*$. We will call the geometry given by the data $(\eta, K_\pm, H+\rd b)$ above a {\bf bi-para-Hermitian} geometry.


\subsubsection{Isomorphism between $\TT$ and $T\oplus T$}\label{sec:isomorphism}
In both the GK and GpK cases, the product of the pair of G(p)C structures defines a generalized (indefinite) metric $\GG$, which means that
\begin{align*}
G(\cdot,\cdot)\coloneqq \la \GG\cdot,\cdot\ra,
\end{align*}
defines a genuine (indefinite) metric on the bundle $\TT$. It can be shown (see eg. \cite{Gualtieri:2003dx,Hu:2019zro}) that such metric can be always locally expressed in terms of a two-form $b$ and a metric $g$
\begin{align}\label{genmetric}
\GG=\GG(g,b)=\begin{pmatrix}
\id & 0 \\
b & \id
\end{pmatrix}
\begin{pmatrix}
0 & g^{-1} \\
g & 0
\end{pmatrix}
\begin{pmatrix}
\id & 0 \\
-b & \id
\end{pmatrix},
\end{align}
and the signature of $g$ then determines the signature of $\GG(g,b)$. Therefore, a generalized metric determines a splitting of $\TT\simeq e^b(T)\oplus T^*$ in which the eigenbundles $C_\pm$ of $\GG$ are $C_\pm=graph(\pm g)$ and the flux of this splitting is given by the closed three-form $H\overset{loc.}{=}\rd b$. In the usual splitting $\TT$, we then have $C_\pm=graph(b\pm g)$.


An important property of the generalized metric $\GG$ is that its  eigenbundles $C_\pm$ are isomorphic to the tangent bundle $T$ via the projection:
\begin{align}\label{pi_iso}
\begin{aligned}
\pi_\pm:C_\pm &\simeq T,\\
X+\ap &\overset{\pi_\pm}{\longmapsto} X,\\
X+(b\pm g)X &\overset{\pi^{-1}_\pm}{\longmapsfrom} X,
\end{aligned}
\end{align}
where $g$ and $b$ are the defining data of $\GG$. Therefore, since $\TT=C\oplus C_-$, we also have $\TT\simeq T\oplus T$:
\begin{align}
\begin{aligned}
(X_+,X_-)\overset{\pi^{-1}_+\oplus \pi^{-1}_-}{\longmapsto}& X_++(b+g)X_++X_-+(b-g)X_-\\
&=
\begin{pmatrix}
 X_++X_- \\
 g(X_+-X_-)+b(X_++X_-)
\end{pmatrix}.
\end{aligned}\label{map_pi_pm}
\end{align}
In particular, one can check that $\pi_+\oplus \pi_-$ maps the eigenbundles of the generalized structures $\KK_\pm$ ($\JJ_\pm$) to eigenbundles of the corresponding pair of tangent bundle endomorphisms $\KK_\pm$ ($\JJ_\pm$). This is because of the following relationship \cite{Hu:2019zro},
\begin{align*}
K_+=\pi_+\KK_\pm\pi_+^{-1},\quad K_-=\pm\pi_-\KK_\pm\pi_-^{-1},
\end{align*}
and analogously for $\JJ_\pm$ and $J_\pm$. From here, it is easy to see that the $+1$ eigenbundle of $\KK_+$ $\Lb_+$, for example, is given by
\begin{align*}
\Lb_+=\pi_+^{-1}(T^{(1,0)_+})\oplus\pi_-^{-1}(T^{(1,0)_-}),
\end{align*}
where $T^{(1,0)_\pm}$ denotes the $+1$ eigenbundle of $K_\pm$. Analogously, the $-1$ eigenbundle of $\KK_-$,
$\widetilde{\Lb}_-$, is given by
\begin{align*}
\widetilde{\Lb}_-=\pi_+^{-1}(T^{(0,1)_+})\oplus\pi_-^{-1}(T^{(1,0)_-}),
\end{align*}
and so on. Therefore, one should think about the pair $K_\pm$ corresponding to the GpK structure as each of the $K_\pm$ acting on its own copy of $T$, which are then mapped via $\pi_\pm$ to $C_\pm$ in $\TT$.

We will also notice that $\pi_\pm$ map the the pseudo-Riemannian structures $G$ and $g$ on each other:
\begin{align}\label{pi:gG_relationship}
g(X,Y)=\pm\frac{1}{2}\la \pi_\pm^{-1}X,\pi_\pm^{-1}Y\ra = \frac{1}{2}\la \GG \pi_\pm^{-1}X,\pi_\pm^{-1}Y\ra =\frac{1}{2}G(\pi_\pm^{-1}X,\pi_\pm^{-1}Y).
\end{align}
Another important property we will make use of is the fact that the connections $\n^\pm$ give rise via $\pi_\pm$ to a generalized connection $D$ on $\TT$ called the generalized Bismut connection \cite{Gualtieri:2007bq}:
\begin{align}\label{genBismut_pi_nabla_pm}
D_uv=\pi^{-1}_+\n^+_{\pi u}\pi_+ v_++\pi^{-1}_-\n^-_{\pi u}\pi_- v_-.
\end{align}
This connection preserves $\lara$ and $\GG$ and in the case when $\GG$ is a generalized metric corresponding to a GpK structure, $D$ also preserves all the eigenbundles in \eqref{GpK_bundles}. 

There is a tensorial quantity associated to any generalized connection called generalized torsion, which is defined as \cite{Gualtieri:2007bq}
\begin{align}\label{gentorsion_def}
T^D(u,v,w)=\la D_uv-D_vu-[u,v]_H,w\ra +\la D_wu,v\ra.
\end{align}
For the generalized Bismut connection, this torsion has only pure components in $\Lambda^3C_\pm$ and satisfies the following formula \cite[Prop.~2.29]{Hu:2019zro}:
\begin{align}\label{gentorsion_H}
T^D(\pi_\pm^{-1}X,\pi_\pm^{-1}Y,\pi_\pm^{-1}Z)=2H_b(X,Y,Z).
\end{align}


%\paragraph*{Relationship to Kapustin-Li}
%On pg. 12 of Kapustin-Li paper they implicitly use the isomorphisms $\pi_\pm$ to map the sections $\chi \in \se(T^{(0,1)_+})$ and $\lambda \in \se(T^{(0,1)_-})$ to the $-i$ eigenbundle of $\JJ_+$ as 
%\begin{align*}
%\begin{pmatrix}
%\chi+\lambda \\
%g(\chi-\lambda)
%\end{pmatrix},
%\end{align*}
%which is the same as \eqref{map_pi_pm} with $b=0$, which they implicitly absorb inside the $H$-flux.

\subsubsection{SKT Geometry and half generalized structures} \label{SKT}

An {\bf SKT} structure is a Hermitian structure $(I,g)$ for which the fundamental form $\omega$ even though is not closed, is $\rd \rd^c$-closed, i.e. $\rd \rd^c \omega=0$. The version of this geometry is easily formulated for the para- case where we will call it para-SKT. It follows directly from the definition that the bi-para-Hermitian geometry consists of two such structures -- one for each chirality -- and here we will describe how one SKT structure can be described using the language of generalized geometry.

For this purpose we define a positive-chirality\footnote{Similarly one could define a negative-chirality half G(p)C structure by changing $C_+$ for $C_-$} {\bf half generalized almost (para-)complex structure} as a pair $(\GG,\JJ_{C_+})$, where $\GG$ is a (neutral) generalized metric which induces a splitting $\TT=C_+\oplus C_-$ and an isomorphism $C_+\oplus C_-\simeq T\oplus T$, and $\JJ_{C_+}$ is a (para-)complex endomorphism of $C_+$, i.e.
\begin{align*}
\JJ_{C_+}\in \End(C_+),\quad \JJ_{C_+}=\pm \id_{C_+},\quad \la \JJ_{C_+} u_+,v_+\ra=-\la u_+,\JJ_{C_+} v_+\ra,
\end{align*}
for any $u_+,v_+ \in \se(C_+)$. The integrability condition on $\JJ_{C_+}$ is then that its eigenbundles $\ell_+\oplus \ell_-=C_+$ (in the complex case $\ell\oplus \bar{\ell}=C_+\otimes \mathbb{C}$) are involutive with respect to the (twisted) Dorfman bracket. It is easy to see that $\JJ_{C_+}$ defines a (para-)Hermitian structure $(J_+,g)$, where
\begin{align*}
g(X,Y)=\frac{1}{2}\la \pi_+^{-1}X,\pi_+^{-1}Y\ra,\quad J_+=\pi_+\JJ_{C_+}\pi^{-1}_+
\end{align*}
and the integrability condition on $\JJ_{C_+}$ translates into the condition
\begin{align*}
\n^+J_+=0,\ \n^+=\lc+\frac{1}{2}g^{-1}(H+\rd b) \Longleftrightarrow \rd^{c/p}\omega_+=H+\rd b,
\end{align*}
i.e. giving exactly half of the data of a G(p)K geometry. If there is a negative-chirality half structure $\JJ_{C_-}$, then $(\GG,\JJ\coloneqq \JJ_{C_+}\oplus \JJ_{C_-})$ defines a genuine G(p)K structure.

\section{The AKSZ formalism and para-geometry} \label{sec: AKSZ}

\def\fg{\mathfrak{g}}

In this section we will see how to associate a class of $2D$ topological field theories to generalized para-complex structures.
The construction closely follows the relationship between ordinary (non-para) generalized complex structures \cite{Gualtieri:2003dx} and $2D$ topological field theories developed in \cite{Cattaneo:2009zx, Pestun:2006rj}.
For another class of $2D$ boundary theories to the $3D$ Courant algebroid $\sigma$-model see \cite{SeveraTduality}. \btd{Please add any more references you think fit.}

For us, the prescription of defining the $2D$ topological theory associated to a generalized para-complex structure starts with a $3D$ topological field theory which is defined for any manifold $M$. 
More generally, this $3D$ theory can be defined for any Courant algebroid on $M$ \cite{Roytenberg:2002nu}, but for most examples we will only consider the standard exact Courant algebroid (or twisted versions) defined by the Dorfman bracket.
The key to our construction is to realize the data of a generalized para-complex structure in terms of its eigenbundles, which are Dirac structures in the real Courant algebroid $\TT$.
More precisely, we obtain a pair of Dirac structures $L, \Tilde{L}$ corresponding to the $\pm$ eigenbundles of the generalized para-complex structure. 

Via the AKSZ construction \cite{AKSZ}, this correspondence translates into the statement that the pair of Dirac structures define a pair of topological boundary theories of the $3D$ topological theory.
In turn, we will directly describe the BRST observables of the $2D$ theories in terms of the cohomology of the Lie algebroid $L$. 

By similar reasoning, one can obtain a single $2D$ TFT from the data of a generalized para-complex structure. 
One considers the para-complexification $(\TT) \otimes_{\RR} \Cc$ of the standard Courant algebroid. 
The real Dirac structures $L, \Tilde{L}$ combine to give a para-complex Dirac structure $L \oplus \Tilde{L}$. 
In turn, we obtain a single $2D$ TFT defined over the para-complex numbers.

In Section \ref{sec: twist}, we will relate these theories topological twists of $2D$ $(2,2)$ (para-)SUSY sigma models described in Section \ref{sec:toptwist} closely following the methods in \cite{Kapustin:2004gv} for ordinary generalized complex geometry.

All of the topological field theories in this section are of $\sigma$-model type and arise through the AKSZ formalism \cite{AKSZ}. 
Throughout, we will use the formalism of dg manifolds and shifted symplectic geometry. 
%Later we will use the idea of Lagrangian intersections \brian{PTVV, ButsonYoo, Costello,...}. 

\subsection{Courant algebroids, $3D$ TFTs, and boundary conditions}


From the point of view of topological field theory, Courant algebroids are important because they provide geometric examples of $2$-shifted symplectic spaces. 
Via the AKSZ construction, $2$-shifted symplectic spaces are the natural home for $3$-dimensional topological field theories in the BV formalism. 
We recall the construction of the $3D$ topological field theory from the data of a Courant algebroid. 
For other references, see \cite{Roytenberg:2002nu, Cattaneo:2009zx}. 

\subsubsection{Shifted symplectic geometry} 
\label{sec: dgman}

We begin by setting up our model for the theory of derived manifolds. 
For us, this is a well-behaved class of $NQ$ manifolds which are appropriate for setting up quantization of $\sigma$-models, see for instance \cite{CostelloSUSY}. 
A dg manifold is a pair $\cN = (N, \sA^*_\cN)$ where $N$ is a smooth manifold, called the body, and $\sA^*_\cN$ a sheaf of graded commutative algebras over the de Rham complex $\Omega^*_N$.
Here, $\d_\cN$ is a linear differential operator of degree $+1$, and together the data must satisfy the following conditions:
\begin{itemize}
\item[(1)] $\sA_\cN$ is concentrated in finitely many degrees;
\item[(2)] For each $k$, $\sA^k_\cN$ is a locally free sheaf of $C^\infty_N$-modules of finite rank;
\item[(3)] The differential $\d_N : \sA_\cN^* \to \sA_\cN^{*+1}$ is square zero differential operator making $(\sA_\cN , \d_\cN)$ into a sheaf of commutative dg algebras over the de Rham complex $\Omega^*_N$.
\end{itemize}

In particular, as a graded algebra $\sA^*_\cN$ is given by functions on the total space of some graded vector bundle $A_\cN$ on $N$ (which is of finite rank and concentrated in finitely many degrees). 
In the language of $NQ$ manifolds, the homological vector field defining the $Q$-structure is $\d_\cN$. 

\brian{Introduce symplectic form}

A striking result of \cite{Roytenberg:2002nu} classifies all $2$-shifted symplectic spaces in terms of Courant algebroids. 

\begin{theorem}[\cite{Roytenberg:2002nu}]
There is an equivalence between isomorphism classes of $2$-shifted symplectic NQ manifolds with body $M$ and isomorphism classes of Courant algebroids on $M$.
\end{theorem}

\begin{remark}
In \cite{PymSafronov} it was shown that a more general class of $2$-shifted symplectic derived spaces, called $L_\infty$ algebroids, are equivalent to {\em twisted} Courant algebroids. 
This is similar to a Courant algebroid, where the Jacobi identity only holds up to homotopy given by some closed $4$-form. 
\end{remark}

We briefly recount the equivalence between the data of a $2$-shifted symplectic dg manifold and the data of a Courant algebroid.

\def\Sym{{\rm Sym}}

%Given any Courant algebroid $E \to M$, we can have the following sequence of maps $T^* \xto{a^*} E \xto{a} T$ where $a$ is the anchor and $a^*$ is linear dual to the anchor.
%Using this sequence, we can define the associated $NQ$ manifold $X_E$, following \cite{Ryotenberg:2002nu}.
Let $E \to M$ be a vector bundle equipped with a fiberwise nondegenerate inner product $\left<-,-\right>$.
The body of the dg manifold is the manifold $M$, and the underlying graded manifold $X_E$ is given by $T^*M [2] \oplus E[1]$. 

We will use coordinates $\{x^i\}$ on $M$, $\{p_i\}$ for the fiber coordinate of $T^*$, and $\{e^a\}$ for the fiber of $E$. 
Note that $x^i$ is of degree zero, $\eta_i$ is of degree $2$, and $e^a$ is of degree $1$.
Suppose also that $\left<e^a, e^b\right> = g^{ab}$ and let $(g_{ab})$ be the inverse to the inner product. 
Notice, $X_E$ comes equipped with a natural $2$-shifted symplectic form, which in coordinates is 
\[
\omega_E = d x^i d p_i + g_{ab} d e^a d e^b .
\]
Denote by $\{-,-\}$ the shifted Poisson bracket corresponding to this symplectic structure. 
An arbitrary degree $3$ function on the graded manifold $X_E$ has the form
\[
\Theta = p_i a_a^i (x) e^a + \frac{1}{6} f_{abc} (x) e^a e^b e^c .
\]
Globally, $a = (a_a^i)$ defines a bundle map $a : E \to T$ and through the pairing the collection $(f_{abc})$ defines a bilinear map $[-,-] : E \times E \to E$. 
It is a result of \cite{Roytenberg:2002nu} that this function satisfies $\{\Theta, \Theta\} = 0$ if and only if the data $(E, \left<-,-\right>, a, [-,-])$ has the structure of a Courant algebroid. 
Here $a$ is the anchor map, and $[-,-]$ is the bracket. 

%\[
%Q : \Gamma(M , \Sym \left(T^*[2] \oplus E[1]\right)^*) \to \Gamma(M , \Sym \left(T^*[2] \oplus E[1]\right)^*) 
%\]
%as follows: for $\psi \in \Sym^{k + \ell + 2} \left(T^*[2] \oplus E[1]\right)^*)$ and
%\[
%(\xi_0 \otimes \cdots \otimes \xi_k) \otimes (\alpha_0 \otimes \cdots \otimes \alpha_\ell) \in \Sym^{k+1}(T^*[2]) \otimes \Sym^{\ell + 1}(E[1]) 
%\]
%define
%\begin{align*}
%(Q \psi) ((\xi_0 \otimes \cdots \otimes \xi_k) \otimes (\alpha_0 \otimes \cdots \otimes \alpha_\ell)) = \brian{finish}
%\end{align*}
%
%All that remains is to describe the $2$-shifted symplectic structure on the NQ manifold $X_E$. 
%This is defined via the obvious pairing between $T^*[2]$ and $T$ together with the pairing $\langle - ,- \rangle$ on $E[1]$.
%In local coordinates \brian{finish}

\subsubsection{The AKSZ construction}

Fix the following data:
\begin{itemize}
\item an {\em $n$-oriented} dg manifold $\cN = (N, (\sA^*_N, \d_\cN))$ with body a smooth oriented manifold $N$ and orientation $\mu$. 
\item An $(n-1)$-shifted symplectic dg manifold $(X,\omega)$. 
\end{itemize}

The starting point of the AKSZ construction is the mapping space
\begin{equation}\label{eqn: map}
\cE (\cN, X) = {\rm Map}\left(\cN, X\right)
\end{equation}
As a graded manifold, the mapping space $\cE(N,X)$ is given by the space of smooth maps between the underlying graded manifolds $A_N$ and $X$. 
The graded manifold $\cE(\cN, X)$ is equipped with the homological vector field $\d_\cN + Q$ where $Q$ is the homological vector field on $X$. 

The AKSZ construction endows the mapping space $\cE(N, X)$ in (\ref{eqn: map}) with a $(-1)$-shifted symplectic symplectic form as follows, compatible with the homological vector field $\d_{\cN} + Q$ as follows.
The fundamental observation is that the diagram
%\[
%\xymatrix{
%& N \times \cE(N, X) \ar[dr]^-{{\rm ev}} \ar[dl]_-{\pi_N} & \\
%N & & X
%}
%\]
\begin{equation*}
	\begin{tikzcd}
	& N \times \cE(N, X)\arrow{rd}{{\rm ev}}\arrow[swap]{ld}{\pi_N} &\\
	N & & X
	\end{tikzcd}
\end{equation*}
induces a pairing
\[
\begin{array}{ccccc}
 \Omega^p_N& \times & \Omega^q_X & \to & \Omega^{p+q-n}_{\cE(N, X)} \\
\displaystyle \alpha & \times & \beta & \mapsto & \displaystyle \int_N \pi_N^* \alpha \wedge {\rm ev}^* \beta 
 \end{array}
 \]
Applied to the element $1 \times \omega_X \in \Omega^0_N \times \Omega^{2}_X$ we obtain a $(-1)$-shifted symplectic form that we denote $\int_N \omega_X$.

\begin{Ex}
The most important example for us will be the source dg manifold $\cN = (N, \Omega^*_N)$, that is $\sA_\cN = \Omega^*_N$ equipped with de Rham differential. 
We denote this dg manifold by $\cN = N_{\rm dR}$.
Take the dg manifold $(N, \Omega^*_N)$ where $N$ is any smooth manifold. 
If $N$ is closed and oriented there exists an integration map
\[
\int_N : \Omega^*_N \to \RR[n]
\]
of degree $-n$ thus equipping the dg manifold $N_{\rm dR}$ with an $n$-orientation. 
\end{Ex}

\begin{Ex}
If $N$ has a para-complex structure, $(N, \mathbf{\Omega}^{0,\bullet})$ has the structure of a dg manifold. 
Here $\mathbf{\Omega}^{0,\bullet}$ is the para-Dolbeault complex from Section \ref{sec: paracomplex} with differential given by the para $\pd$-operator. 
We will denote this dg manifold by $N_{\pd}$. 
To equip the dg manifold $(N, \mathbf{\Omega}^{0,\bullet})$ with an orientation, one must fix additional data. 
One way to do this is to assume that the para-complex manifold $N$ is equipped with a para-holomorphic volume form $\Omega_N$.
\brian{Relate to Lie algebroid description.}
\end{Ex}

\brian{Hamiltonians and action functional. This is just more review and citations.}

\subsubsection{The AKSZ theory associated to a Courant algebroid}

From here on, to any Courant algebroid $E$ on a manifold $M$ we associate the dg manifold $X_E$ which carries a $2$-shifted symplectic structure. 
In the case that the Courant algebroid is exact, call the associated $2$-shifted symplectic space $X_H$, where $H$ labels the Severa class. 

We first review some basic examples of AKSZ theories associated to Courant algebroids. 
The simplest type of Courant algebroid is one over a point, see Example \ref{Ex: point}. 
The resulting AKSZ theory is a familiar one.

\begin{Ex}\label{ex: cs}
Any Lie algebra $\fg$ together with a non-degenerate invariant pairing defines a $2$-shifted symplectic structure on the graded manifold $X_E = \fg[1]$ living over a point. 
The dg algebra of functions is the Chevalley-Eilenberg cochain complex computing Lie algebra cohomology $C^*(\fg)$. 
The differential is the Chevalley-Eilenberg differential.
All such $2$-symplectic dg manifolds over a point are of this form. 
%The resulting AKSZ theory is Chern-Simons theory. 
The resulting AKSZ theory on a $3$-manifold $M$ is Chern-Simons theory on $M$. 
Indeed, the space of graded maps $N_{dR} \to \fg[1]$ is identified as a graded vector space with $\Omega^*(N, \fg)[1]$.
The linear part of the BRST operator encodes the de Rham differential on $M$ and the remaining piece encodes the Lie bracket on $\fg$. 
\end{Ex}

\begin{Ex} \label{ex: exact CA}
If $E$ is an exact Courant algebroid on $M$ let $X_E$ be the corresponding $2$-shifted symplectic manifold.
We use the same coordinates as in Section \ref{sec: dgman}.
The fields are the graded maps $M_{dR} \to X_E = T^*[2] T[1] M$ which we can identify in local coordinates with
\begin{align*}
x^i &\in C^\infty(N) \\
p_i & \in \Omega^2(N) \\
e^a & \in \Omega^1(N) 
\end{align*}
where $i = 1,\ldots, {\rm dim}(M)$ indexes the local coordinates on $M$ and $a = 1, \ldots, {\rm rank}(E)$ the coordinates on the fiber of $E$.
The action functional is
\[
S = \int_N p_i \d x^i + \frac{1}{2} g_{ab}(x) e^a \d e^b - a_a^i(x) p_i e^a + \frac{1}{6} f_{abc} (x) e^{a} e^b e^c .
\]
\end{Ex}

\subsubsection{Dirac structures and boundary conditions}

The AKSZ formalism is a construction which produces a $(-1)$-shifted symplectic space from the data of a closed manifold and a target shifted symplectic manifold. 
There is a generalization of this construction which produces $(-1)$-shifted symplectic spaces on manifolds with boundary. 

Suppose $N$ is a manifold with boundary, and let $\cE(N)$ denote the space of fields of a classical theory on $M$. 
For us, $\cE(N)$ will be a mapping space of the form ${\rm Map}(M, X)$ where $X$ is shifted symplectic. 
Let $\cE(\partial N)$ denote the restriction of the fields to the boundary of $N$. 
For the mapping space example, this is simply the space of maps ${\rm Map}(\partial N, X)$.

In general, when $N$ is not closed, the pairing defined by the AKSZ construction will not endow $\cE(N)$ will with $(-1)$-shifted symplectic structure. 
However, the space $\cE(\partial N)$ does carry a natural ordinary ($0$-shifted) symplectic structure from the restriction of the pairing defined on $\cE(N)$. 
Moreover, the natural restriction map $\cE(N) \to \cE(\partial N)$ is a Lagrangian morphism.

\def\cL{\mathcal{L}}

A boundary condition is the choice of an additional Lagrangian subspace $\cL$ of $\cE(\partial N)$.
One then forms the intersection $\cL \times_{\cE(\partial N)} \cE(N)$ of the two Lagrangians $\cL$ and $\cE(\partial N)$ inside of the phase space $\cE(\partial N)$. 
The intersection defines the fields of the theory on $\partial N$. 
When the bulk theory is trivial, so there are no degrees of freedom, the resulting theory on $\partial N$ is also equipped with a $(-1)$-symplectic structure. 
For proofs of these facts about the Lagrangian intersection we refer to \cite{Calaque,PTVV}.

This formalism can be made explicit in the case of the $3D$/$2D$ system. \brian{I basically do this already below, but I might add some filler statements here.}

\begin{Ex}
Suppose $(M, \pi)$ is a Poisson manifold. 
Then consider the graph of $\pi$ as a Lagrangian subbundle of $\TT$:
\[
{\rm Graph}(\pi) = \{(\pi \vee \alpha , \alpha) \; | \; \alpha \in T^* \} \subset \TT .
\]
By the Jacobi identity, this subbundle is integrable and so determines a Dirac structure. 
If we place the 3D AKSZ theory with target the Courant algebroid $\TT$ on $N^3$ with $\partial N = \Sigma$, the boundary condition determined by this Dirac structure is equivalent to the Poisson $\sigma$-model on $\Sigma$, see \cite{KSSdirac}, for instance.
\end{Ex}

\subsection{Two dimensional TFT from generalized para-complex structures}

Let $M$ be a manifold.
Throughout this section we will only consider the standard Dorfman Courant algebroid $X_H$ on $M$, possibly twisted by an $H$-flux.
By Theorem \ref{thm:pairofdirac}, the choice of a generalized para-complex structure $\KK$ determines a pair of transversal Dirac structures $L_{\KK}$ and $\tilde{L}_{\KK}$ on $M$ given by the positive and negative eigenbundles of $\KK$. 
In particular, both $L_{\KK}, \Tilde{L}_{\KK}$determine Lagrangians in the shifted symplectic space $X_{H}$. 
This Lagrangian then defines a pair of boundary conditions for the three-dimensional AKSZ theory.
Thus, we see that any generalized para-complex structure $\KK$ defines a 2D TFT living at the boundary of the 3D theory. 
%By Proposition \ref{prop:dirac_Liealg} the eigenbundle $L_\KK$ has the structure of a Lie algebroid compatible with the anchor and Dorfman bracket on $\TT$. 

More generally, if $\KK$ is a generalized para-complex structure in an arbitrary Courant algebroid $E$, then we obtain a pair of $2$-shifted Lagrangians inside of the $2$-shifted symplectic space $X_E$. 
Many of our examples take $E$ to be the standard Courant algebroid, but the following discussion applies in general. 
We will continue to denote $L_{\KK}$ and $\Tilde{L}_{\KK}$ the $\pm$ eigenbundles in $E$. 

There is a different presentation of the $2$-symplectic manifold $X_E$ in terms of $L_{\KK}$ and $\Tilde{L}_{\KK}$ that we will use in our identification of the boundary $2D$ TFT's. 
Consider the graded manifold $T^*[2] \Tilde{L}_{\KK} [1]$.
In coordinates, we can parametrize this graded manifold by: $\{l^m, \Tilde{l}_n, x^i, p_j\}$ where $l^m$ parameterizes the fiber of $\Tilde{L}^*[1] \cong L [1]$ and is of degree $1$, $\Tilde{l}_n$ parametrizes the fiber of $\Tilde{L}[1]$ and is of degree $1$, and $x^i, p_j$ are as before.

\brian{define $\omega_{\KK}$ see Appendix A of \cite{Cattaneo:2009zx}. 
If dbro is OK, I say we just translate his result and cite it. The proof seems to follow immediately.}
\begin{proposition}\label{prop: symp}
Suppose $\KK$ is a generalized para-complex structure on a Courant algebroid $E$.
Then, there are symplectomorphisms of $2$-shifted symplectic dg manifolds
\[
\left(T^*[2] \Tilde{L}, \omega_{\KK} \right) \cong X_E \cong \left(T^*[2] \overset{\;}{L}, \Tilde{\omega}_{\KK} \right) .
\]
\end{proposition}
\begin{proof}
The proof is completely analogous to the proofs of the results in Appendix A of \cite{Cattaneo:2009zx}.
\end{proof}

Equipped with the description of the $2$-symplectic manifold $X_E$ in terms of the generalized para-complex structure of $E$, we can read off the resulting $2D$ boundary TFT as follows.
The $3D$ theory is given by the AKSZ formalism as ${\rm Map}(N_{dR}, X_E)$ where $N_{dR}$ is a smooth oriented $3$-manifold.
In the case that $\partial N = \Sigma$, we see that by Proposition \ref{prop: symp} the phase space can be written as
\[
{\rm Map}(\Sigma_{dR} , X_E) \cong {\rm Map}(\Sigma_{dR}, T^*[2] \Tilde{L}_{\KK}) 
\]
which is a dg manifold equipped with a natural $0$-shifted symplectic structure.
On the right-hand side, we use the symplectic structure defined by $\omega_{\KK}$. 

The $2D$ theory is given as a Lagrangian inside of this symplectic dg manifold. 
In the coordinates granted by Proposition \ref{prop: symp} this Lagrangian takes a particularly nice form.
The Dirac structure $L_\KK$ defines the Lagrangian submanifold $L_\KK[1] \subset T^*[2] \Tilde{L}_{\KK}$. 
At the level of the AKSZ theory, this Lagrangian is simply
\[
{\rm Map}(\Sigma_{dR}, L_{\KK} [1])  \hookrightarrow {\rm Map}(\Sigma_{dR} , T^*[2] \Tilde{L}_{\KK}) .
\]
For a general Courant algebroid $E$, this graded manifold is equipped with a natural $(-1)$-shifted Poisson structure. 
In the case that $E$ is exact, the AKSZ construction equips the $\sigma$-model with a $(-1)$-shifted symplectic structure. 
We will refer to this boundary condition as the {\bf positive} $2D$ TFT corresponding to the generalized para-complex structure $\KK$. 

An analogous construction applies to the $(-)$-eigenbundle.
There, we present the phase space as ${\rm Map}(\Sigma_{dR}, T^*[2] L_{\KK})$ equipped with the symplectic structure $\Tilde{\omega}_{\KK}$. 
The Lagrangian is ${\rm Map}(\Sigma_{dR}, \Tilde{L}_{\KK} [1])$. 
Again, when $E$ is exact this is equipped with a $(-1)$-symplectic structure. 
We will refer to this boundary condition as the {\bf negative} $2D$ TFT corresponding to the generalized para-complex structure $\KK$. 

A completely similar construction works at the level of the para-complexification. 
By Proposition \ref{prop: pcomplexified} the sum $L_\KK \oplus \Tilde{L}_{\KK}$ determines a Dirac structure on $E \otimes_\RR \Cc$. 
The proof of the following is identical to Proposition \ref{prop: symp}. 

\begin{proposition}\label{prop: complexifiedsymp}
Suppose $\KK$ is a generalized para-complex structure on a Courant algebroid $E$, and consider the para-complexified Courant algebroid $E \otimes_{\RR} \Cc$. 
Then, there is a symplectomorphisms of $2$-shifted symplectic dg manifolds
\[
X_{E\otimes \Cc} \cong \left(T^*[2] (L \oplus \Tilde{L}), \Tilde{\omega}_{\KK} \right) .
\]
\end{proposition}

By the same reasoning as above, we see that the Lagrangian $(L \oplus \Tilde{L})[1] \subset T^*[2] (L \oplus \Tilde{L})$ determines a the $2D$ TFT with fields
\[
{\rm Map} \left(\Sigma_{dR}, (L \oplus \Tilde{L})[1]\right)
\]
lying at the boundary of the para-complexified Courant $\sigma$-model. 
Again, in the case that $E$ is exact, the AKSZ construction endows this $2D$ $\sigma$-model with a $(-1)$-symplectic structure. 
We will refer to this boundary condition as the {\bf full} $2D$ TFT corresponding to the generalized para-complex structure $\KK$. 

There is a familiar description of the local BRST operators of the resulting theory.
First, we recall the following general construction.
Given any Lie algebroid $L$ on $M$ consider the graded manifold $L[1]$ with body $M$. 
Functions on this graded manifold are given by sections of the graded vector bundle
\[
\cO(L[1]) = \bigoplus_{k \geq 0} \wedge^k L [-k] .
\]
Here, $\wedge^k(L)$ is placed in cohomological degree $+k$. 
There is square-zero derivation $\d_L$ of degree $+1$ on $\cO(L[1])$ defined by the following rule. 
If $\varphi$ is a section of $\wedge^k L$, then
\begin{align*}
\left( \d_L \varphi \right) (\ell_0, \ldots, \ell_k) & = \sum_{i = 0}^k (-1)^i a (\ell_i) \varphi(\ell_0, \ldots, \Hat{\ell_i}, \ldots, \ell_k) \\ & + \sum_{i < j} (-1)^{i+j} \varphi([\ell_i, \ell_j] , \ell_0 \cdots, \Hat{\ell_i}, \ldots, \Hat{\ell_j}, \ldots, \ell_k)
\end{align*}
Here $a : L \to T_M$ is the anchor and $[-,-]$ is the bracket on $L$. 
In other words $\d_L$ is a homological vector field on $L$, which can be written in coordinates:
\[
\d_L = a_a^i \theta^a \frac{\partial}{\partial x^i} + f_{ab}^c \theta^a \theta^b \frac{\partial}{\partial \theta^c}
\]
where $\{x^i\}$ are the coordinates on $M$ and $\{\theta^a\}$ are the coordinates for the fiber of $L[1]$.

Equipped with this differential, the cohomology
\[
H^*( \cO(L[1]), \d_L )
\]
is called the Lie algebroid cohomology of $L$.

\begin{proposition}
Let $\KK$ be a generalized para-complex structure on a Courant algebroid $E$.
The following is true about the local observables of the $2D$ TFTs determined by $\KK$:
\begin{itemize}
\item[(1)] the BRST cohomology of the positive $2D$ TFT associated to $\KK$ is isomorphic to the Lie algebroid cohomology of the real Lie algebroid $L_{\KK}$; 
\item[(2)] the BRST cohomology of the negative $2D$ TFT associated to $\KK$ is isomorphic to the Lie algebroid cohomology of the real Lie algebroid $\Tilde{L}_{\KK}$;
\item[(3)] the BRST cohomology of the full $2D$ TFT is isomorphic to the Lie algebroid cohomology of para-complex Lie algebroid $L_{\KK} \oplus \Tilde{L}_{\KK}$.
 \end{itemize}
\end{proposition}

\begin{proof}
The BRST operator is of the form $\d_{dR} + Q$ where $Q$ is a degree $+1$ vector field on the target and $\d_{dR}$ is the de Rham differential on $\Sigma$.  
Locally, the BRST cohomology reduces to the cohomology of functions on the target graded manifold equipped with differential $Q$.
In each case at hand, the target is of the form $L[1]$ with differential $Q = \d_L$.
\end{proof}

We now restrict to the case $E = \TT$ with its standard Courant algebroid structure and outline important special cases.

\subsubsection*{Trivial generalized para-complex structures}

Let $M$ be a smooth manifold.
Consider the case of the trivial generalized para-complex structure on $M$ defined by
\begin{align*}
\KK_0=
\begin{pmatrix}
\id & 0 \\
0 & -\id
\end{pmatrix}.
\end{align*}
The positive eigenbundle is simply $L_\KK = T$ and the resulting Lie algebroid structure is the standard one.
The positive $2D$ theory is the trivial $2D$ theory.  

The negative eigenbundle is $L_{\KK} = T^*$ with zero Lie algebroid structure.
The negative $2D$ theory is given by the AKSZ theory
\[
{\rm Map}(\Sigma_{dR} , T^*[1] M)
\]
where $T^*[1]M$ is equipped with its standard $1$-symplectic structure. \david{so is this the BFF theory?}

\subsubsection*{Poisson $\sigma$-model}
Consider the case of the generalized para-complex structure on $M$ defined by
\begin{align*}
\KK_0=
\begin{pmatrix}
\id & 2 \beta \\
0 & -\id
\end{pmatrix}.
\end{align*}
where $\beta \in \wedge^2 T$.
We have seen that this is integrable if and only if the bivector $\beta$ is a Poisson structure. 

The positive eigenbundle is $L_{\KK} = T$ with standard Lie algebroid structure. 
Thus, the negative $2D$ theory is the same as in the previous example.

The negative eigenbundle is $\Tilde{L}_\KK = T^*$ equipped with Lie algebroid structure determined by the Poisson structure $\beta$. 
The $2D$ TFT is the Poisson $\sigma$-model with target $(M, \beta)$. 

\subsubsection*{Para-complex $A/B$-models}
Consider the case of the generalized para-complex structure on $M$ defined by
\begin{align*}
\KK_\omega =
\begin{pmatrix}
0 & \omega^{-1} \\
\omega & 0
\end{pmatrix}.
\end{align*}
where $\omega \in \wedge^2 T^*$ is a symplectic form.

The positive and negative eigenbundles are isomorphic to $L = T^*$ with anchor $\omega^{-1} : T^* \to T$. 
The $2D$ AKSZ theory is equivalent to the usual $A$-model. 

Finally, consider 
\begin{align*}
\KK_K =
\begin{pmatrix}
K & 0 \\
0 & -K^*
\end{pmatrix}.
\end{align*}
where $K$ is an ordinary para-complex structure, see Example \ref{eg: pc}. 

In this case, the $2D$ theories specialize to the following para-complex versions of the $B$-model.
First, $L_{\KK}$ can be identified with $T^{1,0} \oplus T^{*0,1}$ and $\Tilde{L}_{\KK}$ is $T^{0,1} \oplus T^{*1,0}$. 

\begin{Def}
The {\bf para-$B$-model} with source a closed Riemann surface $\Sigma$ and target a para-complex manifold $X$ is the AKSZ theory with source the de Rham space $\Sigma_{\rm dR}$ and target the $1$-shifted symplectic space $T^*[1] X_{\pd}$:
\[
{\rm Map}\left(\Sigma_{dR}, T^*[1]  X_{\pd}\right) .
\]
\end{Def}

The classical BRST cohomology of the para-B-model is isomorphic to the cohomology of para-holomorphic polyvector fields on the target para-complex manifold $X$.

\btd{I'm getting snagged on this a bit. 
What I want to show is that the full theory has as its BRST cohomology the complex I describe below. 
Do you see how this works?
}

The Lie algebroid cohomology $H^*(\cO(L_{\KK}[1]), \d_{L_\KK})$ is computed by a differential of the form
\[
\d_{L_\KK} : \Gamma(M , \wedge^k (T^{1,0} \oplus T^{*0,1})) \to \Gamma(M , \wedge^{k+1} (T^{1,0} \oplus T^{*0,1})) 
\]
Using the splitting $\wedge^*(T^{1,0} \oplus T^{*0,1}) = \wedge^* T^{1,0} \otimes \wedge^*(T^{*0,1})$ we can identify this differential with the Dobleault differential for the bundle of para-holomorphic polyvector fields $\Theta = \wedge^* T^{1,0}$:
\[
\d_L = \pd : \mathbf{\Omega}^{0,q} (M , \Theta) \to \mathbf{\Omega}^{0,q+1} (M , \Theta)  .
\]
Thus, the BRST cohomology is precisely the Dolbeault cohomology of para-holomorphic polyvector fields.

%\brian{Case of para structure on $M \times M$. Two copies of $BF$ theory.}

\subsection{Anomalies and para holomorphic variants}

\brian{discuss anomalies. Para-holomorphic volume form.}

\brian{Discuss Si's work on perturbative quantization}

%\brian{***para Kahler example, **example where the paracomplex structures commute but are not equal (this is where people see ``twisted chiral"), para version of T-duality, **Poisson $\sigma$-model, going back and forth between generalized (para) complex.
%}  

\brian{My plan is to just turn this into an extended remark}

So far, in each of the $\sigma$-models we have discussed in the AKSZ formalism, the fields have depended only topologically on the source Riemann surface. 
There are closely related $\sigma$-models which depending {\em para-holomorphically} on the source Riemann surface that we briefly discuss. 
\section{Two-dimensional parasupersymmetry}  \label{sec: parasusy}

In this section we introduce the $(2,2)$ para-SUSY algebra in two dimensions, which is an extension of the $(1,1)$ SUSY similarly to the usual $(2,2)$ SUSY. We then realize this superalgebra as a symmetry of a $2D$ $\sigma$-model.

It was shown in \cite{HullTwistedSUSY} that in order to do this, one needs to introduce a pair of para-Hermitian structures $(\eta,K_\pm)$ that form a so-called bi-para-Hermitian geometry. In \cite{Hu:2019zro} it was then shown that such geometry can be equivalently described by {\bf generalized para-K\"ahler} geometry. Because generalized para-K\"ahler geometry is defined by a pair of generalized para-complex structures that obey extra algebraic conditions (see Section \ref{sec:GpK}), this geometry supports the topological $\sigma$-models discussed in Section \ref{sec:GpC_AKSZ}.

The main content of the present section is therefore to define the topological twists of the $(2,2)$ para-SUSY sigma models and show that they are exactly equal to the topological sigma models defined by the pair of GpC structures.
%
%It has been known since \cite{Zumino:1979et} that the $(1,1)$ supersymmetric $\sigma$-model \eqref{eq:(1,1)action} admits $(2,2)$ supersymmetry when $(M,g)$ is a K\"{a}hler manifold. In general, $(2,2)$ supersymmetry in fact requires the target to be {\bf generalized K\"ahler} \cite{Gualtieri:2003dx}, which is equivalent to a data of a bi-Hermitian geometry discovered in \cite{Gates:1984nk}. For $(2,2)$ para-SUSY, the story is analogous. 
%
%will see later that in order to realize the para-$(2,2)$ superalgebra as a symmetry of a $2D$ $\sigma$ model, the target manifold must carry a pair of para-complex structures $K_\pm$. This is in contrast to the usual case that requires complex structures $I_\pm$ on the target instead.

\subsection{A reminder on $(1,1)$ supersymmetry}

The $2$-dimensional $(1,1)$ supertranslation algebra is the real super Lie algebra
\[
\mathfrak{t}_{(1,1)} = \RR^{1,1} \oplus \Pi (S_+) \oplus \Pi(S_-)
\]
where $S_\pm \cong \RR$ are the real spin representations of ${\rm Spin}(1,1)$ labeled by helicities $\pm \frac{1}{2}$. 
Note that there are natural ${\rm Spin}(1,1)$-equivariant maps
\[
\Gamma_{\pm} : {\rm Sym}^2(S_\pm) \to \RR^{1,1}
\]
which are non-degenerate and whose images hit the subalgebras spanned by $\partial_{\pm}$, respectively. 
The only nonvanishing Lie brackets in $\mathfrak{t}_{(1,1)}$ are defined by
\[
[Q_+, Q_+'] = \Gamma_+(Q_+ \otimes Q_+') \;\; , \;\; [Q_-, Q_-'] = \Gamma_-(Q_- \otimes Q_-')
\]  
where $Q_\pm, Q_\pm'$ are generic elements in $S_\pm$. 

We will use the lightcone basis $\p_{\pm}$ for $\RR^{1,1}$ and we also fix a basis $Q_\pm^1$ for $S_{\pm}$ with a relative normalization so that
\begin{align}\label{eq:(1,1)_susy}
[Q^1_\pm,Q^1_\pm] = 2\p_\pm.
\end{align}

\subsubsection{The $(1,1)$ supersymmetric $\sigma$-model} 

Let $\Sigma^{1,1|2}$ be a surface of split signature with two real odd directions. 
The even part of this supermanifold is a real surface (with coordinates $\tau$, $\sigma$).
For instance, in the flat case we consider the superspace $\RR^{1,1|2} = \{(\tau,\sigma ; \theta_1^{\pm})\}$, where $\theta_1^{\pm}$ are odd variables. 

The most general $(1,1)$ supersymmetric sigma model into a target pseudo-Riemannian manifold $(M,\eta)$ is given by the action functional
\begin{align}\label{eq:(1,1)action}
S_{(1,1)}(\Phi)=\int_{\Sigma^{1,1|2}} [g(\Phi)+b(\Phi)]_{ij}D^1_+\Phi^iD^1_-\Phi^j,
\end{align}
where $\Phi=(\Phi^i)_{i=1\cdots n}$ are local representatives for the $(1,1)$ {superfields} which are maps $\Phi: \Sigma^{1,1|2} \rightarrow M$.
In the action, $b$ denotes a local two-form potential for a closed three-form, $H=\rd b$, and $D^1_\pm$ are the superderivatives
\begin{align}\label{eq:D1}
D^1_\pm=\frac{\p}{\p \theta_1^\pm}-\theta_1^\pm \p_\pm,
\end{align}
where $\p_\pm=\frac{\p}{\p x_\pm}$ are derivatives with respect to the lightcone coordinates on $\Sigma$, $x_\pm= \frac{1}{\sqrt{2}} (\tau\pm\sigma)$. In superworldsheet coordinates, the superfields decompose as
\begin{align}\label{fields_(1,1)}
\Phi^i(\sigma,\tau,\theta_+,\theta_-)=\phi^i(\sigma,\tau)+\theta^+\psi^i_+(\sigma,\tau)+\theta^-\psi_-(\sigma,\tau)+\theta^+\theta^-F^i(\sigma,\tau).
\end{align}
Here, $(\phi^i)$ are local representatives for a smooth map $\phi: \Sigma \rightarrow M$. The fact that this action carries $(1,1)$ supersymmetry means that the action \eqref{eq:(1,1)action} is invariant under transformations generated by the two {supercharges} $Q^1_\pm$, 
\begin{align}\label{eq:Q1}
Q^1_\pm=\frac{\p}{\p \theta_1^\pm}+\theta_1^\pm \p_\pm,
\end{align}
obeying the supercommutation relations of the $(1,1)$ supersymmetry algebra (\ref{eq:(1,1)_susy}). 
\begin{remark}
The sigma model \eqref{eq:(1,1)action} is fully determined by the geometric data of the generalized pseudo-metric $\GG(g=\eta,b)$ \eqref{genmetric}.
\end{remark}

\paragraph{Expanding the $(1,1)$ action}
We expand the action \eqref{eq:(1,1)action} by plugging in \eqref{fields_(1,1)} and \eqref{eq:D1}. The expansions read
%In components, the superfield has the form
%\[
%\Phi^i =\phi^i+\theta^+\psi^i_++\theta^-\psi^i_-+\theta^+\theta^-F^i
%\]
%The super covariant derivatives $D_{\pm}$ applied to $\Phi^i$ read
\[
D_\pm\Phi^i =\psi^i_\pm\pm\theta^\mp F^i-\theta^\pm\p_\pm\phi^i\mp\theta^+\theta^-\p_\pm\psi^i_\mp,
\]
and
\begin{align*}
[\eta(\Phi)+b(\Phi)]_{ij} = & [\eta(\phi)+b(\phi)]_{ij}+\p_k[\eta(\phi)+b(\phi)]_{ij}(\theta^+\psi_+^k+\theta^-\psi_-^k+\theta^+\theta^-F^k)\\
& +\frac{1}{2}\p_k\p_l[\eta(\phi)+b(\phi)]_{ij}(\theta^+\psi_+^k+\theta^-\psi_-^k)(\theta^+\psi_+^l+\theta^-\psi_-^l).
\end{align*}
After plugging the above in the action and performing the odd integration, the only terms that survive are the $\theta^+\theta^-$ coefficients. From this we find that
%\begin{align*}
%\theta^+\theta^-&\left[(g+b)_{ij}(\psi_+^i\p_-\psi^j_++\psi^j_-\p_-\psi^i_-+F^iF^j+\p_+\phi^i\p_-\phi^j)\right. \\
%&+\p_k(g+b)_{ij}\left(F^k\psi^i_+\psi^j_-+\psi^k_+(-F^i\psi^j_--\psi^i_+\p_-\phi^j)+\psi^k_-(-\p_+\phi^i\psi^j_-+\psi_+^iF^j)\right)\\
%&\left. +\frac{1}{2}\p_k\p_l(g+b)_{ij}(-\psi_+^k\psi_-^l\psi_+^i\psi_-^j+\psi_-^k\psi_+^l\psi^i_+\psi^j_-) \right].
%\end{align*}
the field $F$ enters only algebraically and not via its derivatives. Therefore, it is auxiliary and can be integrated out. To obtain the equations of motion for $F$,
%we use the formula $\p_kg_{ij}=\Gamma_{ikj}+\Gamma_{jki}$ and collect the terms involving $F$:
%\begin{align*}
%g_{ij}F^iF^j+&(\Gamma_{ikj}+\Gamma_{jki})(F^k\psi^i_+\psi^j_--F^i\psi^k_+\psi^j_-+F^j\psi^k_-\psi_+^i)\\
%+&F^k\psi^i_+\psi^j_-(\p_kb_{ij}+\p_ib_{jk}+\p_jb_{ki}).
%\end{align*}
%Using $\Gamma_{ikj}=\Gamma_{kij}$ and $H_{ijk}=(\p_kb_{ij}+\p_ib_{jk}+\p_jb_{ki})$, we can rewrite this as
%\begin{align*}
%g_{ij}F^iF^j+2\Gamma_{ikj}F^j\psi^k_-\psi_+^i+F^k\psi^i_+\psi^j_-H_{ijk}=g_{ij}(F^iF^j+2F^j(\Gamma^-)_{lk}^i\psi_-^k\psi_+^l),
%\end{align*}
%where $\Gamma^-$ are the Christoffel symbols of the connection $\n^-=\lc-\frac{1}{2}g^{-1}H$. This yields the equation of motion for $F$,
we vary the resulting bosonic action with respect to $F^i$:
\begin{align} \label{F_EoM}
\frac{\delta S_{(1,1)}}{\delta F^i}=0\quad \Longleftrightarrow\quad  F^i=-(\Gamma^-)^i_{jk}\psi^k_-\psi^j_+.
\end{align}
Here, $\Gamma^-$ are the Christoffel symbols of the connection $\n^-=\lc-\frac{1}{2}\eta^{-1}H$.

\subsection{$(2,2)$ para-supersymmetry}\label{sec:(2,2)parasusy}
By requiring that the target manifold $M$ be equipped with additional geometric structure, it is possible to enhance the symmetry by the $(1,1)$-supersymmetry algebra to a bigger supersymmetry algebra.
We will consider the so-called $(2,2)$ para-supersymmetry algebra. 
This version of supersymmetry and the corresponding sigma models were first discussed in \cite{HullTwistedSUSY}.

As before, let $S_\pm$ be the real spin representations of ${\rm Spin}(1,1)$.
The para (2,2) super Poincar\'{e} Lie algebra is of the form $\mathfrak{so}(1,1) \ltimes \mathfrak{t}_{(2,2)}$ where $\mathfrak{t}_{(2,2)}$ is the {\bf (2,2) para-supersymmetry algebra} which has as its underlying supervector space
\[
\mathfrak{t}_{(2,2)} = \RR^{1,1} \oplus \Pi(S_+ \oplus S_+) \oplus \Pi(S_- \oplus S_-) ,
\]
with the Lie brackets defined as follows.
Here $\RR^{1,1}$ is the abelian Lie algebras of translations generated in light cone coordinates by the basis elements $\{\partial_{\pm}\}$.
To write down the Lie bracket, fix a basis $\{Q^1_{\pm}, Q^2_\pm\}$ for $S_\pm \oplus S_\pm$.
The nontrivial brackets are defined by 
\begin{equation}\label{eq:(2,2)para}
(2,2)\ \text{para-SUSY:}\quad [Q_\pm^1, Q_\pm^1] = 2 \partial_{\pm}, \quad [Q_{\pm}^2, Q_{\pm}^2] = -2 \partial_{\pm} .
\end{equation}

\begin{remark}
Note that this differs from ordinary $(2,2)$ supersymmetry by a sign:
\begin{equation*}
(2,2)\ \text{(ordinary) SUSY:}\quad [Q_\pm^1, Q_\pm^1] = 2 \partial_{\pm}, \quad [Q_{\pm}^2, Q_{\pm}^2] = 2 \partial_{\pm} .
\end{equation*}
\end{remark}

We can also write the para-SUSY algebra in the more conventional off-diagonal basis $\{{Q}_\pm,\wtl{Q}_\pm\}$, in which the only non-zero brackets read:
\begin{align*}
[{Q}_\pm,\wtl{Q}_\pm]=-2\p_\pm.
\end{align*}
The two bases $\{Q^1_{\pm}, Q^2_\pm\}$ and $\{{Q}_\pm,\wtl{Q}_\pm\}$ are related by
\begin{align}\label{parasusy_offdiagonal_basis}
Q_\pm=\frac{1}{\sqrt{2}}(Q^2_\pm-Q^1_\pm),\quad \wtl{Q}_\pm=\frac{1}{\sqrt{2}}(Q^2_\pm+Q^1_\pm).
\end{align}
%Again, this is analogous to the complex basis of the ordinary $(2,2)$ superalgebra
%\begin{align}\label{complex_(2,2)}
%[Q_\pm,\overline{Q}_\pm]=2i\p_\pm.
%\end{align}

\begin{remark}
\label{rmk: generalsusy}
More generally, suppose $W_\pm$ are vector spaces of dimension $k_\pm$, respectively.
In addition, suppose $\eta_\pm$ are symmetric nondegenerate forms on $W_\pm$. 
Then, we can define a super Lie algebra of the form
\[
\mathfrak{t}_{W} =  \RR^{1,1} \oplus \Pi(S_+ \otimes W_+) \oplus \Pi(S_- \otimes W_-)
\]
with brackets defined by
\[
[Q_\pm^{a}, Q_\pm^{b}] = 2 \eta^{ab}_\pm \partial_\pm .
\] 
This is a general form of the $\cN = (k_+, k_-)$ supersymmetry algebra for which ordinary supersymmetry and para-supersymmetry are special cases.
The case of $\mathfrak{t}_{(2,2)}$, which will be of most interested to us, corresponds to taking $k_+ = k_- = 2$ and 
\[
\eta^{ab}=\begin{pmatrix}
1 & 0 \\
0 & -1
\end{pmatrix}
\]
whereas ordinary $(2,2)$ supersymmetry corresponds to taking $\eta^{ab} = \delta^{ab}$. 

\end{remark}

\paragraph{R-symmetry}
%We have seen above that the $(2,2)$ para-SUSY algebra $\mathfrak{t}_{(2,2)}$ is diagonalised in terms of the charges
%\begin{align}\label{eq:R-sym_charges}
%\begin{aligned}
%[Q_\pm^1,Q_\pm^1]&=2\p_\pm\\
%[Q_\pm^2,Q_\pm^2]&=-2\p_\pm.
%\end{aligned}
%\end{align}
%We make a change of basis by the formulas $Q_\pm=\frac{1}{\sqrt{2}}(Q^2_\pm+Q^1_\pm)$ and $\tl{Q}_\pm=\frac{1}{\sqrt{2}}(Q^2_\pm-Q^1_\pm)$. 
There is a symmetry of the $(2,2)$ para-SUSY algebra \eqref{eq:(2,2)para} that separately rotates the $\pm$-chirality charges, called the R-symmetry. To see this, we write the equations \eqref{eq:(2,2)para} collectively as
\begin{align}\label{eq:R-sym_eta}
[Q_\pm^a,Q_\pm^b]=2\eta^{ab}\p_\pm, \quad 
\eta^{ab}=\begin{pmatrix}
1 & 0 \\
0 & -1
\end{pmatrix}
\end{align}
From here it is easy to see that matrices $(M_\pm)^a_b$ that rotate the $\pm$-chirality charges among each other while preserving the commutation relations \eqref{eq:R-sym_eta},
\begin{align*}
[(M_\pm)^a_cQ_\pm^c,(M_\pm)^b_dQ_\pm^d]=2\eta^{ab}\p_\pm
\end{align*}
have to satisfy
\begin{align}\label{r_symmetry_matrix}
(M_\pm)^a_c\eta^{cd}(M_\pm)^a_d=\eta^{ab},
\end{align}
i.e. $M_\pm$ have to lie in $SO(1,1)$ for each subscript $\pm$. 
Thus, we find that the R-symmetry group $G_R=SO(1,1)_+\times SO(1,1)_-$ acts on $\mathfrak{t}_{(2,2)}$.

%\david{this is only relevant for the (2,2) formalism, will se if we need this or not}
%The R-symmetry acts on the odd coordinates accordingly as
%\begin{align*}
%\begin{pmatrix}
%\theta^1_\pm \\
%\theta^2_\pm	
%\end{pmatrix}
%\mapsto
%\begin{pmatrix}
%\cosh(\alpha_\pm) & \sinh(\alpha_\pm) \\
%\sinh(\alpha_\pm) & \cosh(\alpha_\pm)
%\end{pmatrix}
%\begin{pmatrix}
%\theta^1_\pm \\
%\theta^2_\pm	
%\end{pmatrix}
%\end{align*}
%or, equivalently, $(\theta^\pm,\tth^\pm)\mapsto (e^{\alpha_\pm} \theta^\pm,e^{-\alpha_\pm}\tth^\pm)$. There are therefore two inequivalent embeddings of the Lorentz group $SO(1,1)$ into the R-symmetry group, we label them by $V$ and $A$:
%\begin{align*}
%R_V(\alpha)&:(\theta^\pm,\tth^\pm)\mapsto (e^\alpha \theta^\pm,e^{-\alpha}\tth^\pm)\\
%R_A(\alpha)&:(\theta^\pm,\tth^\pm)\mapsto (e^{\pm \alpha} \theta^\pm,e^{\mp \alpha}\tth^\pm),
%\end{align*}
%mapping the supercharges as
%\begin{align*}
%R_V(\alpha):& (Q_\pm,\tl{Q}_\pm)\mapsto (e^{-\alpha}Q_\pm,e^{\alpha}\tl{Q}_\pm)\\
%R_A(\alpha):& (Q_\pm,\tl{Q}_\pm)\mapsto (e^{\mp\alpha}Q_\pm,e^{\pm \alpha}\tl{Q}_\pm).
%\end{align*}
%The generators of these transformations are
%\begin{align*}
%F_{V/A}=\theta^+\frac{\p}{\p\theta^+}-\tth^+\frac{\p}{\p\tth^+}\pm \theta^-\frac{\p}{\p\theta^-}\mp\tth^-\frac{\p}{\p\tth^-}.
%\end{align*}
%
%Because $F_{V/A}$ has same commutators with other operators as $M$, the Lorentz generator, and additionally all $F_{V/A}$ and $M$ commute among themselves, we can define new "Lorentz" generators in one of the two following ways:
%\begin{align*}
%M_A=M+F_V,\quad M_B=M+F_A.
%\end{align*}

\subsubsection{The $(2,2)$ parasupersymmetric $\sigma$-model}
We now describe the $2D$ sigma models that carry the $(2,2)$ para-supersymmetry described above.
%\paragraph{(2,2) SUSY}
%We propose the following symmetry by the $(2,2)$ parasupersymmetry algebra (\ref{(2,2)para}) (setting $\theta^\pm_1\coloneqq \theta^\pm$ for simplicity), which read in superspace notation:
%\begin{align*}
%Q^1_\pm&=\frac{\p}{\p\theta^\pm}+\theta^\pm\p_\pm \\
%Q^2_\pm\Phi^i&=(K_\pm)^i_j(\Phi)D_\pm\Phi^j \\
%\end{align*}
%where $D_\pm =\frac{\p}{\p\theta^\pm}-\theta^\pm\p_\pm$. 
\paragraph{Extended supersymmetry}

When $\Sigma = \RR^{1,1|2}$, the $\sigma$-model \eqref{eq:(1,1)action} has manifest $(1,1)$ supersymmetry.  We now impose additional symmetries of the action $S_{(1,1)}$ that generate the $(2,2)$ parasupersymmetry algebra (\ref{eq:(2,2)para}) and discuss the required geometric properties of the target manifold for such extension.
%In general, this extended symmetry exists only after we make some additional assumptions on the geometric data prescribing the action. 

Setting $\theta^\pm_1\coloneqq \theta^\pm$ for notational simplicity, we propose the following $(2,2)$ para-supersymmetry algebra representation on the $(1,1)$ superfields:
\begin{align}
\label{q1} Q^1_\pm&=\frac{\p}{\p\theta^\pm}+\theta^\pm\p_\pm \\
\label{q2} Q^2_\pm\Phi^i&=(K_\pm)^i_j(\Phi)D_\pm\Phi^j,
\end{align}
where $D_\pm =\frac{\p}{\p\theta^\pm}-\theta^\pm\p_\pm$. In \cite{HullTwistedSUSY}, it was shown that in order for \eqref{q1} and \eqref{q2} to generate the superalgebra \eqref{eq:(2,2)para}, the bosonic part of $(K_\pm)^i_j(\Phi)$, $(K_\pm)^i_j(\phi)$, must be a tangent bundle endomorphism on the target manifold such that $K^2=\id$. Moreover, $K_\pm$ must be integrable in the sense that the Nijenhuis tensors $N_{K_\pm}$ \eqref{eq:nijenhuis} vanish and $K_\pm$ are parallel with respect to the following connections $\n^\pm$
\begin{align*}
\n^\pm K_\pm=0, \quad \n^\pm=\lc\pm\frac{1}{2}\eta^{-1}H,
\end{align*}
$\lc$ being the Levi-Civita connection of $\eta$. Additionally, the requirement that \eqref{q2} is a symmetry of the action \eqref{eq:(1,1)action} imposes the following algebraic relations between $K,\eta$ and $b$:
\begin{align*}
g(T_\pm\cdot,\cdot)+g(\cdot,T_\pm\cdot)&=0\\
b(T_\pm\cdot,\cdot)+b(\cdot,T_\pm\cdot)&=0.
\end{align*}

We conclude that $(\eta,K_\pm,b)$, where $\rd b=H$ is a local potential for the $3$-form flux $H\in\Omega_{cl}^3$, is a {\bf bi-para-Hermitian} structure. Equivalently, this is the defining data of a {\bf generalized para-K\"ahler structure} $(\KK_\pm)$ on the Courant algebroid $(\TT)$ with a flux $H$.



\paragraph{Action on superfields} The infinitesimal action of the element $Q^1_{\pm} \in \mathfrak{t}_{(2,2)}$ is:
\begin{align}\label{Q1_phi}
\Phi^i \mapsto \Phi^i + \epsilon^{\pm}_1 \left( \frac{\p}{\p\theta^\pm}+\theta^\pm\p_\pm \right) \Phi^i
\end{align}
where $\epsilon^{\pm}_1$ is an infinitesimal odd parameter. Similarly, the infinitesimal action by the element $Q^2_{\pm} \in \mathfrak{t}_{(2,2)}$ is:

\begin{align}\label{Q2_phi}
 \Phi \mapsto \Phi + \epsilon^{\pm}_2 (K_\pm)^i_j(\Phi)D_\pm\Phi^j
\end{align}
where $\epsilon^\pm_2$ is again an infinitesimal parameter. Expanding \eqref{Q1_phi}, we get
\begin{align}\label{Q1_action}
\begin{aligned}
Q^1_{+} : (\phi^i, \psi^i_+, \psi^i_-, F^i) &\mapsto (\phi^i, \psi^i_+,\psi^i_-, F^i) + \epsilon^+_1 (\psi^i_+, -\partial_+ \phi, -F^i, \partial_+ \psi^i_-),\\
Q^1_{-} : (\phi^i, \psi^i_+, \psi^i_-, F^i)  &\mapsto (\phi^i, \psi^i_+,\psi^i_-, F^i) + \epsilon^-_1 (\psi^i_-, -\partial^i_- \phi^i, + F^i, - \partial_- \psi^i_+). 
\end{aligned}
\end{align}
Similarly, expanding \eqref{Q2_phi} gives \textcolor{red}{variations of $F$ are still wrong, need to be finished}
\begin{align}\label{Q2_action}
\begin{aligned}
Q^2_{+} : (\phi^i, \psi^i_+, \psi^i_-, F^i) \mapsto &\  (\phi^i, \psi^i_+, \psi^i_-, F^i)
+\epsilon_2^+\Big((K_+)^i_j \psi_+ ^j\ ,\  (K_+)^{i}_{j} \partial_+ \phi^j\\
& -\partial_k(K_+)^i_j \psi_+^k \psi_+^j\ ,\ -(K_+)^i_j F^j{-}\p_k(K_+)^i_j\psi^k_-\psi^j_+\ ,\ \\
&-(K_+)^i_j\p_+\psi_-^j+\p_k(K_+)^i_j\psi^k_+F^j+\p_k(K_+)^i_j\psi^k_-\p_+\phi^j\Big)\\
Q^2_{-} : (\phi^i, \psi^i_+, \psi^i_-, F^i) \mapsto &\ (\phi^i, \psi^i_+, \psi^i_-, F^i)
+\epsilon_2^-\Big((K_-)^i_j \psi_- ^j\ ,\ - (K_-)^i_j F^j  \\
&-\partial_k(K_-)^i_j \psi_+^k \psi_-^j\ ,\ -\p_k(K_-)^i_j\psi^k_-\psi^j_-+(K_-)^{i}_{j} \partial_- \phi^j\ ,\  \\
&(K_-)^i_j\p_-\psi_+^j+\p_k(K_-)^i_j\psi^k_-F^j-\p_k(K_-)^i_j\psi^k_+\p_-\phi^j\Big).
\end{aligned}
\end{align}
%
%\begin{lemma}
%Expanding out in the components of the $(1,1)$ superfield, the infinitesimal action by $Q^1_{+}$ and in Equation (\ref{q1}) reads:
%\[
%Q^1_{+} : (\phi^i, \psi^i_+, \psi^i_-, F^i) \mapsto (\phi^i, \psi^i_{\pm}, F^i) + \epsilon^+_1 (\psi^i_+, \partial_+ \phi, F^i, \partial_+ \psi^i_-) . 
%\]
%and by $Q^1_-$:
%\[
%Q^1_{-} : (\phi^i, \psi^i_+, \psi^i_-, F^i)  \mapsto (\phi^i, \psi^i_{\pm}, F^i) + \epsilon^-_1 (\psi^i_-, \partial^i_- \phi^i, - F^i, - \partial_- \psi^i_+)
%\]
%Similarly, the infinitesimal action by $Q^2_{+}$ in Equation (\ref{q2}) reads:
%\begin{equation}\label{q2plus}
%Q^2_{+} : (\phi^i, \psi^i_+, \psi^i_-, F^i) \mapsto ((K_+)^i_j \psi_+ ^j , - (K_+)^{i}_{j} \partial_+ \phi^j - \partial_k(K_+)^i_j \psi_+^k \psi_+^k, (K_+)^i_j F^j+\p_k(K_+)^i_j\psi^k_-\psi^j_+, \brian{F term})
%\end{equation}
%and for $Q^2_-$:
%\begin{equation}\label{q2minus}
%Q^2_{-} : (\phi^i, \psi^i_+, \psi^i_-, F^i) \mapsto ((K_-)^i_j \psi_- ^j , - (K_-)^{i}_{j} \partial_- \phi^j - \partial_k(K_-)^i_j \psi_-^k \psi_-^k, (K_-)^i_j F^j+\p_k(K_-)^i_j\psi^k_+\psi^j_-, \brian{F term})
%\end{equation}
%\end{lemma}
%
%%\begin{align}\label{Q12_action}
%%\begin{aligned}
%%[Q_\pm^1 ,\phi^i]&=\psi_\pm^i,& [Q_\pm^2 ,\phi^i] &=(K_\pm)^i_j\psi_\pm^j,\\
%%[Q_\pm^1 ,\psi^i_\pm] &=\p_\pm\phi^i,& [Q_\pm^2 ,\psi^i_\pm] &=-(K_\pm)^i_j\p_+\phi^j{\textcolor{red}{+}}\p_k(K_\pm)^i_j\psi_+^k\psi_+^j\\
%%[Q_\pm^1 ,\psi^i_\mp]&=\pm F^i,& [Q_\pm^2 ,\psi^i_\mp] &=(K_\pm)^i_jF^j+\p_k(K_\pm)^i_j\psi^k_\mp\psi^j_\pm.
%%\end{aligned}
%%\end{align}
%%
%%\textcolor{red}{There's a sign issue above, the $+$ needs to be $-$.}
%\begin{proof}
%Recall, in components the superfield is of the form
%\[
%\Phi^i =\phi^i+\theta^+\psi^i_++\theta^-\psi^i_-+\theta^+\theta^-F^i .
%\]
%The action of the $Q^1_\pm$ charges is straightforward and can be simply read off from the infinitesimal action. 
%%\begin{align*}
%%[Q^1_\pm,\Phi]=[Q_\pm^1,\phi^i]+[Q_\pm^1,\psi^i_\pm]\theta^\pm+[Q_\pm^1,\psi^i_\mp]\theta^\mp+[Q_\pm^1,F]\theta^\pm\theta^\mp.
%%\end{align*}
%%For instance $[Q_{\pm}^1, \phi^i] = \theta^{\pm} \partial_{\pm} \phi^i$, thus 
%To obtain the action by the charges $Q^2_\pm$ on the individual components we first need to expand $K_\pm(\Phi)$ in terms of the components of the superfield:
%\begin{align*}
%(K_\pm)^i_j(\Phi)=(K_\pm)^i_j(\phi)+\p_k(K_\pm)^i_j\theta^+\psi_+^k+\p_k(K_\pm)^i_j\theta^-\psi_-^k+\p_k(K_\pm)\theta^+\theta^-F^k,
%\end{align*}
%so that the infinitesimal action by $Q^2_\pm$ on $\Phi^i$ is:
%\begin{align}
%\Phi^i + \epsilon_2^\pm (K_\pm)^i_j(\Phi)(\psi^j_\pm\pm\theta^\mp F^j-\theta^\pm\p_\pm\phi^j\mp\theta^+\theta^-\p_\pm\psi^j_\mp) .
%\end{align}
%Reading off the various components this expression yields Equations (\ref{q2plus}) and (\ref{q2minus}).
%\end{proof}


%
%\paragraph{The $(2,2)$ formalism}
%\brian{
%There is the following superspace model for $(2,2)$ parasupersymmetry. 
%Consider the superspace $\RR^{1,1 | 4}$ 
%}

\subsection{$(2,1)$ and $(2,0)$ parasupersymmetry}
Many of the constructions above make sense when there is less total supersymmetry, and we briefly mention two important cases. The first concerns the $(2,1)$ parasupersymmetry algebra, see Remark \ref{rmk: generalsusy}. 
This algebra $\mathfrak{t}_{(2,1)}$ consists of generators $\{Q^1_\pm, Q^2_+\}$ satisfying the same relations as in (\ref{eq:(2,2)para}). 
Starting with the $(1,1)$ supersymmetric $\sigma$-model (\ref{eq:(1,1)action}), and labeling the odd coordinates by $\theta^\pm = \theta_1^\pm$, one can ask for an additional odd symmetry 
\[
Q^2_+ :  \Phi \mapsto \Phi + \epsilon_2^+ K_+(\Phi)^j_i D_+ \Phi^i .
\]
The odd vector field $Q_2^+$ is a symmetry of the $(1,1)$ supersymmetric $\sigma$-model provided the target has the structure of a para SKT manifold, see Section \ref{SKT}. 
Thus, the $(1,1)$ supersymmetric $\sigma$-model has $(2,1)$ parasupersymmetry provided the target has the structure of a para SKT manifold.

Similarly, by forgetting part of the supersymmetry, this means that the $(1,0)$ supersymmetric $\sigma$-model has $(2,0)$ parasupersymmetry provided the target is a para SKT manifold. 

\subsection{Topological twists}\label{sec:toptwist}

\btd{Can you outline what is meant here by twisting? We aren't going to use the full language of twisting as in Kevin's paper, but it'd be good to highlight two things: (1) how to use the $R$-symmetry to twist the transformation properties of the fields and (2) how the supercharge deforms the BRST differential.
I think you already do both below, but it'd be good to add some more discussion. 
For instance, in accomplishing (1) we should say concretely which bundles each of the fields live in.}

In this section we show that the R-symmetry of the para-$(2,2)$ superalgebra can be used to produce topological theories from the $(2,2)$ para-supersymmetric model presented in the Section \ref{sec:(2,2)parasusy} via a procedure called topological twisting, exactly analogous to topological twisting in ordinary supersymmetry. We show that the theories we obtain match the topological theories constructed in the Section \ref{sec:GpC_AKSZ} in the AKSZ formalism. 

In our presentation, we use the similarities between the generalized K\"{a}hler (GK) and generalized para-K\"{a}hler (GpK) geometries, and we follow the approach presented in \cite{Kapustin:2004gv} for the GK case. Let us briefly review the construction now.

Let $(\eta,K_\pm,H=\rd b)$ be the bi-para-Hermitian geometry of the $(2,2)$ para-SUSY sigma model and let $\{Q^1_\pm,Q^1_\pm\}$ be the basis of the superalgebra \eqref{eq:(2,2)para}. Our goal is to use the data of the $(2,2)$ para-SUSY theory to define a new, {\it topological} theory. In this section, we will show how to define the BRST complex for this new theory, which in particular requires the definition of a nilpotent BRST operator $\QQ^2=0$. Additionally, we want all fields in this complex to be scalars under the Lorentz group. For this, we must alter the original Lorentz group $SO(1,1)$ by its embedding in the R-symmetry group, $SO(1,1)_{G_R}\hookrightarrow G_R=SO(1,1)_+\times SO(1,1)_-$, so that the new Lorentz group action is given by the diagonal embedding of $SO(1,1)$ into $SO(1,1)\times SO(1,1)_{G_R}$
\begin{align}\label{embeddings}
SO(1,1)\overset{diag.}{\hookrightarrow} SO(1,1)\!\times\! SO(1,1)_{G_R}\!\!\hookrightarrow SO(1,1)\!\times\! \overbrace{SO(1,1)_+\!\times\! SO(1,1)_-}^{G_R},
\end{align}
defining a new Lorentz action, which will be a sum of the original one and the action of $SO(1,1)_{G_R}$ in $G_R$. In this sense, the Lorentz group gets {\it twisted} by the embedding into the R-symmetry group. Finally, the operator $\QQ$ must be chosen accordingly to the choice of $SO(1,1)_{G_R}\hookrightarrow G_R$ so that it transforms as a scalar itself under the new Lorentz action, which ensures that the fields in the BRST complex it defines are all scalars too.

%\david{how does the embedding of $SO(1,1)$ into $G_R$ reflect on the definition of $\QQ$?}
%\brian{In Euclidean space, we choose an embedding $SO(2) \hookrightarrow G_R$. 
%Then, the twisting supercharge is required to be invariant.
%The obvious analog is to require an embedding $SO(1,1) \to G_R$ for which $\QQ$ is invariant.
%This is the twisting homomorphism which tells us how to modify the Lorentz group.
%}
%
%\brian{
%There is another piece of data we have to fix to get the $\mathbb{Z}$-grading in the twisted theory. 
%Ordinarily this amounts to picking a homomorphism $U(1) \to G_R$ for which $\QQ$ is weight one. 
%It appears no such twisting homomorphism is possible.
%Indeed, every group homomorphism $U(1) \to SO(1,1) \times SO(1,1)$  are necessarily trivial since $U(1)$ is compact and $SO(1,1) \cong \RR \oplus \RR$ (the first copy is the connected component of the identity, and the second component is the connected component of minus identity).
%This means we may have to slightly reimagine what ``twisting data" is here.
%Since there is at least a copy of $\RR$ in $SO(1,1)$ we can probably use this to define a ``weight".}
\bigskip
Let us start with the second, twisting, embedding in \eqref{embeddings}
\begin{align*}
\rho:\ SO(1,1)_{G_R}\hookrightarrow G_R=SO(1,1)_+\times SO(1,1)_-.
\end{align*}
Clearly, there are two inequivalent choices -- the diagonal and anti-diagonal -- and we shall label them as $SO(1,1)_V$ and $SO(1,1)_A$, respectivelly, and  calling the twist by $SO(1,1)_{V/A}$ the {\bf generalized para-A/B-twist}. Explicitly, if we represent $M_\pm\in SO(1,1)$ in \eqref{r_symmetry_matrix} as
\begin{align*}
 M_\pm(\alpha)\begin{pmatrix}
 Q^1_\pm\\
 Q^2_\pm
 \end{pmatrix}=\begin{pmatrix}
 \cosh(\alpha) & \sinh(\alpha) \\
 \sinh(\alpha) & \cosh(\alpha)
 \end{pmatrix}
 \begin{pmatrix}
 Q^1_\pm\\
 Q^2_\pm
 \end{pmatrix},
 \end{align*}
then the diagonal embedding is given by $M(\alpha) \mapsto (M(\alpha),M(\alpha))$ while the anti-diagonal is $M(\alpha) \mapsto (M(\alpha),M(-\alpha))$.

To determine the corresponding nilpotent operators $\QQ_{V/A}$ on which $V$ and $A$ act trivially, we find that the following combinations of the supercharges
\begin{align*}
Q_{L\pm}=\frac{1}{2}(Q^1_+\pm Q^2_+),\quad Q_{R\pm}=\frac{1}{2}(Q^1_-\pm Q^2_-),
\end{align*}
are all nilpotent and satisfy $[Q_{L\pm},Q_{L\pm}]=0$, so that their sums and differences will also be nilpotent. However, they are charged under the Lorentz group. Their Lorentz charges along with their charges under $SO(1,1)_{V/A}$ are summarized in the following table
\bigskip
\begin{center}
\begin{tabular}{c|c c c}
		& $SO(1,1)_{Lor.}$ & $SO(1,1)_V$ & $SO(1,1)_A$\\ \hline
$Q_{L+}$& $-1$ & $+1$ & $+1$\\
$Q_{L-}$& $-1$ & $-1$ & $-1$\\
$Q_{R+}$& $+1$ & $+1$ & $-1$\\
$Q_{R-}$& $+1$ & $-1$ & $+1$
\end{tabular}.
\end{center}
\bigskip
From the above table, we can read off that the combination $\QQ_V=Q_{L+}+Q_{R-}$ will transform as a scalar under the new Lorentz group twisted by $U(1,1)_V$ and $\QQ_A=Q_{L+}+Q_{R+}$ will transform as a scalar under the new Lorentz group twisted by $U(1,1)_A$. Note also that the two are related by $Q_-^2\mapsto -Q_-^2$ and because $Q_-^2=K_-D_-$, this is acheived by changing the sign of $K_-$. Therefore, in the following we will study without loss of generality the twist given by $U(1,1)_A$ and $\QQ_B$, i.e. the generalized para-B-model:

\begin{Def}\label{def:B-twist}
The generalized para-B-model is a topological twist (in the above sense) of a $(2,2)$ para-SUSY sigma model given by the diagonal embedding of the Lorentz group $SO(1,1)\hookrightarrow SO(1,1)\times SO(1,1)_A$, where $SO(1,1)_A$ denotes the anti-diagonal embedding of $SO(1,1)$ into the R-symmetry group $G_R=SO(1,1)\times SO(1,1)$, and the nilpotent scalar BRST operator
\begin{align}\label{QQ_nilpotent}
\QQ=Q_L+Q_R,\quad Q_L=\frac{1}{2}(Q^1_++Q^2_+),\quad Q_R=\frac{1}{2}(Q^1_-+Q^2_-).
\end{align}
\end{Def}
\bigskip
We now turn to the BRST complex that describes the scalar fields in the twisted theory. For this, we introduce the usual notation 
\begin{align*}
\chi=\frac{1}{2}(\id+K_+)\psi_+=P_+\psi_+=\psi_+^{(1,0)_+},\quad\lambda=\frac{1}{2}(\id+K_-)\psi_-=P_-\psi_-=\psi_-^{(1,0)_-},
\end{align*}
where both $\chi$ and $\lambda$ are scalars under the twisted Lorentz action, which is the content of the following statement:
\begin{proposition}\label{prop_toptwist}
The action of $Q_{L/R}$ on the fields $\chi$, $\lambda$ and $\phi$ is given by
\begin{align}\label{eq:Qcoh}
\begin{aligned}
[Q_L,\phi^i]&=\chi^i & [Q_R,\phi^i]&=\lambda^i\\
[Q_L,\chi^i]&=0 & [Q_R,\lambda^i]&=0\\
[Q_L,\lambda^i]&=-(\Gamma^-)^i_{jk}\chi^j\lambda^k & [Q_R,\chi^i]&=-(\Gamma^+)^i_{jk}\lambda^j\chi^k.
\end{aligned}
\end{align}
\end{proposition}
\begin{proof}
See Appendix \ref{appendix:proof_prop}.
\end{proof}
Recalling that $(\Gamma^+)^i_{jk}=(\Gamma^-)^i_{kj}$ (see \eqref{Gamma-pm}), we find that the BRST degrees of the fields are given by the following complex
\begin{align*}
&\phi \overset{\QQ}{\longrightarrow}\chi+\lambda\overset{\QQ}{\longrightarrow} 0\\
&0 \phantom{\longrightarrow\longrightarrow} 1
\end{align*}
\noindent
The observables of the twisted theory are therefore polynomials in the odd variables $\chi,\lambda$ with even coefficients dependent on $\phi$:
\begin{align*}
{\cal O}_f=f(\phi)_{i_1\cdots i_a,j_1\cdots j_b}\chi^{i_1}\cdots\chi^{i_a}\lambda^{j_1}\cdots\lambda^{j_b}.
\end{align*}

Because $\chi \in \XX^{(1,0)_+}$ and $\lambda \in \XX^{(1,0)_-}$, we can identify ${\cal O}_f$ with a function on the supermanifold $(T^{(1,0)_+}\oplus T^{(1,0)_-})[1]$, which in turn can be identified with the mixed differential form
\begin{align*}
\Omega_f=f(\phi)_{i_1\cdots i_a,j_1\cdots j_b}dx^{i_1}_+\cdots dx^{i_a}_+ dx^{j_1}_- \cdots dx^{j_b}_-,
\end{align*}
where $dx_\pm$ are the one-forms satisfying $K_\pm dx_\pm=dx_\pm$, i.e. sections of the bundle $T^*_{(1,0)_\pm}$. Via this identification, the operator ${\cal Q}\coloneqq Q_L+Q_R$ defines an operator
\begin{align*}
\rd_\QQ:\ \Lambda^\bullet(T^{*(1,0)_+}\oplus T^{*(1,0)_-})\rightarrow \Lambda^{\bullet +1}(T^{*(1,0)_+}\oplus T^{*(1,0)_-})
\end{align*}
Since $\QQ$ squares to zero the operator $\rd_\QQ$ also squares to zero, hence it defines a differential. 
In fact, we will see that $\rd_\QQ$ gives rise to a Chevalley-Eilenberg complex $\text{CE}(L_{twist})$ computing the Lie algebroid cohomology of a certain Lie algebroid.

\paragraph{Lie algebroid from bi-para-Hermitian data} 
Consider the Lie algebroid $L_{twist}\overset{a}{\rightarrow} T$ with $L_{twist}=T^{(1,0)_+}\oplus T^{(1,0)_-}$ and anchor given by the sum of the two inclusions $\imath_\pm:\ T^{(1,0)_\pm}\hookrightarrow T$:
\begin{align*}
a:L_{twist}=T^{(1,0)_+}\oplus T^{(1,0)_-} &\rightarrow T\\
 e=(x_+,x_-) &\mapsto \imath_+(x_+)+\imath_-(x_-).
\end{align*}
From \eqref{eq:Qcoh} we can read off the action of $\rd_\QQ$ on degree one elements, i.e. sections $\alpha=(\alpha^+,\alpha^-)$ of $L^*_{twist}=T^*_{(1,0)_+}\oplus T^*_{(1,0)_-}$:
\begin{align*}
\rd_\QQ\alpha&=\p_k\ap_i^+[dx_+^k+dx_-^k]\w dx_+^i-\ap^+_i(\Gamma^+)^i_{jk}dx_-^j\w dx_+^k\\
&+\p_k\ap_i^-[dx_+^k+dx_-^k]\w dx_-^i-\ap^-_i(\Gamma^-)^i_{jk}dx_+^j\w dx_-^k.
\end{align*}
%In coordinate-free form, we get
%\begin{align*}
%\rd_\QQ\ap=(\p_++\n^+_{P_-(\bullet)})\ap^++(\p_-+\n^-_{P_+(\bullet)})\ap^-,
%\end{align*}
By contracting in two sections of $L_{twist}$, $e_1=x_++x_-$, $e_2=y_-+y_-$, we get the coordinate-free expression
\begin{align}\label{eq:dQ}
\begin{aligned}
(\rd_\QQ\ap)(e_1,e_2)&=(\p_+\ap^+)(x_+,y_+)+\n^+_{x_-}\ap^+(y_+)-\n^+_{y_-}\ap^+(x_+)\\
&+(\p_-\ap^-)(x_-,y_-)+\n^-_{x_+}\ap^-(y_-)-\n^-_{y_+}\ap^-(x_-),
\end{aligned}
\end{align}
where $\p_\pm$ denotes the para-complex Dolbeault operators \eqref{paracomplex_dels} of the para-complex structure $K_\pm$. We can further rewrite this expression by writing $\ap=(\eta(\tl{z}_+),\eta(\tl{z}_-))$ for $\tl{z}_\pm \in \se(T^{(1,0)_\pm})$, invoking the formula $\p_\pm\ap^\pm(x_\pm,y_\pm)=\lc_{x_\pm}\ap^\pm(y_\pm)-\lc_{y_\pm}\ap^\pm(x_\pm)$ and recalling $\n^\pm=\lc\pm \frac{1}{2}\eta^{-1}H$:
\begin{align}
\begin{aligned}
(\rd_\QQ\ap)(e_1,e_2)&=\eta((\lc_{x_+}+\n_{x_-}^+)\zt_+,y_+)-\eta((\lc_{y_+}+\n_{y_-}^+)\zt_+,x_+)\\
&+\eta((\lc_{x_-}+\n_{x_+}^-)\zt_-,y_-)-\eta((\lc_{y_-}+\n_{y_+}^-)\zt_-,x_-)\\
&=\eta(\n^+_{x_++x_-}\zt_+,y_+)-\eta(\n_{y_++y_-}^+\zt_+,x_+)-H(x_+,\zt_+,y_+)\\
&+\eta(\n_{x_++x_-}^-\zt_-,y_-)-\eta(\n_{y_++y_-}^-\zt_-,x_-)+H(x_-,\zt_-,y_-).
\end{aligned}
\end{align}
This fully determines the Lie algebroid structure of $L_{twist}$. Therefore, the topological twist gives rise to the Lie algebroid we just described explicitly using only the data of the bi-para-Hermitian geometry $(\eta,K_\pm,H)$. In the following paragraph, we will discuss how the same Lie algebroid arises from the data of the corresponding generalized para-K\"ahler structure.

\paragraph{Lie algebroid from generalized para-K\"ahler data}
%
%\brian{Do we use the full GpK structure, or are we only using the GpC?}
%\btd{The title of this section is OK, but we should stress that the Lie algebroid {\em explicitly} does not depend on any Kahler data.}
%\btd{I suggest renaming $\Lb_+$ to $L_{AKSZ}$, or something else to denote the fact that is the Lie algebroid coming from the AKSZ description in Section \ref{sec: AKSZ}.}
We will now show that the Lie algberoid $(L_{twist},\rd_\QQ,a)$ is isomorphic to one defined by the $+1$-eigenbundle of the generalized para-complex structure $\KK_+$ of the generalized para-K\"ahler structure $(\KK_+,\KK_-)$ underlying the $(2,2)$ para-SUSY theory. Therefore, $(L_{twist},\rd_\QQ,a)$ coincides with the Lie algebroid used in Section \ref{sec: AKSZ} to define a topological theory in the AKSZ formalism from an arbitrary generalized para-complex structure $\KK$. Consequently, we find that the theories constructed in Section \ref{sec: AKSZ} labeled by a GpC structure $\KK$ can be realized as topological twists of a $2D$ $(2,2)$ para-SUSY $\sigma$-models whenever $\KK$ is a part of a GpK data $(\KK_+=\KK,\KK_-)$.

Recall that a GpK structure defines the decomposition \eqref{GpK_bundles}
\begin{align*}
\TT=\ell_+\oplus\ell_-\oplus \ellt_+\oplus \ellt_-,
\end{align*}
where the the eigenbundles of $\KK_+$ are $\Lb_+=\ell_+\oplus\ell_-$ and $\wtl{\Lb}_+=\ellt_+\oplus \ellt_-$. Because $\Lb$ is a Dirac structure, it defines a Lie algebroid by restriction of the Courant algebroid on $\TT$ to $\Lb_+$, i.e. the Lie algebroid bracket and anchor are given by the restriction of the Dorfman bracket and the projection onto tangent bundle, respectively (see Proposition \ref{prop:dirac_Liealg}). We now show that such Lie algebroid is isomorphic to the Lie algebroid defined by $(L_{twist},\rd_\QQ,a)$ above:
\begin{theorem}
The Lie algebroids $(L,a,\rd_\QQ)$ and $(\Lb_+,\pi_T\!\!\mid_{\Lb_+},\brac\!\!\mid_{\Lb_+})$ are isomorphic via the map $\pi=\pi_+\oplus \pi_-$.
\end{theorem}
\begin{proof}
First, we notice that the bundles themselves are related by $\pi$,  $\pi(\Lb_+)=\pi(\ell_+\oplus \ell_-)=\pi_+(\ell_+)\oplus \pi_-(\ell_-)=T^{(1,0)_+}\oplus T^{(1,0)_-}=L$ and also $\pi$ is clearly an isomorphism since both $\pi_\pm$ are. Additionally, the anchors are related by $a\circ\pi=\pi_T\!\!\mid_{\Lb_+}$. The only non-trivial task is therefore to show that the respective differentials satisfy
\begin{align}\label{eq:d_pi}
\pi^*\circ\rd_Q  =\rd_{\Lb_+}\circ \pi^*,
\end{align}
where $\pi^*=\pi^*_+\oplus\pi_-^*$ is given by the dual maps to $\pi_\pm:\ell_+\rightarrow T^{(1,0)_\pm}$, extended to a map $\pi^*:\Lambda^k(L^*)\rightarrow\Lambda^k(\Lb_+^*)$.

Because of the chain rule, we only need to check \eqref{eq:d_pi} on degree $0$ and degree $1$ elements in the respective complexes. Degree $0$ follows immediately:
\begin{align*}
(\rd_Q f)(\pi(u))= a\circ \pi(u)[f]=\pi_T\!\!\mid_{\Lb_+}(u)[f]=(\rd_{\Lb_+}f)(u),
\end{align*}
%For degree one, we use the decomposition of the respective bundles and check for each component
%\begin{alignat*}{2}
%\rd_Q:&&T^*_{(1,0)_+}\oplus T^*_{(1,0)_-} &\rightarrow \Lambda^2(T^*_{(1,0)_+}\oplus T^*_{(1,0)_-})\\
%\rd_{\Lb_+}:&& \ell_+^*\oplus \ell_-^* &\rightarrow \Lambda^2(\ell_+^*\oplus \ell_-^*)
%\end{alignat*}
and we will now prove \ref{eq:d_pi} for degree $1$ elements. In the following we will make use of the identifications provided by the metric $\eta$ and pairing $\lara$:
\begin{alignat*}{2}
\eta:&&\  T^*_{(1,0)_\pm} &\leftrightarrow T^{(0,1)_\pm}\\
\lara:&& \ell^*_\pm &\leftrightarrow \ellt_\pm,
\end{alignat*}
so that elements in $T^*_{(1,0)_\pm}$ can be written as $\eta(\tl{z}_\pm)$ with $\tl{z}_\pm$ vector in $T^{(0,1)_\pm}$ and similarly, elements in $\ell^*_\pm$ can be expressed as $\la \tl{w}_\pm, \cdot\ra$ with $\tl{w}_\pm$ section of $\ellt_\pm$. To declutter notation, we shall also denote the bracket and anchor on $\Lb_+$ by $\brac$ and $\pi$, respectively.

The differential $\rd_{\Lb_+}$ is defined by the bracket on the Lie algebroid $\Lb_+$ via the following formula:
\begin{align}\label{eq:proof_Liethm}
\rd_{\Lb_+}  w^* (u,v)=\pi(u) w^*(v)-\pi(u) w^*(v)- w^*([u,v]),
\end{align}
where $ w^* \in \se(\Lb_+^*)$ and $u,v\in \se(\Lb_+)$. Writing $ w^*$ as $\la \tl{w},\cdot\ra$, with $\tl{w}\in\se (\tl{\Lb}_+)$ we can rewrite this as
\begin{align}\label{dL}
\begin{aligned}
\rd_{\Lb_+}  w^* (u,v)&=\pi(u)\la v ,\tl{w}\ra-\pi(v)\la u,\tl{w}\ra-\la [u,v],\tl{w}\ra\\
&=\la D_uv,\tl{w}\ra+\la v,D_u \tl{w}\ra-\la D_v u,\tl{w}\ra-\la u,D_v\tl{w}\ra-\la D_uv-D_vu,\tl{w}\ra\\
&+{T}(u,v,\tl{w})\\
&=\la D_u \tl{w},v\ra-\la D_v\tl{w},u\ra+{T}(u,v,\tl{w})\\
&=\la D_u \tl{w}_+,v_+\ra-\la D_v\tl{w}_+,u_+\ra+{T}(u_+,v_+,\tl{w}_+)\\
&+\la D_u \tl{w}_-,v_-\ra-\la D_v\tl{w}_-,u_-\ra+{T}(u_-,v_-,\tl{w}_-),
\end{aligned}
\end{align}
where we used \eqref{gentorsion_def} to express $\brac$ in terms of $D$, the generalized Bismut connection of the generalized metric for the GpK structure. We then used the fact that $D$ preserves $\lara$, $C_\pm$ as well as $\Lb_+$ and that $T$ only has components in $\Lambda^3 C_+\oplus\Lambda^3 C_-$ \cite{Gualtieri:2010fd}.

We will now show that $\rd_{\Lb_+}$ is related to $\rd_\QQ$ via \eqref{eq:d_pi}, so that $\rd_\QQ=(\pi^{-1})^*\circ \rd_{\Lb_+}\circ \pi^*$, i.e.
\begin{align*}
\rd_\QQ\ap(e_1,e_2)=\rd_{\Lb_+}(\pi^*\ap)(\pi^{-1}e_1,\pi^{-1}e_2).
\end{align*}
To read this off from \eqref{dL} and compare with \eqref{eq:dQ}, we need to express $\pi^*\ap$ as $\pi^*\ap=w^*=\la \tl{w},\cdot\ra$ for some $\tl{w}\in\se (\tl{\Lb}_+)$ and $\ap=(\eta(\zt_+),\eta(\zt_-))$:
\begin{align*}
\pi^*\ap=\pi^*(\eta(\zt_+),\eta(\zt_-))=(\eta(\zt_+,\pi_+\cdot),\eta(\zt_-,\pi_-\cdot))=\frac{1}{2}\la\pi^{-1}(\zt_+,-\zt_-),\cdot\ra,
\end{align*}
where $\zt_\pm\in \se(T^{(1,0)_\pm})$ and we used \eqref{pi:gG_relationship}. Therefore, denoting $e_1=(x_+,x_-)$ and $e_2=(y_+,y_-)$ as in \eqref{eq:dQ}, \eqref{dL} yields (denoting $e_1=(x_+,x_-)$ and $e_2=(y_+,y_-)$ as in \eqref{eq:dQ})
\begin{align*}
\rd_{\Lb_+}(&\pi^*\ap)(\pi^{-1}e_1,\pi^{-1}e_2)\\
&=\eta(\n^+_{x_++x_-}\zt_+,y_+)-\eta(\n^+_{y_++y_-}\zt_+,x_+)+\frac{1}{2}T(\pi_+^{-1}x_+,\pi_+^{-1}y_+,\pi_+^{-1}\zt_+)\\
&+\eta(\n^-_{x_++x_-}\zt_-,y_-)-\eta(\n^-_{y_++y_-}\zt_-,x_-)-\frac{1}{2}T(\pi_-^{-1}x_-,\pi_-^{-1}y_-,\pi_-^{-1}\zt_-),
\end{align*}
where we used \eqref{genBismut_pi_nabla_pm} and once again \eqref{pi:gG_relationship}. Finally, invoking \eqref{gentorsion_H}, we arrive at \eqref{eq:dQ}, showing that \eqref{eq:d_pi} holds and completing the proof.
\end{proof}

We have therefore shown that the observables in the topologically twisted theories are given by the Lie algebroid cohomology of $\Lb$, which is exactly the same as for the $2D$ boundary topological theories constructed in the Section \ref{sec: AKSZ}:

\begin{corollary}
Let $\Phi:\Sigma^{2|2}\rightarrow \Mb$ be a $(2,2)$ para-supersymmetric $\sigma$-model with the target $(\Mb,\KK_\pm,\GG(\eta,b))$ a Generalized para-K\"ahler manifold. Then the generalozed para-B-twist of this $\sigma$-model defined in Definition \ref{def:B-twist} coincides with the $2D$ topological boundary theory determined by the generalized para-complex structure on the Courant algebroid $(\TT)\Mb$ with flux $H\overset{loc.}{=}\rd b$.
\end{corollary}

\section{Para-complex semi-flat mirror symmetry}
In this section we present our first observations about a possible notion of mirror symmetry for para-complex geometry. We show that the initial observations about $2D$ $(2,2)$ SUSY $\sigma$-models that initiated the vast research program of mirror symmetry apply perfectly well to $(2,2)$ para-SUSY models and therefore give rise to parallel statements in the para-holomorphic category.

The original example of mirror symmetry in physics is based on the observation that the $(2,2)$ SUSY algebra (in its complex form) exhibits a $\mathbb{Z}_2$ outer endomorphism, which exchanges the vector and axial $U(1)$ generators and in particular then exchanges the two topological twists \cite{Witten_mirror_topological}. On the level of the supercharges \eqref{complex_(2,2)}, this is manifested by the exchange (see for example \cite{susy_hori} for details)
\begin{align*}
Q_-\longleftrightarrow \overline{Q}_-,
\end{align*}
which clearly preserves the commutation relations. In our case, the $\mathbb{Z}_2$ symmetry acts on \eqref{parasusy_offdiagonal_basis} as
\begin{align*}
Q_-\longleftrightarrow \widetilde{Q}_-,
\end{align*}
or, equvalently on the diagonal basis \eqref{eq:(2,2)para} $Q^2_-\mapsto -Q^2_-$, as discussed in Section \ref{sec:toptwist}, where we also argued that this exchanges the two types of topological twists.

The geometric realization of the above-described $\mathbb{Z}_2$ action on the $(2,2)$ superalgebra was described in \cite{SYZ} and since has acquired the name SYZ mirror symmetry. There, the mirror symmetry is described as a correspondence between pairs of Calabi-Yau manifolds $M$ and $\Mt$, specifically ones that admit a particular types of fibrations over the same base $B$. The mirror symmetry in this case acts geometrically as an exchange of these fibrations, which are essentially linear duals to each other. One of the hallmarks of mirror symmetry is the feature that it relates the moduli space of symplectic structures on $M$ with a moduli space of complex structures on $\Mt$. On the physics side, this exchange is realized as a correspondence between an $A$-model on $M$ and a $B$-model on $\Mt$.

In the following paragraphs, we will study the semi-flat case of SYZ mirror symmetry, where the pair of K\"ahler Calabi-Yau manifolds is given by $M=TB$ and $\Mt=T^*B$, the tangent and cotangent bundles of an affine manifold $B$. Our result is that this setting admits an equally natural description in terms of para-K\"ahler and para-Calabi-Yau geometry, where the mirror map exchanges the {\bf symplectic moduli} on $M$ with the {\bf para-complex moduli} on $\Mt$. The underlying symplectic fundamental forms of both the K\"ahler and para-K\"ahler gometries coincide, while the complex and para-complex structures anti-commute and therefore the triple $(\omega,K,I)$ defines {\it Born geometry} \cite{freidel2014born,Freidel:2018tkj,mythesis} (see also \cite{Marotta:2018myj,Hassler:2019wvn,Marotta:2018swj,Marotta:2019eqc}). Consequently, the mirror map in the semi-flat case relates Born geometries on $M$ and $\Mt$, mapping between the symplectic moduli on one side and the para-complex and complex moduli on the other side.

In the following we closely follow the discussions presented in \cite{mythesis,mirror_w-o_corrections,hitchin1997moduli}

\paragraph{The para-complex point of view} We start with the discussion of the para-Calabi-Yau geometry on the side of $M$ and then contrast it with the Calabi-Yau geometry of $M$. Consider an affine manifold $B$ and take a neighbourhood $U\subset B$ with local coordinates $u^i$. Let $U\times \RR^n$ with the coordinates $(u^i,v^i)$, $v^i$ denoting the coordinates on $\RR^n$, be the local model for $M=TB$. We define the para-complex structure on $M$ in terms of its adapted coordinates $(x^i=u^i+v^i,\xt_j=u^k-v^k)$. The fact that $M$ is affine means that the transition functions are in $GL_n(\RR)\ltimes \RR^n$, i.e. of the form
\begin{align*}
    u^i\mapsto u'^i(u^i)=A_{\ j}^{i}u^j+B^i,\quad A \in GL_n,\ B\in \RR^n.
\end{align*}
This implies that the natural fiber coordinates of $TM$, $v^i=du^i$,  transform as $u^i\mapsto A^i_{\ j}u^j$, so that the combinations $(x^i=u^i+v^i,\xt^j=u^k-v^k)$ transform by
\begin{align*}
    (x^i,\xt_j)\mapsto (A_{\ k}^{i}x^k+B^i,A_{\ k}^{j}\xt^k+B^j),
\end{align*}
which assures that the total space of $TB$ is an affine manifold with $(x^i,\xt^j)$ affine coordinates. In particular $(x^i,\xt^j)$ are adapted coordinates of an integrable para-complex structure on $TB$, because the transition functions are para-holomorphic.

We now endow $M$ with a para-Calabi-Yau structure defined by the fundamental form $\omega$ and a para-Holomorphic volume form $\Omega$:
\begin{align*}
    \omega&=\omega_{ij}\ dx^i\w d \xt^j=2\omega_{ij}\ du^i\w dv^j,\\
    \hat{\Omega} &=dx^1\w\cdots\w dx^n+d\xt^1\w\cdots\w d\xt^n=\Omega+\tl{\Omega}.
\end{align*}
We also assume that the local para-K\"ahler potential $\phi$ on $U\subset M$, which defines $\omega$ via
\begin{align*}
\omega_{ij}=\p_i\pt^j\phi=\left(\frac{\p^2}{\p u^i \p u^j}-\frac{\p^2}{\p v^i \p v^j}\right)\phi,
\end{align*}
is of the form $\phi(u,v)=\phi(u)$. This means it is invariant under the translations in the fiber directions of $TB$ and the geometry is therefore {\it semi-flat}, hence the name semi-flat mirror symmetry. The fact that the pair $(\omega,\hat{\Omega})$ defines a para-Calabi-Yau structure means that it satisfies the Monge-Ampere equation:
\begin{align*}
    C\omega^n=\Omega\w\tl{\Omega} \Longleftrightarrow   \text{det}(\frac{\p^2\phi}{\p u^i\p u^j})=C.
\end{align*}
The above has a unique solution for $\phi$ \cite{calabi1975construction} assuming $\phi\mid_{\p U}=0$ and $\phi$ is convex ($\p_i\p_j\phi>0$). For the Monge-Ampere equation to be invariant under the affine transformations, we must have $\text{det}(A)=1$, which is a requirement that defines a {\it special affine manifold}. Therefore, we will from now on require that $B$ is special affine. For completeness, we note that the para-K\"ahler metric $\eta$ takes the form
\begin{align*}
    \eta=\omega_{ij} (dx^i\otimes d\xt^j+d\xt^i\otimes dx^j)=2\omega_{ij}(du^i\otimes du^j-dv^i\otimes dv^j).
\end{align*}

\paragraph{The complex point of view}
In the usual discussion of the semi-flat mirror symmetry, one considers a complex structure defined by the holomorphic coordinates $z^i=u^i+iv^i$ instead of the para-complex structure above. The fundamental form is then taken to be $\omega_I=i\p\bar{\p}\phi$, which in the semi-flat case $\phi(u,v)=\phi(u)$ coincides with $\omega$. The Riemannian K\"ahler metric is then
\begin{align*}
    g=\omega_{ij}(du^i\otimes du^j+dv^i\otimes dv^j).
\end{align*}
It is easy to show that $I$ and $K$ anticommute, and because $\omega_I=\omega$, we see that this geometric setting defines Born geometry.
\newcommand{\uh}{\hat{u}}
\newcommand{\vh}{\hat{v}}
\newcommand{\xh}{\hat{x}}
\newcommand{\xth}{\hat{\xt}}

\paragraph{Legendre transform and the mirror map}
Consider new coordinates $\uh_i$ on $U$ given by
\begin{align*}
\frac{\p\uh_i(u)}{\p u^j}=\omega_{ij}=  \frac{\p^2\phi}{\p u^i \p u^j},
\end{align*}
with the inverse transformation $\frac{\p u^i(\uh)}{\p \uh_j}=\omega^{ij}=(\omega_{ij})^{-1}$. Integrating this, we can write the relationship between the coordinates
\begin{align*}
    u^i(\uh)=\frac{\p \psi(\uh)}{\p \uh_i},
\end{align*}
for some local function $\psi(\uh)$ so that $\omega^{ij}=\frac{\p \psi(\uh)}{\p \uh_i\uh_j}$ and $\phi(u)$ and $\psi(\uh)$ are Legendre transforms of one another:
\begin{align*}
    \psi(\uh)=u^i \uh_i-\phi(u).
\end{align*}
This ensures that also $\psi$ satisfies the Monge-Ampere equation, but with an inverse constant
\begin{align}\label{dual-monge-ampere}
    \text{det}(\frac{\p^2\psi}{\p \uh_i\p \uh_j})=C^{-1}.
\end{align}

We continue by discussing the cotangent bundle $\Mt=T^*B$. The metric $\eta$ on $M=TB$ defines a negative-definite metric $-2\omega_{ij}dv^i\otimes dv^j$ on each fiber, which can be inverted to give a negative-definite metric on fibres of $T^*B$ and consequently on the whole $\Mt$:
\begin{align*}
    \hat{\eta}=2(\omega_{ij}du^i\otimes du^j-\omega^{ij}d\vh_i\otimes d\vh_j),
\end{align*}
denoting the fibre coordinates dual to $v^i=dx^i$ by $\vh_i=\frac{\p}{\p u^i}$. Using the canonical symplectic form on $T^*B$,
\begin{align*}
    \hat{\omega}=du^i\w d\vh_i
\end{align*}
we define a para-K\"ahler structure $\hat{K}$ via
\begin{align*}
    \hat{K}\coloneqq \hat{\eta}^{-1}\hat{\omega}.
\end{align*}
It can be checked that $\hat{K}$ is again integrable and the corresponding adapted coordinates are $(\xh_i=\uh_i+\vh_i,\xth_j=\uh_i-\vh_i)$, in terms of which we get
\begin{align*}
    \hat{\omega}&=\hat{\omega}^{ij}d\xh_i\w d\xth_j\\
    \hat{\eta}&=\hat{\omega}^{ij}(d\xh_i\otimes d\xth_j + d\xth_i\otimes d\xh_j),
\end{align*}
where $\hat{\omega}^{ij}=\frac{\p^2 \psi}{\p \uh^i\p \uh^j}$. Crucially, $\Mt$ is also a para-Calabi-Yau manifold because $\psi$ satisfies the Monge-Ampere equation \eqref{dual-monge-ampere}. Moreover, we get a relationship between the moduli of symplectic structures on $M$ and para-complex structures on $\Mt$, since varying the symplectic structure on $M$ corresponds to a change of $\phi$, which also changes $\psi$ as well as the coordinates $\uh_i$. This in turn changes the para-complex structure $\hat{K}$ defined by its adapted coordinates, and in particular depends on $\uh_i$.

We find that in this case the mirror symmetry not only relates the symplectic and complex moduli of the mirror manifolds $M$ and $\Mt$, but also relates the symplectic and para-complex moduli and therefore gives a map of moduli of Born geometries on $M$ and $\Mt$, where the variation of the symplectic structure on one side corresponds to a variation of the complex and para-complex structures on the other.

\paragraph{Speculations about para-mirror symmetry}
Because the semi-flat case is so simple, the mirror pair is the same in both the complex and para-complex cases. However, this does not always have to be the case and the para-complex mirror symmetry could relate pairs of manifolds that are not related by the usual mirror symmetry. Moreover, this new notion of mirror symmetry gives rise to a possibility of {\it mirror trialities}, i.e. situations when the symplectic moduli space on $M$ is related to the complex moduli on $\Mt$ as well as the para-complex moduli on $\Mt'$:
\begin{equation*}
\begin{tikzcd}
& (M,\omega)\arrow[leftrightarrow,sloped,swap]{ld}{\mathclap{mirror}}\arrow[leftrightarrow,sloped]{rd}{\mathclap{para-mirror}}&\\
(\Mt,I)\arrow[leftrightarrow, dotted]{rr}{?}&&(\Mt',K)
\end{tikzcd}
\end{equation*}

Additionally, the para-mirror symmetry could open many doors, for example to study the symplectic invariants -- such as the Gromov-Witten invariants -- of manifolds that do not have mirrors but have para-mirrors, so that the corresponding A-model is mirror to a para-B-model but not to a usual B-model. Similarly, para-complex mirror symmetry could also prove useful in the homological approach to mirror symmetry, providing new insights about the Fukaya category on the symplectic side.

\appendix
\section{Proof of Proposition \ref{prop_toptwist}}\label{appendix:proof_prop}
\begin{proof}
The action of $Q_{L/R}$ on $\phi$ can be immediately read off \eqref{Q12_action}:
\begin{align}
[Q_L,\phi^i]=\frac{1}{2}[Q_+^1+Q^2_+,\phi^i]=\chi^i.
\end{align}

We now prove the formulas for the action of $Q_{L/R}$ on $\chi$, the action on $\lambda$ is entirely analogous. First, we expand
\begin{align}\label{QL_chi}
[Q_L,\chi^i]=[Q_L,(P_+)^i_j\psi_+^j]=[Q_L,(P_+)^i_j]\psi_+^j{+}(P_+)^i_j[Q_L,\psi_+^j].
\end{align}
Because $K_+$ is a function of the bosonic coordinates, we also have $P_+=P_+(\phi)$ and so
\begin{align}\label{QL_P}
[Q_L,(P_+)^i_j]=\p_k(P_+)^i_j[Q_L,\phi^k]=\p_k(P_+)^i_j\chi^k.
\end{align}
Combining \eqref{QL_chi} and \eqref{QL_P} and using \eqref{Q12_action}, we arrive at
\begin{align*}
[Q_L,\chi^i]&=\p_k(P_+)^i_j\chi^k\psi_+^j{+}(P_+)^i_j\left(-(\tl{P}_+)^j_k\p_+\phi^k{-}\frac{1}{2}\p_k(K_+)^j_l\psi_+^k\psi_+^l\right)\\
&=\p_k(P_+)^i_j(P_+)^k_l\psi_+^l\psi_+^j{-}(P_+)^i_j\p_k(P_+)^j_l\psi_+^k\psi_+^l,
\end{align*}
where we denoted $\tl{P}_+=\frac{1}{2}(\id-K_+)$ and used $P_+\tl{P}_+=0$. We now expand the projectors, $(P_+)^i_j=\frac{1}{2}(\delta^i_j+(K_+)^i_j)$:

\begin{align*}
[Q_L,\chi^i]&=\p_k(P_+)^i_j(P_+)^k_l\psi_+^l\psi_+^j{-}(P_+)^i_j\p_k(P_+)^j_l\psi_+^k\psi_+^l\\
&=\frac{1}{2}(\p_j(K_+)^i_l(P_+)^j_k{-}(P_+)^i_j\p_k(K_+)^j_l)\psi_+^k\psi_+^l\\
&=\frac{1}{4}(\p_j(K_+)^i_l(K_+)^j_k{-}(K_+)^i_j\p_k(K_+)^j_l)\psi_+^k\psi_+^l\\
&+\frac{1}{4}(\p_j(K_+)^i_l\psi_+^j\psi_+^l{-}\p_k(K_+)^i_l\psi_+^k\psi_+^l)=-\frac{1}{2}(N_{K_+})^i_{kl}\psi_+^k\psi^l_+,
\end{align*}
where in the last line we rewrote $\psi_+^k\psi^l_+=\frac{1}{2}(\psi^k\psi^l-\psi^l\psi^k)$ to get the expression for the Nijenhuis tensor of $K_+$:
\begin{align*}
4(N_{K_+})^i_{kl}=(K_+)^j_k\p_j(K_+)^i_l-(K_+)^j_l\p_j(K_+)^i_k {-} (K_+)^i_j(\p_k(K_+)^j_l-\p_l(K_+)^j_k).
\end{align*}
Invoking integrability of $K_+$, $N_{K_+}=$ and we conclude that $[Q_L,\chi^i]=0$.

Similarly, $[Q_L,\lambda]$ can be expanded
\begin{align}\label{QL_lambda}
[Q_L,\lambda^i]&=[Q_L,(P_-)^i_j\psi_-^j]=[Q_L,(P_-)^i_j]\psi_-^j+(P_-)^i_j[Q_L,\psi_-^j]\\
&=\p_k(P_-)^i_j\chi^k\psi_-^j-(P_-)^i_k\left((P_+)^k_jF^j+\frac{1}{2}\p_l(K_+)^k_j\psi^l_-\psi_+^j\right),
\end{align}
where we used
\begin{align*}
[Q_L,\psi_-^i]=-(P_+)^i_jF^j\textcolor{red}{-}\frac{1}{2}\p_k(K_+)^i_j\psi^k_-\psi_+^j.
\end{align*}
Next, we rewrite the derivative in terms of $\n^-$ and use the property $\n^-P_-=0$:
\begin{align*}
\p_k(P_-)^i_j&=\n^-_k(P_-)^i_j-(\Gamma^-)^i_{kl}(P_-)^l_j+(\Gamma^-)^l_{kj}(P_-)^i_l\\
&=-(\Gamma^-)^i_{kl}(P_-)^l_j+(\Gamma^-)^l_{kj}(P_-)^i_l,
\end{align*}
so that
\begin{align}\label{eq:calc1}
\begin{aligned}
\p_k(P_-)^i_j\chi^k\psi_-^j=\left(-(\Gamma^-)^i_{kl}(P_-)^l_j+(\Gamma^-)^l_{kj}(P_-)^i_l\right)(P_+)^k_m\psi_+^m\psi_-^j.
\end{aligned}
\end{align}
Similarly,
\begin{align}\label{eq:calc2}
\begin{aligned}
\frac{1}{2}(P_-)^i_k\p_l(K_+)^k_j\psi^l_-\psi_+^j&=(P_-)^i_k\p_l(P_+)^k_j\psi^l_-\psi_+^j\\
&=(P_-)^i_k\left(-(\Gamma^-)^k_{ml}(P_+)^m_j+(\Gamma^-)^m_{jl}(P_+)^k_m\right)\psi^l_-\psi_+^j,
\end{aligned}
\end{align}
where we used
\begin{align}\label{Gamma-pm}
(\Gamma^+)^i_{jk}=g^{im}(\Gamma_{mjk}+\frac{1}{2}H_{jkm})=g^{im}(\Gamma_{mkj}-\frac{1}{2}H_{kjm})=(\Gamma^-)^i_{kj}.
\end{align}
Plugging \eqref{eq:calc1} and \eqref{eq:calc2} into \eqref{QL_lambda} with use of the equation of motion \eqref{F_EoM} for $F$ then yields
\begin{align*}
[Q_L,\lambda^i]&=((\Gamma^-)^i_{km}(P_-)^m_l-(\Gamma^-)^m_{kl}(P_-)^i_m)(P_+)^k_j\psi_-^l\psi_+^j\\
&{-}(P_-)^i_k(-(\Gamma^-)^k_{ml}(P_+)^m_j+(\Gamma^-)^m_{jl}(P_+)^k_m)\psi^l_-\psi_+^j\\
&+(P_-)^i_k(P_+)^k_m(\Gamma^-)^m_{jl}\psi_-^l\psi_+^j.
 \end{align*}
Here all terms cancel except for one, which finally reads
\begin{align*}
[Q_L,\lambda^i]=-(\Gamma^-)^i_{kl}\chi^k\lambda^l.
\end{align*}
\end{proof}

\bibliographystyle{alpha}
\bibliography{mybib}
\end{document}
