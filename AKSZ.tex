\documentclass{article}
\usepackage{amssymb, amsmath, amsthm, mathtools, bbm, tikz-cd,stmaryrd,enumerate,hyperref}
%-------------------
\usepackage{fancyhdr}
\pagestyle{fancy}
\fancyhf{}
\fancyhead[L]{\leftmark}
\fancyhead[R]{\thepage}
%----------------
\newcommand{\TT}{{T\oplus T^*}}
\newcommand{\JJ}{\mathcal{J}}
\newcommand{\KK}{\mathcal{K}}
\newcommand{\GG}{\mathcal{G}}
\newcommand{\Cc}{\mathbf{C}}
\newcommand{\RR}{\mathbb{R}}
\newcommand{\XX}{\mathfrak{X}}
\newcommand{\HH}{\mathcal{H}}
\newcommand{\FF}{\mathcal{F}}
\newcommand{\QQ}{\mathcal{Q}}
\newcommand{\cE}{\mathcal{E}}
\newcommand{\cN}{\mathcal{N}}
\newcommand{\cO}{\mathcal{O}}
%----------------------------------------
\newcommand{\PP}{\mathrm{P}}
\newcommand{\PPt}{\tilde{\mathrm{P}}}
\newcommand{\id}{\mathbbm{1}}
\newcommand{\nlr}{\overset{\leftrightarrow}{\n}}
\newcommand{\lc}{\mathring{\n}}
\newcommand{\im}{\mathrm{Im}\,}
\newcommand{\Ker}{\mathrm{Ker}\,}
\newcommand{\Lie}{\mathcal{L}}
\newcommand{\PS}{\mathcal{P}}
\newcommand{\ap}{\alpha}
\newcommand{\bt}{\beta}
\def\w{\wedge}
\newcommand{\p}{\partial}
\newcommand{\pt}{\tilde{\partial}}
\newcommand{\xt}{{\tilde{x}}}
\newcommand{\n}{\nabla}
\newcommand{\rd}{\mathrm{d}}
\newcommand{\PH}{(\PS,\eta,\omega)}
\newcommand{\Lt}{\tl{L}}
\newcommand{\Lb}{\mathbb{L}}
\newcommand{\s}{\mathbf{s}}
\newcommand{\se}{\Gamma}
\newcommand{\Endo}{\text{End}}
\newcommand{\ellt}{{\tl{\ell}}}
\newcommand{\ot}{{1/2}}
\newcommand{\inv}{{-1}}
\newcommand{\Aa}{\mathcal{A}}
\newcommand{\la}{\langle}
\newcommand{\ra}{\rangle}
\newcommand{\lara}{\la\ ,\ \ra}
\newcommand{\brac}{[\ ,\ ]}
\newcommand{\bl}{[\![}
\newcommand{\br}{]\!]}
\newcommand{\bracd}{\bl \ ,\ \br}
\newcommand{\yt}{\tl{y}}
\newcommand{\zt}{\tl{z}}
\newcommand{\tth}{\tl{\theta}}
\newcommand{\kk}{\mathrm{k}}
\newcommand{\Mt}{\tl{M}}
\newcommand{\pd}{\overline{\p\!\!\!\p}}
\newcommand{\Mb}{\mathbb{M}}
\newcommand{\Drm}{\mathrm{D}}
\newcommand{\Dcal}{\mathcal{D}}
\newcommand{\wtl}{\widetilde}

\def\sA{\mathcal{A}}
\def\d{{\rm d}}
\def\dbar{\Bar{\partial}}

%----------------------------------------
\def\gld{generalized Lie derivative }
\def\glds{generalized Lie derivatives }
\def\tl{\tilde}

\def\xto{\xrightarrow}

% These will be typeset in italics
\newtheorem{theorem}{Theorem}[section]
\newtheorem{proposition}[theorem]{Proposition}
\newtheorem{lemma}[theorem]{Lemma}
\newtheorem{corollary}[theorem]{Corollary}
\newtheorem{fact}[theorem]{Fact}
\newtheorem*{theorem*}{Theorem}
\newtheorem*{lemma*}{Lemma}
\newtheorem*{proposition*}{Proposition}
\newtheorem{Rem}[theorem]{Remark}

% These will be typeset in Roman
\theoremstyle{definition}
\newtheorem{Def}[theorem]{Definition}
\newtheorem{Conj}[theorem]{Conjecture}
\newtheorem{remark}[theorem]{Remark}

\newtheorem*{notation*}{Notation}

\theoremstyle{remark}
\newtheorem{Ex}[theorem]{Example}
\newtheorem{question}[theorem]{Question}
\newenvironment{claim}[1]{\par\noindent\underline{Claim:}\space#1}{}
\newenvironment{claimproof}[1]{\par\noindent\underline{Proof:}\space#1}{\hfill $[acksquare$}


\input xy

\xyoption{all}

\DeclareMathOperator{\End}{End}
\DeclareMathOperator{\rk}{rk}

\def\brian{\textcolor{blue}{BM: }\textcolor{blue}}
\def\david{\textcolor{red}{DB: }\textcolor{red}}
\def\btd{\textcolor{orange}{BM to DB: }\textcolor{orange}}

\begin{document} 

\section{The AKSZ formalism and para-geometry}
\def\fg{\mathfrak{g}}
In this section we will show how to associate a class of $2D$ topological field theories to generalized para-complex structures.
The construction closely follows the relationship between ordinary (non-para) generalized complex structures \cite{Gualtieri:2003dx} and $2D$ topological field theories developed in \cite{Cattaneo:2009zx, Pestun:2006rj}.
For another class of $2D$ boundary theories to the $3D$ Courant algebroid $\sigma$-model see \cite{SeveraTduality}. \btd{Please add any more references you think fit.}

For us, the prescription of defining the $2D$ topological theory associated to a GpC structure starts with a $3D$ topological field theory which is defined for any manifold $M$. 
More generally, this $3D$ theory can be defined for any Courant algebroid on $M$ \cite{Roytenberg:2002nu}, but for most examples we will only consider the standard exact Courant algebroid (or twisted versions) defined by the Dorfman bracket.
The key to our construction is to realize the data of a G(p)C structure in terms of its eigenbundles, which are Dirac structures $L$ in the Courant algebroid $\TT$. 

Via the AKSZ construction \cite{AKSZ}, this correspondence translates into the statement that the Dirac structure $L$ defines a topological boundary theory of the $3D$ topological theory defined by the standard Courant algebroid.
In turn, we will directly describe the BRST observables of the $2D$ theory in terms of the cohomology of the Lie algebroid $L$. 

In Section \ref{sec: twist}, we will relate these theories topological twists of $2D$ $(2,2)$ (para-)SUSY sigma models described in Section \ref{sec:toptwist} closely following the methods in \cite{Kapustin:2004gv} for ordinary generalized complex geometry.

All of the topological field theories in this section are of $\sigma$-model type and arise through the AKSZ formalism \cite{AKSZ}. 
Throughout, we will use the formalism of dg manifolds and shifted symplectic geometry. 
%Later we will use the idea of Lagrangian intersections \brian{PTVV, ButsonYoo, Costello,...}. 

\subsection{Courant algebroids, $3D$ TFTs, and boundary conditions}


From the point of view of topological field theory, Courant algebroids are important because they provide geometric examples of $2$-shifted symplectic spaces. 
Via the AKSZ construction, $2$-shifted symplectic spaces are the natural home for $3$-dimensional topological field theories in the BV formalism. 
We recall the construction of the $3D$ topological field theory from the data of a Courant algebroid. 
For other references, see \cite{Roytenberg:2002nu, Cattaneo:2009zx}. 

\subsubsection{Shifted symplectic geometry} 
\label{sec: dgman}

We begin by setting up our model for the theory of derived manifolds. 
For us, this is a well-behaved class of $NQ$ manifolds which are appropriate for setting up quantization of $\sigma$-models, see for instance \cite{CostelloSUSY}. 
A dg manifold is a pair $\cN = (N, \sA^*_\cN)$ where $N$ is a smooth manifold, called the body, and $\sA^*_\cN$ a sheaf of graded commutative algebras over the de Rham complex $\Omega^*_N$.
Here, $\d_\cN$ is a linear differential operator of degree $+1$, and together the data must satisfy the following conditions:
\begin{itemize}
\item[(1)] $\sA_\cN$ is concentrated in finitely many degrees;
\item[(2)] For each $k$, $\sA^k_\cN$ is a locally free sheaf of $C^\infty_N$-modules of finite rank;
\item[(3)] The differential $\d_N : \sA_\cN^* \to \sA_\cN^{*+1}$ is square zero differential operator making $(\sA_\cN , \d_\cN)$ into a sheaf of commutative dg algebras over the de Rham complex $\Omega^*_N$.
\end{itemize}

In particular, as a graded algebra $\sA^*_\cN$ is given by functions on the total space of some graded vector bundle $A_\cN$ on $N$ (which is of finite rank and concentrated in finitely many degrees). 
In the language of $NQ$ manifolds, the homological vector field defining the $Q$-structure is $\d_\cN$. 

\brian{Symplectic form}

A striking result of \cite{Roytenberg:2002nu} classifies all $2$-shifted symplectic spaces in terms of Courant algebroids. 

\begin{theorem}[\cite{Roytenberg:2002nu}]
There is an equivalence between isomorphism classes of $2$-shifted symplectic NQ manifolds with body $M$ and isomorphism classes of Courant algebroids on $M$.
\end{theorem}

\begin{remark}
In \cite{PymSafronov} it was shown that a more general class of $2$-shifted symplectic derived spaces, called $L_\infty$ algebroids, are equivalent to {\em twisted} Courant algebroids. 
This is similar to a Courant algebroid, where the Jacobi identity only holds up to homotopy given by some closed $4$-form. 
\end{remark}

We briefly recount the equivalence between the data of a $2$-shifted symplectic dg manifold and the data of a Courant algebroid.

\def\Sym{{\rm Sym}}

%Given any Courant algebroid $E \to M$, we can have the following sequence of maps $T^* \xto{a^*} E \xto{a} T$ where $a$ is the anchor and $a^*$ is linear dual to the anchor.
%Using this sequence, we can define the associated $NQ$ manifold $X_E$, following \cite{Ryotenberg:2002nu}.
Let $E \to M$ be a vector bundle equipped with a fiberwise nondegenerate inner product $\left<-,-\right>$.
The body of the dg manifold is the manifold $M$, and the underlying graded manifold $X_E$ is given by $T^*M [2] \oplus E[1]$. 

We will use coordinates $\{x^i\}$ on $M$, $\{p_i\}$ for the fiber coordinate of $T^*$, and $\{e^a\}$ for the fiber of $E$. 
Note that $x^i$ is of degree zero, $\eta_i$ is of degree $2$, and $e^a$ is of degree $1$.
Suppose also that $\left<e^a, e^b\right> = g^{ab}$ and let $(g_{ab})$ be the inverse to the inner product. 
Notice, $X_E$ comes equipped with a natural $2$-shifted symplectic form, which in coordinates is 
\[
\omega_E = d x^i d p_i + g_{ab} d e^a d e^b .
\]
Denote by $\{-,-\}$ the shifted Poisson bracket corresponding to this symplectic structure. 
An arbitrary degree $3$ function on the graded manifold $X_E$ has the form
\[
\Theta = p_i a_a^i (x) e^a + \frac{1}{6} f_{abc} (x) e^a e^b e^c .
\]
Globally, $a = (a_a^i)$ defines a bundle map $a : E \to T$ and through the pairing the collection $(f_{abc})$ defines a bilinear map $[-,-] : E \times E \to E$. 
It is a result of \cite{Roytenberg:2002nu} that this function satisfies $\{\Theta, \Theta\} = 0$ if and only if the data $(E, \left<-,-\right>, a, [-,-])$ has the structure of a Courant algebroid. 
Here $a$ is the anchor map, and $[-,-]$ is the bracket. 

%\[
%Q : \Gamma(M , \Sym \left(T^*[2] \oplus E[1]\right)^*) \to \Gamma(M , \Sym \left(T^*[2] \oplus E[1]\right)^*) 
%\]
%as follows: for $\psi \in \Sym^{k + \ell + 2} \left(T^*[2] \oplus E[1]\right)^*)$ and
%\[
%(\xi_0 \otimes \cdots \otimes \xi_k) \otimes (\alpha_0 \otimes \cdots \otimes \alpha_\ell) \in \Sym^{k+1}(T^*[2]) \otimes \Sym^{\ell + 1}(E[1]) 
%\]
%define
%\begin{align*}
%(Q \psi) ((\xi_0 \otimes \cdots \otimes \xi_k) \otimes (\alpha_0 \otimes \cdots \otimes \alpha_\ell)) = \brian{finish}
%\end{align*}
%
%All that remains is to describe the $2$-shifted symplectic structure on the NQ manifold $X_E$. 
%This is defined via the obvious pairing between $T^*[2]$ and $T$ together with the pairing $\langle - ,- \rangle$ on $E[1]$.
%In local coordinates \brian{finish}

\subsubsection{The AKSZ construction}

Fix the following data:
\begin{itemize}
\item an {\em $n$-oriented} dg manifold $\cN = (N, (\sA^*_N, \d_\cN))$ with body a smooth oriented manifold $N$ and orientation $\mu$. 
\item An $(n-1)$-shifted symplectic dg manifold $(X,\omega)$. 
\end{itemize}

The starting point of the AKSZ construction is the mapping space
\begin{equation}\label{eqn: map}
\cE (\cN, X) = {\rm Map}\left(\cN, X\right)
\end{equation}
As a graded manifold, the mapping space $\cE(N,X)$ is given by the space of smooth maps between the underlying graded manifolds $A_N$ and $X$. 
The graded manifold $\cE(\cN, X)$ is equipped with the homological vector field $\d_\cN + Q$ where $Q$ is the homological vector field on $X$. 

The AKSZ construction endows the mapping space $\cE(N, X)$ in (\ref{eqn: map}) with a $(-1)$-shifted symplectic symplectic form as follows, compatible with the homological vector field $\d_{\cN} + Q$ as follows.
The fundamental observation is that the diagram
\[
\xymatrix{
& N \times \cE(N, X) \ar[dr]^-{{\rm ev}} \ar[dl]_-{\pi_N} & \\
N & & X
}
\]
induces a pairing
\[
\begin{array}{ccccc}
 \Omega^p_N& \times & \Omega^q_X & \to & \Omega^{p+q-n}_{\cE(N, X)} \\
\displaystyle \alpha & \times & \beta & \mapsto & \displaystyle \int_N \pi_N^* \alpha \wedge {\rm ev}^* \beta 
 \end{array}
 \]
Applied to the element $1 \times \omega_X \in \Omega^0_N \times \Omega^{\brian{??}}_X$ we obtain a $(-1)$-shifted symplectic form that we denote $\int_N \omega_X$.

\begin{Ex}
The most important example for us will be the source dg manifold $(N, \Omega^*_N)$, that is $\sA_N = \Omega^*_N$ equipped with de Rham differential. 
We denote this dg manifold by $N_{\rm dR}$.
Take the dg manifold $(N, \Omega^*_N)$ where $N$ is any smooth manifold. 
If $N$ is closed and oriented there exists an integration map
\[
\int_N : \Omega^*_N \to \RR[n]
\]
of degree $-n$ thus equipping the dg manifold $N_{\rm dR}$ with an $n$-orientation. 
\end{Ex}

\begin{Ex}
If $N$ has a para-complex structure, $(N, \mathbf{\Omega}^{0,\bullet})$ has the structure of a dg manifold. 
Here $\mathbf{\Omega}^{0,\bullet}$ is the para-Dolbeault complex from Section \ref{sec: paracomplex} with differential given by the para $\pd$-operator. 
We will denote this dg manifold by $N_{\pd}$. 
To equip the dg manifold $(N, \mathbf{\Omega}^{0,\bullet})$ with an orientation, one must fix additional data. 
One way to do this is to assume that the para-complex manifold $N$ is equipped with a para-holomorphic volume form $\Omega_N$. \brian{Cite David's new section.}
\end{Ex}

\brian{Hamiltonians and action functional}

\subsubsection{The AKSZ theory associated to a Courant algebroid}

From here on, to any Courant algebroid $E$ on a manifold $M$ we associate the dg manifold $X_E$ which carries a $2$-shifted symplectic structure. 
In the case that the Courant algebroid is exact, call the associated $2$-shifted symplectic space $X_H$, where $H$ labels the Severa class. 

We first review some basic examples of AKSZ theories associated to Courant algebroids. 
The simplest type of Courant algebroid is one over a point, see Example \ref{Ex: point}. 
The resulting AKSZ theory is a familiar one.

\begin{Ex}\label{ex: cs}
Any Lie algebra $\fg$ together with a non-degenerate invariant pairing defines a $2$-shifted symplectic structure on the graded manifold $\fg[1]$ living over a point. 
The dg algebra of functions is the Chevalley-Eilenberg cochain complex computing Lie algebra cohomology $C^*(\fg)$. 
All such $2$-symplectic dg manifolds over a point are of this form. 
%The resulting AKSZ theory is Chern-Simons theory. 
The resulting AKSZ theory on a $3$-manifold $M$ (whose target base manifold is a point) is equivalent to Chern-Simons theory on $M$. 
\end{Ex}

\begin{Ex} \label{ex: exact CA}
If $E$ is a Courant algebroid on $M$ let $X_E$ be the corresponding $2$-shifted symplectic manifold.
We use the same coordinates as in Section \ref{sec: dgman}.
The action functional is
\[
S = \int_N p_i \d x^i + \frac{1}{2} g_{ab}(x) e^a \d e^b - a_a^i(x) p_i e^a + \frac{1}{6} f_{abc} (x) e^{a} e^b e^c .
\]
\end{Ex}

\subsubsection{Dirac structures and boundary conditions}

The AKSZ formalism is a construction which produces a $(-1)$-shifted symplectic space from the data of a closed manifold and a target shifted symplectic manifold. 
There is a generalization of this construction which produces $(-1)$-shifted symplectic spaces on manifolds with boundary. 

Suppose $N$ is a manifold with boundary, and let $\cE(N)$ denote the space of fields of a classical theory on $M$. 
For us, $\cE(N)$ will be a mapping space of the form ${\rm Map}(M, X)$ where $X$ is shifted symplectic. 
Let $\cE(\partial N)$ denote the restriction of the fields to the boundary of $N$. 
For the mapping space example, this is simply the space of maps ${\rm Map}(\partial N, X)$.

In general, when $N$ is not closed, the pairing defined by the AKSZ construction will not endow $\cE(N)$ will with $(-1)$-shifted symplectic structure. 
However, the space $\cE(\partial N)$ does carry a natural ordinary ($0$-shifted) symplectic structure from the restriction of the pairing defined on $\cE(N)$. 
Morever, the natural restriction map $\cE(N) \to \cE(\partial N)$ is a Lagrangian morphism in the sense of \cite{PTVV}. 

\def\cL{\mathcal{L}}

A boundary condition is the choice of an additional Lagrangian subspace $\cL$ of $\cE(\partial N)$.
The intersection $\cL \times_{\cE(N)} \cE(\partial N)$ of the two Lagrangians $\cL$ and $\cE(\partial N)$ 

For an algebraic proof of the fact that the Lagrangian intersection \brian{finish} \cite{Calaque}

\begin{Ex}
Suppose $(M, \pi)$ is a Poisson manifold. 
Then consider the graph of $\pi$ as a Lagrangian subbundle of $\TT$:
\[
{\rm Graph}(\pi) = \{(\pi \vee \alpha , \alpha) \; | \; \alpha \in T^* \} \subset \TT .
\]
By the Jacobi identity, this subbundle is integrable and so determines a Dirac structure. 
If we place the 3D AKSZ theory with target the Courant algebroid $\TT$ on $N^3$ with $\partial N = \Sigma$, the boundary condition determined by this Dirac structure is equivalent to the Poisson $\sigma$-model on $\Sigma$, see \cite{KSSdirac}, for instance.
\end{Ex}

\subsection{Two dimensional TFT from GpC structures}

Let $M$ be a manifold.
Throughout this section we will only consider the standard Dorfman Courant algebroid $X_H$ on $M$, possibly twisted by an $H$-flux.
By Theorem \ref{thm:pairofdirac}, the choice of a GpC structure $\KK$ determines a pair of transversal Dirac structures $L_{\KK}$ and $\tilde{L}_{\KK}$ on $M$ given by the positive and negative eigenbundles of $\KK$. 
In particular, $L$ determines a Lagrangian in the shifted symplectic space $X_{H}$. 
This Lagrangian then defines a boundary condition for the three-dimensional AKSZ theory.
Thus, we see that any GpC structure $\KK$ defines a 2D TFT living at the boundary of the 3D theory. 
By Proposition \ref{prop:dirac_Liealg} the eigenbundle $L_\KK$ has the structure of a Lie algebroid compatible with the anchor and Dorfman bracket on $\TT$. 

\begin{remark}
More generally, if $\KK$ is a generalized complex structure in an arbitrary Courant algebroid $E$, then we obtain a $2$-shifted Lagrangian inside of the $2$-shifted symplectic space $X_E$. 
\end{remark}

Before stating the main result of this section, we recall the following general construction.
Given any Lie algebroid $L$ on $M$ consider the graded manifold $L[1]$ with body $M$. 
Functions on this graded manifold are given by sections of the graded vector bundle
\[
\cO(L[1]) = \bigoplus_{k \geq 0} \wedge^k L [-k] .
\]
Here, $\wedge^k(L)$ is placed in cohomological degree $+k$. 
There is square-zero derivation $\d_L$ of degree $+1$ on $\cO(L[1])$ defined by the following rule. 
If $\varphi$ is a section of $\wedge^k L$, then
\begin{align*}
\left( \d_L \varphi \right) (\ell_0, \ldots, \ell_k) & = \sum_{i = 0}^k (-1)^i a (\ell_i) \varphi(\ell_0, \ldots, \Hat{\ell_i}, \ldots, \ell_k) \\ & + \sum_{i < j} (-1)^{i+j} \varphi([\ell_i, \ell_j] , \ell_0 \cdots, \Hat{\ell_i}, \ldots, \Hat{\ell_j}, \ldots, \ell_k)
\end{align*}
Here $a : L \to T_M$ is the anchor and $[-,-]$ is the bracket on $L$. 
In other words $\d_L$ is a homological vector field on $L$, which can be written in coordinates:
\[
\d_L = a_a^i \theta^a \frac{\partial}{\partial x^i} + f_{ab}^c \theta^a \theta^b \frac{\partial}{\partial \theta^c}
\]
where $\{x^i\}$ are the coordinates on $M$ and $\{\theta^a\}$ are the coordinates for the fiber of $L[1]$.

Equipped with this differential, the cohomology
\[
H^*( \cO(L[1]), \d_L )
\]
is called the Lie algebroid cohomology of $L$.

\begin{proposition}
Let $\KK$ be a generalized para-complex structure on $M$ and $L_{\KK}$ the corresponding positive eigenbundle. 
The BRST cohomology of the local observables of the 2D TFT determined by $\KK$ is equivalent to the Lie algebroid cohomology of $L_{\KK}$. 
\end{proposition}

\subsubsection*{Trivial GpC structures}

Let $M$ be a smooth manifold.
Consider the case of the trivial generalized para-complex structure on $M$ defined by
\begin{align*}
\KK_0=
\begin{pmatrix}
\id & 0 \\
0 & -\id
\end{pmatrix}.
\end{align*}
The positive eigenbundle is simply $L_\KK = T$ and the resulting Lie algebroid structure is the trivial one with zero anchor.

\brian{Trivial Poisson $\sigma$-model, AKA $BF$ theory.}

\subsubsection*{(Real) Poisson $\sigma$-model}



\subsubsection*{Para-complex $A/B$-models}

We have seen that symplectic structures are a special case of a generalized para-complex structure. 
If $(X, \omega)$ is a symplectic the resulting 2D AKSZ theory is equivalent to the ordinary $A$-model on $X$. 

Another special case of a generalized para-complex structure is para-complex structure, see Example \ref{eg: pc}. 
In this case, the generalized AKSZ theory specializes to the following para-complex version of the $B$-model.

\begin{Def}
The {\bf para-$B$-model} with source a closed Riemann surface $\Sigma$ and target a para-complex manifold $X$ is the AKSZ theory with source the de Rham space $\Sigma_{\rm dR}$ and target the $1$-shifted symplectic space $T^*[1] X_{\pd}$:
\[
{\rm Map}\left(\Sigma_{dR}, T^*[1]  X_{\pd}\right) .
\]
\end{Def}

The classical BRST cohomology of the para-B-model is isomorphic to the cohomology of para-holomorphic polyvector fields on the target para-complex manifold $X$.
To see the equivalence of this definition with the generalized description of the AKSZ model, note that the Lie algebroid $L_{\KK}$ can be identified with $T^{1,0} \oplus T^{*0,1}$.  
Its dual is $\Bar{L}_{\KK} = T^{0,1} \oplus T^{*1,0}$. 

The Lie algebroid cohomology $H^*(\cO(L_{\KK}[1]), \d_{L_\KK})$ is computed by a differential of the form
\[
\d_{L_\KK} : \Gamma(M , \wedge^k (T^{1,0} \oplus T^{*0,1})) \to \Gamma(M , \wedge^{k+1} (T^{1,0} \oplus T^{*0,1})) 
\]
Using the splitting $\wedge^*(T^{1,0} \oplus T^{*0,1}) = \wedge^* T^{1,0} \otimes \wedge^*(T^{*0,1})$ we can identify this differential with the Dobleault differential for the bundle of para-holomorphic polyvector fields $\Theta = \wedge^* T^{1,0}$:
\[
\d_L = \pd : \mathbf{\Omega}^{0,q} (M , \Theta) \to \mathbf{\Omega}^{0,q+1} (M , \Theta)  .
\]
Thus, the BRST cohomology is precisely the Dolbeault cohomology of para-holomorphic polyvector fields.

\brian{discuss anomalies. Para-holomorphic volume form.}

\brian{Discuss Si's work on perturbative quantization}


\hrulefill

\brian{***para Kahler example, **example where the paracomplex structures commute but are not equal (this is where people see ``twisted chiral"), para version of T-duality, **Poisson $\sigma$-model, going back and forth between generalized (para) complex.
}  

\brian{Case of para structure on $M \times M$. Two copies of $BF$ theory.}

\subsection{Para holomorphic variants}

\brian{My plan is to just turn this into an extended remark}

So far, in each of the $\sigma$-models we have discussed in the AKSZ formalism, the fields have depended only topologically on the source Riemann surface. 
There are closely related $\sigma$-models which depending {\em para-holomorphically} on the source Riemann surface that we briefly discuss. 

\bibliographystyle{alpha}
\bibliography{mybib}
\end{document}



%\section{The AKSZ formalism and 3d TFT}
%\def\fg{\mathfrak{g}}
%In this section we will show how to associate $2D$ topological theories to a generalized (para-)complex structure and relate them to the topological twists of $2D$ $(2,2)$ (para-)SUSY sigma models described in Section \ref{sec:toptwist} and in \cite{Kapustin:2004gv}. 
%The key to our construction is to realize the data of a G(p)C structure in terms of its eigenbundles, which are Dirac structures $L$ in the Courant algebroid $\TT$. 
%Via the AKSZ construction \cite{AKSZ}, this correspondence translates into the statement that the Dirac structure $L$ defines a topological boundary theory of a $3D$ topological theory defined by the Courant algebroid $\TT$ \cite{Roytenberg:2002nu}.
%
%Throughout, we will use symplectic $NQ$ manifold as our model for shifted symplectic geometry. 
%We will recall the general setup of associating a $3D$ topological field theory to the data of a Courant algebroid. 
%For a reference, see \cite{Roytenberg:2002nu,??}.
%
%\subsection{Courant algebroids, $3D$ TFTs, and boundary conditions}
%From the point of view of topological field theory, Courant algebroids are important because they provide geometric examples of $2$-shifted symplectic spaces. 
%Via the AKSZ construction, $2$-shifted symplectic spaces are the natural home for $3$-dimensional topological field theories in the BV formalism. 
%\brian{give citations here.}
%
%In \cite{Roytenberg:2002nu}, it is shown how to associate a $2$-shifted symplectic NQ manifold from the data of a Courant algebroid. 
%
%\begin{theorem}[\cite{Roytenberg:2002nu}]
%There is an equivalence between isomorphism classes of $2$-shifted symplectic NQ manifolds with body $M$ and isomorphism classes of Courant algebroids on $M$.
%\end{theorem}
%
%We briefly recall this construction. 
%
%Recall, the data of a Courant algebroid on a manifold $M$ is a vector bundle $E$, whose sheaf of sections we denote $\cE$, together with a nondegenerate symmetric bilinear pairing $\langle -, -\rangle : \cE \otimes \cE \to C^\infty(M)$ , an anchor map $a : E \to T M$, and a bilinear operator
%\[
%[-,-] : \cE \times \cE \to \cE
%\]
%called the Courant-Dorfman bracket. 
%\brian{spell out properties, find a good reference}
%Denote by $a^* : T^* M \to E^*$ the bundle map dual to $a$. 
%
%To every Courant algebroid we can define the following sequence of vector bundles on $M$
%\[
%T^*M \xto{a^*} E \xto{a} TM .
%\]
%A Courant algebroid is called {\em exact} if this sequence of bundles is exact. 
%
%\def\Sym{{\rm Sym}}
%
%Using this sequence, we can define the associated $NQ$ manifold $X_E$, following \cite{Ryotenberg:2002nu}.
%The body of the NQ manifold is $M$, and the underlying graded vector bundle is given by $T^*[2] \oplus E[1]$. 
%Define the degree $+1$ vector field 
%\[
%Q : \Gamma(M , \Sym \left(T^*[2] \oplus E[1]\right)^*) \to \Gamma(M , \Sym \left(T^*[2] \oplus E[1]\right)^*) 
%\]
%as follows: for $\psi \in \Sym^{k + \ell + 2} \left(T^*[2] \oplus E[1]\right)^*)$ and
%\[
%(\xi_0 \otimes \cdots \otimes \xi_k) \otimes (\alpha_0 \otimes \cdots \otimes \alpha_\ell) \in \Sym^{k+1}(T^*[2]) \otimes \Sym^{\ell + 1}(E[1]) 
%\]
%define
%\begin{align*}
%(Q \psi) ((\xi_0 \otimes \cdots \otimes \xi_k) \otimes (\alpha_0 \otimes \cdots \otimes \alpha_\ell)) = \brian{finish}
%\end{align*}
%
%All that remains is to describe the $2$-shifted symplectic structure on the NQ manifold $X_E$. 
%This is defined via the obvious pairing between $T^*[2]$ and $T$ together with the pairing $\langle - ,- \rangle$ on $E[1]$.
%In local coordinates \brian{finish}
%
%The simplest type of Courant algebroid is one over a point.
%
%\begin{Ex}\label{ex: cs}
%Any Lie algebra $\fg$ together with a non-degenerate invariant pairing defines a $2$-shifted symplectic structure on the graded manifold $\fg[1]$ living over a point. 
%The dg algebra of functions is the Chevalley-Eilenberg cochain complex computing Lie algebra cohomology $C^*(\fg)$. 
%All such $2$-symplectic NQ manifolds over a point are of this form. 
%%The resulting AKSZ theory is Chern-Simons theory. 
%\end{Ex}
%
%\subsubsection*{The $2$-shifted symplectic space of an exact Courant Algebroid}
%
%We recall the description of the $2$-shifted symplectic NQ manifold $X_E$ associated to an exact Courant algebroid $E$.
%Of course, as a vector bundle $E = \TT$, and hence as a graded manifold $X_E$ is of the form $T^*[2] T[1] M$.
%Concretely, the coordinates on this manifold are given by $(x^i,v^a,p_i,\mu_a)_{i,a=1\cdots n}$, where $x^i$ are the usual coordinates on $M$ with degree $0$, $v^a$ are the corresponding ``velocities'', i.e. the fibre coordinates on $T[1]M$ with a degree $1$, and $p_i,\mu_a$ are the momenta corresponding to $x^i,v^a$ with degrees $2,1$, respectively. The canonical symplectic form
%\begin{align*}
%\Omega= dx^idp_i+dv^ad\mu_a,
%\end{align*}
%is then of degree $2$. $T^*[2]T[1]M$ is the minimal symplectic realization of the shifted bundle $(\TT)[1]M$, which is recovered by setting $p_i=0$. The symmetric part of $\Omega$ represents the pairing $\lara$ on $(\TT)M$; this is because we can identify $d\mu_a\leftrightarrow \p_i$ and $dv^a\leftrightarrow dx^i$
%
%\brian{Severa classification}
%
%\begin{theorem}\label{thm: severa}
%The set of isomorphism classes of exact Courant algebroids is a torsor for $H^3(X)$. 
%\end{theorem}
%
%\subsubsection*{The Poisson Courant algebroid}
%
%\brian{Both real and holomorphic versions}
%
%\subsubsection{The AKSZ theory associated to a Courant algebroid}
%
%To any Courant algebroid $E$, we can associate an NQ manifold $X_E$ which carries a $2$-shifted symplectic structure that we denote by $\omega_E$. 
%In the case that the Courant algebroid is exact, call the associated $2$-shifted symplectic space $X_H$, where $H$ labels the Severa class. 
%
%\def\sA{\mathcal{A}}
%\def\d{{\rm d}}
%\def\dbar{\Bar{\partial}}
%
%The starting point of the AKSZ construction is the mapping space
%\[
%{\rm Map}\left((M, (\sA^*_M, \d_M)), X_E\right)
%\]
%where $(M, (\sA^*_M, \d_M))$ is a dg manifold with body a smooth oriented manifold $M$. 
%That is, $\sA^*_M$ is a sheaf of graded commutative algebras over the de Rham complex $\Omega^*_M$ and $\d_M$ is a linear operator of degree $+1$ satisfying:
%\begin{itemize}
%\item[(1)] $\sA_M$ is concentrated in finitely many degrees;
%\item[(2)] For each $k$, $\sA^k_M$ is a locally free sheaf of $C^\infty_M$-modules of finite rank;
%\item[(3)] The differential $\d_M : \sA_M^* \to \sA_M^{*+1}$ is square zero differential operator making $(\sA_M , \d_M)$ into a sheaf of commutative dg algebras over the de Rham complex $\Omega^*_M$
%\end{itemize}
%
%\begin{Ex}
%The most important example for us will be the source dg manifold $(M, \Omega^*_M)$, that is $\sA_M = \Omega^*_M$ with de Rham differential. 
%We denote this dg manifold by $M_{\rm dR}$.
%\end{Ex}
%
%\begin{Ex}
%If $M$ has a para-complex structure, then $(M, \Omega^{0,*}_M)$ has the structure of a dg manifold with differential given by the para $\dbar$-operator. 
%\brian{David, is this consistent notation?}
%We will denote this dg manifold by $M_{p \dbar}$. 
%\end{Ex}
%
%\begin{Ex}
%If $\fg$ is a Lie algebra equipped with a non-degenerate invariant pairing, and $X_E = \fg[1]$ as in Example \ref{ex: cs}, then the AKSZ theory (whose target base manifold is a point) is equivalent to Chern-Simons theory on $M$. 
%\end{Ex}
%
%\subsubsection{Dirac structures and boundary conditions}
%
%\subsection{Two dimensional TFT from GpC structures}
%\label{sec:GpC_AKSZ}
%\brian{Recall that a generalized (para-)complex structure $\JJ$ defines a Dirac structure in $\TT$, and hence a Lagrangian inside of the $2$-shifted symplectic space $\XX_H$.  
%%More generally, if $\JJ$ is a generalized complex structure in an arbitrary Courant algebroid $E$, then we obtain a $2$-shifted Lagrangian inside of the $2$-shifted symplectic space $\XX_E$. 
%This Lagrangian then defines a boundary condition for the three-dimensional AKSZ theory. }
%
%\subsubsection*{Trivial GpC structure}
%
%\subsubsection*{(Real) Poisson $\sigma$-model}
%
%\subsubsection*{Para-complex $A/B$-models}
%
%We point out two special cases of the above general construction. 
%
%\brian{The $A$-model is the usual thing.}
%
%\begin{Def}
%The {\bf para-complex $B$-model} with source a closed Riemann surface $\Sigma$ and target a para-complex manifold $X$ is the AKSZ theory with source the de Rham space $\Sigma_{\rm dR}$ and target the $1$-shifted symplectic space $T^*[1] X_{p \dbar}$:
%\[
%{\rm Map}\left(\Sigma_{dR}, T^*[1]  X_{p \dbar}\right) .
%\]
%\end{Def}
%
%\brian{Discuss Si's work on perturbative quantization}
%
%\hrulefill
%
%\brian{***para Kahler example, **example where the paracomplex structures commute but are not equal (this is where people see ``twisted chiral"), para version of T-duality, **Poisson $\sigma$-model, going back and forth between generalized (para) complex.
%}  
%
% \brian{discuss anomalies}
%
%\subsection{Para holomorphic variants}
%
%So far, in each of the $\sigma$-models we have discussed in the AKSZ formalism, the fields have depended only topologically on the source Riemann surface. 
%There are closely related $\sigma$-models which depending {\em para-holomorphically} on the source Riemann surface that we briefly discuss. 
