\documentclass{article}
\usepackage{amssymb, amsmath, amsthm, mathtools, bbm, tikz-cd,stmaryrd,enumerate,hyperref}
%-------------------
\usepackage{fancyhdr}
\pagestyle{fancy}
\fancyhf{}
\fancyhead[L]{\leftmark}
\fancyhead[R]{\thepage}
%----------------
\newcommand{\TT}{{T\oplus T^*}}
\newcommand{\JJ}{\mathcal{J}}
\newcommand{\KK}{\mathcal{K}}
\newcommand{\GG}{\mathcal{G}}
\newcommand{\Cc}{\mathbf{C}}
\newcommand{\RR}{\mathbb{R}}
\newcommand{\XX}{\mathfrak{X}}
\newcommand{\HH}{\mathcal{H}}
\newcommand{\FF}{\mathcal{F}}
\newcommand{\QQ}{\mathcal{Q}}
\newcommand{\cE}{\mathcal{E}}
%----------------------------------------
\newcommand{\PP}{\mathrm{P}}
\newcommand{\PPt}{\tilde{\mathrm{P}}}
\newcommand{\id}{\mathbbm{1}}
\newcommand{\nlr}{\overset{\leftrightarrow}{\n}}
\newcommand{\lc}{\mathring{\n}}
\newcommand{\im}{\mathrm{Im}\,}
\newcommand{\Ker}{\mathrm{Ker}\,}
\newcommand{\Lie}{\mathcal{L}}
\newcommand{\PS}{\mathcal{P}}
\newcommand{\ap}{\alpha}
\newcommand{\bt}{\beta}
\def\w{\wedge}
\newcommand{\p}{\partial}
\newcommand{\pt}{\tilde{\partial}}
\newcommand{\xt}{{\tilde{x}}}
\newcommand{\n}{\nabla}
\newcommand{\rd}{\mathrm{d}}
\newcommand{\PH}{(\PS,\eta,\omega)}
\newcommand{\Lt}{\tl{L}}
\newcommand{\Lb}{\mathbb{L}}
\newcommand{\s}{\mathbf{s}}
\newcommand{\se}{\Gamma}
\newcommand{\Endo}{\text{End}}
\newcommand{\ellt}{{\tl{\ell}}}
\newcommand{\ot}{{1/2}}
\newcommand{\inv}{{-1}}
\newcommand{\Aa}{\mathcal{A}}
\newcommand{\la}{\langle}
\newcommand{\ra}{\rangle}
\newcommand{\lara}{\la\ ,\ \ra}
\newcommand{\brac}{[\ ,\ ]}
\newcommand{\bl}{[\![}
\newcommand{\br}{]\!]}
\newcommand{\bracd}{\bl \ ,\ \br}
\newcommand{\yt}{\tl{y}}
\newcommand{\zt}{\tl{z}}
\newcommand{\tth}{\tl{\theta}}
\newcommand{\kk}{\mathrm{k}}
\def\fg{\mathfrak{g}}

\newcommand{\Mt}{\tl{M}}
\newcommand{\pd}{\p\!\!\!\p}
\newcommand{\Mb}{\mathbb{M}}
%----------------------------------------
\def\gld{generalized Lie derivative }
\def\glds{generalized Lie derivatives }
\def\tl{\tilde}

\def\xto{\xrightarrow}

% These will be typeset in italics
\newtheorem{theorem}{Theorem}[section]
\newtheorem{proposition}[theorem]{Proposition}
\newtheorem{lemma}[theorem]{Lemma}
\newtheorem{corollary}[theorem]{Corollary}
\newtheorem{fact}[theorem]{Fact}
\newtheorem*{theorem*}{Theorem}
\newtheorem*{lemma*}{Lemma}
\newtheorem*{proposition*}{Proposition}
\newtheorem{Rem}[theorem]{Remark}

% These will be typeset in Roman
\theoremstyle{definition}
\newtheorem{Def}[theorem]{Definition}
\newtheorem{Conj}[theorem]{Conjecture}
\newtheorem{remark}[theorem]{Remark}

\newtheorem*{notation*}{Notation}

\theoremstyle{remark}
\newtheorem{Ex}[theorem]{Example}
\newtheorem{question}[theorem]{Question}
\newenvironment{claim}[1]{\par\noindent\underline{Claim:}\space#1}{}
\newenvironment{claimproof}[1]{\par\noindent\underline{Proof:}\space#1}{\hfill $[acksquare$}


\input xy

\xyoption{all}

\DeclareMathOperator{\End}{End}
\DeclareMathOperator{\rk}{rk}

\def\brian{\textcolor{blue}{BM: }\textcolor{blue}}
\def\david{\textcolor{red}{DB: }\textcolor{red}}

\begin{document} 

\section{The AKSZ formalism and para-geometry}
\def\fg{\mathfrak{g}}
In this section we will show how to associate $2D$ topological theories to a generalized (para-)complex structure and relate them to the topological twists of $2D$ $(2,2)$ (para-)SUSY sigma models described in Section \ref{sec:toptwist} and in \cite{Kapustin:2004gv}. 
The key to our construction is to realize the data of a G(p)C structure in terms of its eigenbundles, which are Dirac structures $L$ in the Courant algebroid $\TT$. 
Via the AKSZ construction \cite{AKSZ}, this correspondence translates into the statement that the Dirac structure $L$ defines a topological boundary theory of a $3D$ topological theory defined by the Courant algebroid $\TT$ \cite{Roytenberg:2002nu}.

Throughout, we will use symplectic $NQ$ manifold as our model for shifted symplectic geometry. 
We will recall the general setup of associating a $3D$ topological field theory to the data of a Courant algebroid. 
For a reference, see \cite{Roytenberg:2002nu,??}.

Later we will use the idea of Lagrangian intersections \brian{Toen, ButsonYoo, Costello,...}. 
\subsection{Courant algebroids, $3D$ TFTs, and boundary conditions}
From the point of view of topological field theory, Courant algebroids are important because they provide geometric examples of $2$-shifted symplectic spaces. 
Via the AKSZ construction, $2$-shifted symplectic spaces are the natural home for $3$-dimensional topological field theories in the BV formalism. 
\brian{give citations here.}

In \cite{Roytenberg:2002nu}, it is shown how to associate a $2$-shifted symplectic NQ manifold from the data of a Courant algebroid. 

\begin{theorem}[\cite{Roytenberg:2002nu}]
There is an equivalence between isomorphism classes of $2$-shifted symplectic NQ manifolds with body $M$ and isomorphism classes of Courant algebroids on $M$.
\end{theorem}

We briefly recount this construction. 


\def\Sym{{\rm Sym}}

Using this sequence, we can define the associated $NQ$ manifold $X_E$, following \cite{Ryotenberg:2002nu}.
The body of the NQ manifold is $M$, and the underlying graded vector bundle is given by $T^*[2] \oplus E[1]$. 
Define the degree $+1$ vector field 
\[
Q : \Gamma(M , \Sym \left(T^*[2] \oplus E[1]\right)^*) \to \Gamma(M , \Sym \left(T^*[2] \oplus E[1]\right)^*) 
\]
as follows: for $\psi \in \Sym^{k + \ell + 2} \left(T^*[2] \oplus E[1]\right)^*)$ and
\[
(\xi_0 \otimes \cdots \otimes \xi_k) \otimes (\alpha_0 \otimes \cdots \otimes \alpha_\ell) \in \Sym^{k+1}(T^*[2]) \otimes \Sym^{\ell + 1}(E[1]) 
\]
define
\begin{align*}
(Q \psi) ((\xi_0 \otimes \cdots \otimes \xi_k) \otimes (\alpha_0 \otimes \cdots \otimes \alpha_\ell)) = \brian{finish}
\end{align*}

All that remains is to describe the $2$-shifted symplectic structure on the NQ manifold $X_E$. 
This is defined via the obvious pairing between $T^*[2]$ and $T$ together with the pairing $\langle - ,- \rangle$ on $E[1]$.
In local coordinates \brian{finish}


\subsubsection{The AKSZ theory associated to a Courant algebroid}

To any Courant algebroid $E$, we associate an NQ manifold $X_E$ which carries a $2$-shifted symplectic structure. 

The simplest type of Courant algebroid is one over a point.

\begin{Ex}\label{ex: cs}
Any Lie algebra $\fg$ together with a non-degenerate invariant pairing defines a $2$-shifted symplectic structure on the graded manifold $\fg[1]$ living over a point. 
The dg algebra of functions is the Chevalley-Eilenberg cochain complex computing Lie algebra cohomology $C^*(\fg)$. 
All such $2$-symplectic NQ manifolds over a point are of this form. 
%The resulting AKSZ theory is Chern-Simons theory. 
\end{Ex}

\subsubsection{The AKSZ theory associated to a Courant algebroid}

To any Courant algebroid $E$, we can associate an NQ manifold $X_E$ which carries a $2$-shifted symplectic structure.
In the case that the Courant algebroid is exact, call the associated $2$-shifted symplectic space $X_H$, where $H$ labels the Severa class. 

\def\sA{\mathcal{A}}
\def\d{{\rm d}}
\def\dbar{\Bar{\partial}}

The starting point of the AKSZ construction is the mapping space
\[
{\rm Map}\left((M, (\sA^*_M, \d_M)), X_E\right)
\]
where $(M, (\sA^*_M, \d_M))$ is a dg manifold with body a smooth oriented manifold $M$. 
That is, $\sA^*_M$ is a sheaf of graded commutative algebras over the de Rham complex $\Omega^*_M$ and $\d_M$ is a linear differential operator of degree $+1$ satisfying:
\begin{itemize}
\item[(1)] $\sA_M$ is concentrated in finitely many degrees;
\item[(2)] For each $k$, $\sA^k_M$ is a locally free sheaf of $C^\infty_M$-modules of finite rank;
\item[(3)] The differential $\d_M : \sA_M^* \to \sA_M^{*+1}$ is square zero differential operator making $(\sA_M , \d_M)$ into a sheaf of commutative dg algebras over the de Rham complex $\Omega^*_M$
\end{itemize}

\begin{Ex}
<<<<<<< HEAD
The most important example for us will be the source dg manifold $(M, \Omega^*_M)$, that is $\sA_M = \Omega^*_M$ equipped with de Rham differential. 
We denote this dg manifold by $M_{\rm dR}$.
\end{Ex}

\begin{Ex}
If $M$ has a para-complex structure, then $(M, \Omega^{0,*}_M)$ has the structure of a dg manifold with differential given by the para $\dbar$-operator. 
\brian{David, is this consistent notation?}
We will denote this dg manifold by $M_{p \dbar}$. 
\end{Ex}

\begin{Ex}
If $\fg$ is a Lie algebra equipped with a non-degenerate invariant pairing, and $X_E = \fg[1]$ as in Example \ref{ex: cs}, then the AKSZ theory (whose target base manifold is a point) is equivalent to Chern-Simons theory on $M$. 
\end{Ex}

\subsubsection{Dirac structures and boundary conditions}

The AKSZ formalism is a construction which produces a $(-1)$-shifted symplectic space from the data of a closed manifold and a target shifted symplectic manifold. 
There is a generalization of this construction which produces $(-1)$-shifted symplectic spaces on manifolds with boundary. 

Suppose $M$ is a manifold with boundary, and let $\cE(M)$ denote the space of fields of a classical theory on $M$. 
For us, $\cE(M)$ will be a mapping space of the form ${\rm Map}(M, X)$ where $X$ is shifted symplectic. 
Let $\cE(\partial M)$ denote the restriction of the fields to the boundary of $M$. 
For the mapping space example, this is simply the space of maps ${\rm Map}(\partial M, X)$.

In general, when $M$ is not closed, the pairing defined by the AKSZ construction will not endow $\cE(M)$ will with $(-1)$-shifted symplectic structure. 
However, the space $\cE(\partial M)$ does carry a natural ordinary ($0$-shifted) symplectic structure from the restriction of the pairing defined on $\cE(M)$. 
Morever, the natural restriction map $\cE(M) \to \cE(\partial M)$ is a Lagrangian morphism in the sense of \cite{PTVV}. 

\def\cL{\mathcal{L}}

A boundary condition is the choice of an additional Lagrangian subspace $\cL$ of $\cE(\partial M)$.
Then, one considers the intersection of the two Lagrangians $\cL$ and $\cE(\partial M)$ 


\subsection{Two dimensional TFT from GpC structures}

\brian{Recall that a generalized (para-)complex structure $\JJ$ defines a Dirac structure in $\TT$, and hence a Lagrangian inside of the $2$-shifted symplectic space $\XX_H$.  
%More generally, if $\JJ$ is a generalized complex structure in an arbitrary Courant algebroid $E$, then we obtain a $2$-shifted Lagrangian inside of the $2$-shifted symplectic space $\XX_E$. 
This Lagrangian then defines a boundary condition for the three-dimensional AKSZ theory. }

\subsubsection*{Trivial GpC structure}

\subsubsection*{(Real) Poisson $\sigma$-model}

\subsubsection*{Para-complex $A/B$-models}

We point out two special cases of the above general construction. 

\brian{The $A$-model is the usual thing.}

\begin{Def}
The {\bf para-complex $B$-model} with source a closed Riemann surface $\Sigma$ and target a para-complex manifold $X$ is the AKSZ theory with source the de Rham space $\Sigma_{\rm dR}$ and target the $1$-shifted symplectic space $T^*[1] X_{p \dbar}$:
\[
{\rm Map}\left(\Sigma_{dR}, T^*[1]  X_{p \dbar}\right) .
\]
\end{Def}

\brian{Discuss Si's work on perturbative quantization}

\hrulefill

\brian{***para Kahler example, **example where the paracomplex structures commute but are not equal (this is where people see ``twisted chiral"), para version of T-duality, **Poisson $\sigma$-model, going back and forth between generalized (para) complex.
}  

 \brian{discuss anomalies}

\subsection{Para holomorphic variants}

So far, in each of the $\sigma$-models we have discussed in the AKSZ formalism, the fields have depended only topologically on the source Riemann surface. 
There are closely related $\sigma$-models which depending {\em para-holomorphically} on the source Riemann surface that we briefly discuss. 

\end{document}

\section{The AKSZ formalism and 3d TFT}
\def\fg{\mathfrak{g}}
In this section we will show how to associate $2D$ topological theories to a generalized (para-)complex structure and relate them to the topological twists of $2D$ $(2,2)$ (para-)SUSY sigma models described in Section \ref{sec:toptwist} and in \cite{Kapustin:2004gv}. 
The key to our construction is to realize the data of a G(p)C structure in terms of its eigenbundles, which are Dirac structures $L$ in the Courant algebroid $\TT$. 
Via the AKSZ construction \cite{AKSZ}, this correspondence translates into the statement that the Dirac structure $L$ defines a topological boundary theory of a $3D$ topological theory defined by the Courant algebroid $\TT$ \cite{Roytenberg:2002nu}.

Throughout, we will use symplectic $NQ$ manifold as our model for shifted symplectic geometry. 
We will recall the general setup of associating a $3D$ topological field theory to the data of a Courant algebroid. 
For a reference, see \cite{Roytenberg:2002nu,??}.

\subsection{Courant algebroids, $3D$ TFTs, and boundary conditions}
From the point of view of topological field theory, Courant algebroids are important because they provide geometric examples of $2$-shifted symplectic spaces. 
Via the AKSZ construction, $2$-shifted symplectic spaces are the natural home for $3$-dimensional topological field theories in the BV formalism. 
\brian{give citations here.}

In \cite{Roytenberg:2002nu}, it is shown how to associate a $2$-shifted symplectic NQ manifold from the data of a Courant algebroid. 

\begin{theorem}[\cite{Roytenberg:2002nu}]
There is an equivalence between isomorphism classes of $2$-shifted symplectic NQ manifolds with body $M$ and isomorphism classes of Courant algebroids on $M$.
\end{theorem}

We briefly recall this construction. 

Recall, the data of a Courant algebroid on a manifold $M$ is a vector bundle $E$, whose sheaf of sections we denote $\cE$, together with a nondegenerate symmetric bilinear pairing $\langle -, -\rangle : \cE \otimes \cE \to C^\infty(M)$ , an anchor map $a : E \to T M$, and a bilinear operator
\[
[-,-] : \cE \times \cE \to \cE
\]
called the Courant-Dorfman bracket. 
\brian{spell out properties, find a good reference}
Denote by $a^* : T^* M \to E^*$ the bundle map dual to $a$. 

To every Courant algebroid we can define the following sequence of vector bundles on $M$
\[
T^*M \xto{a^*} E \xto{a} TM .
\]
A Courant algebroid is called {\em exact} if this sequence of bundles is exact. 

\def\Sym{{\rm Sym}}

Using this sequence, we can define the associated $NQ$ manifold $X_E$, following \cite{Ryotenberg:2002nu}.
The body of the NQ manifold is $M$, and the underlying graded vector bundle is given by $T^*[2] \oplus E[1]$. 
Define the degree $+1$ vector field 
\[
Q : \Gamma(M , \Sym \left(T^*[2] \oplus E[1]\right)^*) \to \Gamma(M , \Sym \left(T^*[2] \oplus E[1]\right)^*) 
\]
as follows: for $\psi \in \Sym^{k + \ell + 2} \left(T^*[2] \oplus E[1]\right)^*)$ and
\[
(\xi_0 \otimes \cdots \otimes \xi_k) \otimes (\alpha_0 \otimes \cdots \otimes \alpha_\ell) \in \Sym^{k+1}(T^*[2]) \otimes \Sym^{\ell + 1}(E[1]) 
\]
define
\begin{align*}
(Q \psi) ((\xi_0 \otimes \cdots \otimes \xi_k) \otimes (\alpha_0 \otimes \cdots \otimes \alpha_\ell)) = \brian{finish}
\end{align*}

All that remains is to describe the $2$-shifted symplectic structure on the NQ manifold $X_E$. 
This is defined via the obvious pairing between $T^*[2]$ and $T$ together with the pairing $\langle - ,- \rangle$ on $E[1]$.
In local coordinates \brian{finish}

The simplest type of Courant algebroid is one over a point.

\begin{Ex}\label{ex: cs}
Any Lie algebra $\fg$ together with a non-degenerate invariant pairing defines a $2$-shifted symplectic structure on the graded manifold $\fg[1]$ living over a point. 
The dg algebra of functions is the Chevalley-Eilenberg cochain complex computing Lie algebra cohomology $C^*(\fg)$. 
All such $2$-symplectic NQ manifolds over a point are of this form. 
%The resulting AKSZ theory is Chern-Simons theory. 
\end{Ex}

\subsubsection*{The $2$-shifted symplectic space of an exact Courant Algebroid}

We recall the description of the $2$-shifted symplectic NQ manifold $X_E$ associated to an exact Courant algebroid $E$.
Of course, as a vector bundle $E = \TT$, and hence as a graded manifold $X_E$ is of the form $T^*[2] T[1] M$.
Concretely, the coordinates on this manifold are given by $(x^i,v^a,p_i,\mu_a)_{i,a=1\cdots n}$, where $x^i$ are the usual coordinates on $M$ with degree $0$, $v^a$ are the corresponding ``velocities'', i.e. the fibre coordinates on $T[1]M$ with a degree $1$, and $p_i,\mu_a$ are the momenta corresponding to $x^i,v^a$ with degrees $2,1$, respectively. The canonical symplectic form
\begin{align*}
\Omega= dx^idp_i+dv^ad\mu_a,
\end{align*}
is then of degree $2$. $T^*[2]T[1]M$ is the minimal symplectic realization of the shifted bundle $(\TT)[1]M$, which is recovered by setting $p_i=0$. The symmetric part of $\Omega$ represents the pairing $\lara$ on $(\TT)M$; this is because we can identify $d\mu_a\leftrightarrow \p_i$ and $dv^a\leftrightarrow dx^i$

\brian{Severa classification}

\begin{theorem}\label{thm: severa}
The set of isomorphism classes of exact Courant algebroids is a torsor for $H^3(X)$. 
\end{theorem}

\subsubsection*{The Poisson Courant algebroid}

\brian{Both real and holomorphic versions}

\subsubsection{The AKSZ theory associated to a Courant algebroid}

To any Courant algebroid $E$, we can associate an NQ manifold $X_E$ which carries a $2$-shifted symplectic structure that we denote by $\omega_E$. 
In the case that the Courant algebroid is exact, call the associated $2$-shifted symplectic space $X_H$, where $H$ labels the Severa class. 

\def\sA{\mathcal{A}}
\def\d{{\rm d}}
\def\dbar{\Bar{\partial}}

The starting point of the AKSZ construction is the mapping space
\[
{\rm Map}\left((M, (\sA^*_M, \d_M)), X_E\right)
\]
where $(M, (\sA^*_M, \d_M))$ is a dg manifold with body a smooth oriented manifold $M$. 
That is, $\sA^*_M$ is a sheaf of graded commutative algebras over the de Rham complex $\Omega^*_M$ and $\d_M$ is a linear operator of degree $+1$ satisfying:
\begin{itemize}
\item[(1)] $\sA_M$ is concentrated in finitely many degrees;
\item[(2)] For each $k$, $\sA^k_M$ is a locally free sheaf of $C^\infty_M$-modules of finite rank;
\item[(3)] The differential $\d_M : \sA_M^* \to \sA_M^{*+1}$ is square zero differential operator making $(\sA_M , \d_M)$ into a sheaf of commutative dg algebras over the de Rham complex $\Omega^*_M$
\end{itemize}

\begin{Ex}
The most important example for us will be the source dg manifold $(M, \Omega^*_M)$, that is $\sA_M = \Omega^*_M$ with de Rham differential. 
We denote this dg manifold by $M_{\rm dR}$.
\end{Ex}

\begin{Ex}
If $M$ has a para-complex structure, then $(M, \Omega^{0,*}_M)$ has the structure of a dg manifold with differential given by the para $\dbar$-operator. 
\brian{David, is this consistent notation?}
We will denote this dg manifold by $M_{p \dbar}$. 
\end{Ex}

\begin{Ex}
If $\fg$ is a Lie algebra equipped with a non-degenerate invariant pairing, and $X_E = \fg[1]$ as in Example \ref{ex: cs}, then the AKSZ theory (whose target base manifold is a point) is equivalent to Chern-Simons theory on $M$. 
\end{Ex}

\subsubsection{Dirac structures and boundary conditions}

\subsection{Two dimensional TFT from GpC structures}
\label{sec:GpC_AKSZ}
\brian{Recall that a generalized (para-)complex structure $\JJ$ defines a Dirac structure in $\TT$, and hence a Lagrangian inside of the $2$-shifted symplectic space $\XX_H$.  
%More generally, if $\JJ$ is a generalized complex structure in an arbitrary Courant algebroid $E$, then we obtain a $2$-shifted Lagrangian inside of the $2$-shifted symplectic space $\XX_E$. 
This Lagrangian then defines a boundary condition for the three-dimensional AKSZ theory. }

\subsubsection*{Trivial GpC structure}

\subsubsection*{(Real) Poisson $\sigma$-model}

\subsubsection*{Para-complex $A/B$-models}

We point out two special cases of the above general construction. 

\brian{The $A$-model is the usual thing.}

\begin{Def}
The {\bf para-complex $B$-model} with source a closed Riemann surface $\Sigma$ and target a para-complex manifold $X$ is the AKSZ theory with source the de Rham space $\Sigma_{\rm dR}$ and target the $1$-shifted symplectic space $T^*[1] X_{p \dbar}$:
\[
{\rm Map}\left(\Sigma_{dR}, T^*[1]  X_{p \dbar}\right) .
\]
\end{Def}

\brian{Discuss Si's work on perturbative quantization}

\hrulefill

\brian{***para Kahler example, **example where the paracomplex structures commute but are not equal (this is where people see ``twisted chiral"), para version of T-duality, **Poisson $\sigma$-model, going back and forth between generalized (para) complex.
}  

 \brian{discuss anomalies}

\subsection{Para holomorphic variants}

So far, in each of the $\sigma$-models we have discussed in the AKSZ formalism, the fields have depended only topologically on the source Riemann surface. 
There are closely related $\sigma$-models which depending {\em para-holomorphically} on the source Riemann surface that we briefly discuss. 
